%------------------------------------------------------------------------------
% Tema 12. Vectors no virals
%------------------------------------------------------------------------------
\section{Vectors no virals}
\label{sec:vectors-no-virals}

Es treballa normalment amb plasmidid però també es fan servir siRNA o oligonucleòtids.

\subsection{Captació cel·lular de DNA nu}
\label{sec:capt-cel.l-de}

S'usa DNA nu quan el cDNA és massa gran o bé quan la utilització de virus implica complicacions immunològiques significatives. El punt clau en l'ús de DNA nu és la purificació, de manera que no quedin pràcticament restes de mureïna, p.e.

Les molècules d'àcids nucleics poden ser captades per la cèl·lula amb baixa eficiència \textit{in vivo}. Aquesta propietat típicament es perd quan les cèl·lules es separen del teixit i es mantenen en cultiu.

És comú a les cèl·lules dels tres llinatges: endoderma (per exemple, els hepatòcits), mesoderma (per exemple, múscul), i l'ectoderma (per exemple, la pell).

Les vies d'administració \textit{in vivo} poden ser:
\begin{itemize}
\item Després d'injecció intravascular l'àcid nucleic entra a fetge i múscul principalment. Més del 60\% de les molècules d'àcids nucleics injectades es capten al fetge. El ronyó filtrarà els oligonucleòtids i s'eliminaran per la orina. La idea d'arribar al fetge pot ser bona per expressar una proteïna que després es secreta en sang. El direccionament al fetge en canvi no serà una bona estratègia si es vol bloquejar l'expressió d'un gen no específic del fetge.

\item Altres vies d'administració local habituals són: intramuscular, intratumoral i cutània. També es pot usar la via intranasal, per vehicular fàrmacs o intraocular per tractar malalties de la retina.
\end{itemize}

L'administració oral queda descartada ja que al tracte gastrointestinal hi ha molt enzims nucleases: DNases i RNases que degradarien qualsevol DNA/RNA que s'administrés.

L'administració de DNA plasmídic nu, intramuscular o a la pell, és d'ús comú per a la immunització genètica, és a dir per les vacunes de DNA. El plàsmid conté un gen que codifica per una proteïna immunogènica d'un patogen de manera que la vacunació no comporti el risc de possibles efectes adversos.

La injecció intramuscular de DNA plasmídic resulta en nivells alts d'expressió gènica amb un pic després d'una setmana que es detectable fins a un mes després.

La injecció de DNA plasmídic a cèl·lules de la pell es pot realitzar per un procediment de tatuatge que posa en moviment agulles (pujant i baixant) perquè la gota d'una solució de DNA dipositada sobre la pell penetri.

La producció de la proteïna antigènica és molt ràpida i transitòria (pic després de sis hores i desapareix en quatre dies). La dosi de DNA és baixa però la resposta immune humoral és millor que l'obtinguda per injecció intramuscular.

La tecnologia d'injecció sense agulla (Bioject) funciona al forçar la medicació líquida a alta velocitat a través d'un petit orifici que es porta a terme contra la pell. El diàmetre de l'orifici és més petit que el diàmetre d'un cabell humà. Això crea un corrent ultrafina de fluid a alta pressió que penetra a la pell, teixit adipós subcutani o múscul sense necessitat d'utilitzar una agulla. S'ha desenvolupat en veterinària.

Donades les dificultats del DNA/RNA per introduir-se a la cèl·lula i arribar intacte al nucli, ens valem d'una sèrie de mètodes per facilitar aquests processos que podem diferenciar per la seva naturalesa entre mètodes fisicoquímics o no vírics i vírics.

Els mètodes físics propulsen mitjançant forces mecàniques el DNA a l'interior cel·lular o/i nucli.

Els mètodes químics compacten el DNA amb d'altres molècules que faciliten la interacció amb la cèl·lula, captació i transferència al nucli.

\subsection{Mètodes químics}
\label{sec:metodes-quimics}
Aquests mètodes comprenen:
\begin{itemize}
\item Partícules inorgàniques: Precipitació en fosfat de calci.
\item Complexes lipídics catiònics: lipoplexes
\item Polímers catiònics: Poliplexes
\item Pèptids que penetren cèl·lules
\item Vectors proteics modulars
\end{itemize}

\subsubsection{Liposomes catiònics}
\label{sec:liposomes-cationics}

Les membranes cel·lulars tenen una gran quantitat de càrregues negatives degut a la presència de fosfolípids típics de la bicapa lipídica i de proteoglicans associats a proteïnes de membrana i a la matriu extracel·lular. De manera natural, el DNA (carregat de fosfats) es repulsaria electrostàticament.

Estan compostos per liposomes de lípids amfipàtics carregats positivament. Els liposomes comercials tenen de l'ordre de 250 nm de diàmetre (100 vegades més petits que una cèl·lula típica).

L'estructura bàsica de la molècula és: grup catiònic-espaiador-lípid. El component lipídic pot ser diacilglicerol o colesterol entre d'altres, i és la part hidròfoba que permet l'auto-agregació com en la bicapa lipídica de la membrana cel·lular. El grup catiònic, és una amina terciària, polilisina, o altres, i permet la condensació amb els àcids nucleics per interaccions electrostàtiques amb els grups fosfat carregats negativament. Els liposomes catiònics acostumen a incloure lípids neutres (colesterol) que tenen funció coadjuvant d'estabilitzar la micel·la.

Molècules molt emprades comercialment són la L-dioleil fosfatidil etanolamina (DOPE) o el 1,2-dioleil-3- trimetilammoni-propà (DOTAP). 

Espontàniament els lípids catiònics formen complexos amb el DNA/RNA a causa de la interacció iònica, anomenats lipoplexes. Els lipoplexes que s'obtenen són aniònics, neutres o catiònics depenent de la proporció de les càrregues catiòniques dels lípids i els grups aniònics del DNA fosfodièster. La preparació de lipoplexes amb un lleuger excés de càrrega positiva confereix major eficàcia de transfecció.

\paragraph{Estructura}

Les estructures dels complexes dels lípids catiònics amb àcids nucleics són de dos tipus:

\begin{itemize}
\item Una estructura multilaminar amb monocapes d'àcid nucleic intercalat entre les membranes de lípids catiònics. Es capten per endocitosi.

\item Una estructura hexagonal invertida amb l'àcid nucleic encapsulat dins de micel·les cilíndriques. Entra a la cèl·lula per fusió amb la membrana.
\end{itemize}

Les propietats fisicoquímiques, a escala micro i macroscòpica dels conjunts supramoleculars formats pels lipoplexes són paràmetres que regeixen el mecanisme i eficiència de la transfecció.

Els complexos d'àcid nucleic-lípid denominats lipoplexes es poden captar activament per les cèl·lules eucariotes per mitjà d'endocitosi.

La capacitat d'un lípid per destruir endosomes és una de les principals característiques d'un reactiu de transfecció. L'estructura endosomal es destrueix mitjançant l'augment de la pressió osmòtica creada per l'acció de tamponament dels lípids dins dels endosomes i per la fusió del lípid amb la membrana endosomal.

La taxa de divisió cel·lular és crítica en la transfecció d'ADN i ha de ser el més alt possible per a la transfecció eficient. El DNA que s'introdueix en el citosol no pot penetrar la membrana nuclear. L'accés al nucli és, per tant, només possible si la membrana nuclear es dissol durant la mitosi.

\paragraph{Captació cel·lular}
Els lipoplexes carregats positivament s'adsorbeixen a la superfície cel·lular.

En el cas dels lipoplexes captats a través d'endocitosi, el baix pH dels endosomes causa la fusió entre les membranes dels endosomes i els lipoplexes. El DNA s'allibera al citosol lliure o associat amb els lípids.

La fusió del lipoplex amb la membrana plasmàtica fa que el lipoplex i el DNA lliure s'alliberi directament en el citosol.

Per acomplir la funció els complexos s'han de dissociar i alliberar l'àcid nucleic al nucli. Els lipoplexes que arriben al nucli es transcriuen molt pobrament.

\paragraph{Aplicacions \textit{in vitro}}
S'han utilitzat profusament durant més de 10 anys en aplicacions per la transfecció in vitro.

La major part de l'àcid nucleic captat amb els liposomes catiònics per la cèl·lula és degradat. Es calcula que 1 de cada 1000 molècules del gen que entren a la cèl·lula arriben a expressar-se. S'estima que l'eficiència de transfecció in vitro és d'un 30-80\% per a la transfecció transitòria en funció del lípid i tipus cel·lular i d'un 0.1\% per la transfecció estable.

Els liposomes catiònics presenten citotoxicitat, ja que els lípids degradats són tòxics.

\paragraph{Aplicacions \textit{in vivo}}
L'ús d'aquestes molècules encara no s'ha traduït a l'aplicació en éssers humans via sistèmica per la seva baixa eficiència de transfecció in vivo i alguns problemes de toxicitat. S'estan desenvolupant aplicacions per tractar cèl·lules epitelials pulmonars mitjançant aerosols en malalties respiratòries (com la fibrosi quística).

Hi ha la necessitat d'identificar noves estructures lipídiques catiòniques que es poden basar en compostos naturals, coneguts per la seva seguretat. Per exemple components lipofílics derivats dels components de les membranes biològiques (fosfolípids o colesterol).

\subsubsection{Polications}
\label{sec:polications}

Els polications poden ser polímers lineals, brancats o dendrímers amb pesos moleculars que tenen mides variables en el rang 22 a 800 kDa.

Polications molt utilitzats comercialment són la polietilenimina (PEI) lineal o brancada, la poli(amido amina) (PAMAM) en dendrímers o el dietilamino-dextrà. Pot ser el quitosan també (polisacàrid de crustacis.)

Els polications formen complexos espontàniament i per interacció electrostàtica amb els àcids nucleics carregats negativament (poliplexes) amb estructura variada i diferent grau de compactació.

Normalment s'utilitzen poliplexes amb càrrega neta positiva.

\paragraph{Captació cel·lular}
Els poliplexes, mitjançant les seves càrregues positives, s'adsorbeixen a la superfície de la cèl·lula.

Els poliplexes s'internalitzen per endocitosi disruptora de les vesícules d'endocitosi. Es dissocien els complexos lligand-receptor i el receptor es recicla a la superfície cel·lular. S'alliberen els poliplexes i l'àcid nucleic lliure dels endosomes al citoplasma.

L'àcid nucleic ha de travessar la membrana nuclear per arribar al nucli. Els poliplexes sembla que també interaccionen amb la membrana nuclear per finalment alliberar el complex al nucli. S'ha demostrat que el DNA es pot transcriure sense dissociar-se del policatió.

Els polications que tenen ponts disulfurs es poden reduir a la cèl·lula, trencant el polímer i això redueix la toxicitat i afavoreix l'alliberament de l'àcid nucleic.

\paragraph{Aplicacions \textit{in vitro}}
L'eficiència pot ésser molt alta per a la transfecció transitòria a cèl·lules en cultiu, entre un 50-80\% en funció del tipus cel·lular, però en canvi la freqüència d'integració és gairebé nul·la, fenomen que es relaciona amb un defecte de desacoblament del poliocatió.
Els polications són citotòxics si no són biodegradables.

S'estan buscant polications biodegradables.

\paragraph{Aplicacions \textit{in vivo}}
El PEI administrat in vivo en ratolins s'ha observat que es distribueix, encara que molt pobrament. Només un 10\% es captat principalment al fetge, ronyó i melsa on el transgen s'hi expressa transitòriament.

Els dendrímers de PAMAM es distribueixen principalment al fetge (> 60\%) i s'eliminen pel ronyó.

Els poliplexes presenten toxicitat en causar l'agregació de cèl·lules sanguínies i no es poden utilitzar via sistèmica en assaigs clínics. Es poden usar localment.

\paragraph{Polications biodegradables}
S'estan desenvolupant poliplexes que tinguin baixa toxicitat (pel fet de ser biodegradables i generar productes de degradació petits i solubles fàcilment eliminables), però que mantinguin una elevada eficiència de transfecció.

Alguns exemples són:
\begin{itemize}
\item Anàleg biodegradable de la polí-lisina poly[$\alpha$-(4-aminobutyl)-L-glycolic acid] (PAGA) en que els enllaços amida són reemplaçats per enllaços èsters.

\item PEI degradable per la combinació de PEI de baix pes molecular amb molècules d'unió degradables intracel·lularment (per hidròlisi a pH baix endosomal, degradació enzimàtica o degradació reductiva pel glutatió al citosol).
\end{itemize}

\subsubsection{Pèptids de penetració cel·lular}
\label{sec:pept-de-penetr}
Els CPPs són pèptids de 7 a 30 aminoàcids, solubles en aigua, polibàsics o polibàsics i en part hidrofòbics, i de càrrega neta positiva a pH fisiològic, els quals són capaços de dirigir la captació a la membrana cel·lular, d'ells i de la molècula que portin associada.

Es poden classificar:
\begin{itemize}
\item Segons les seves propietats en:
	\begin{itemize}
	\item Policatiònics (contenen clústers de polí-arginina)
	\item Amfipàtics primaris (> 20 aminoàcids, amb residus hidrofòbics i hidrofílics al llarg de la seva
	estructura primària)
	\end{itemize}

\item Segons el seu origen en:
	\begin{itemize}
	\item Derivats de proteïnes naturals com TAT (derivat del VIH) i penetratina (fragment que deriva d'antennapaedia, un TF de Drosophila).
	\item Pèptids quimèrics que poden contenir dos o més motius d'altres pèptids (exemple, transportan que deriva de mastoparan i galanina i el seu anàleg més curt TP10).
	\item Pèptids sintètics com la polí-arginina.
	\end{itemize}
\end{itemize}

Es poden aplicar per transportar una àmplia gamma de biomacromolècules (pèptids, proteïnes i àcids nucleics), molècules petites, fàrmacs, agents de contrast d'imatges i nanopartícules de diferent naturalesa. Poden unir àcids nucleics de gran tamany.

L'acoblament dels àcids nucleics a CPPs es pot fer de manera covalent o no covalent.

En el cas d'acoblament covalent, els àcids nucleics es conjuguen als CPPs per enllaç quimíc, el més habitual és un enllaç disulfur, el qual es dissocia en l'ambient reductor intracel·lular i així allibera l'àcid nucleic. La unió covalent és avantatjosa pels oligonucleòtids, però no és adequat per DNA i siRNA.

El complex no covalent es basa en les interaccions electrostàtiques i hidròfobes entre CPPs carregats positivament i àcids nucleics aniònics. Els complexos poden tenir diferent grandària i estabilitat. Aquesta estratègia és aplicable a oligonucleòtids, DNA i siRNA.

L'adsorció a la membrana plasmàtica per la majoria dels complexos CPP-àcid nucleic es produeix mitjançant forces electrostàtiques de la càrrega positiva dels CPPs.

\paragraph{Mecanismes de captació}
Els mecanismes d'internalització són: la penetració directa a través de la membrana o vies endocítiques. Això es funció de factors com la seqüència, grandària, càrrega, estructura, estabilitat del pèptid i l'entitat i característiques de la càrrega i l'associació covalent o no covalent.

Per acomplir la funció biològica els complexos endocitats han d'escapar de l'endosoma.

El complex o l'àcid nucleic es pot localitzar intracel·lularment al nucli o citoplasma en funció de les seves característiques.

Per acomplir la funció biològica els complexos s'han de dissociar i alliberar l'àcid nucleic.

No es poden administrar oralment i tenen una vida mitjana curta.

En funció de la càrrega, si són proteïnes, poden generar una resposta immune.

\paragraph{Aplicació in vitro}
S'ha demostrat el seu funcionament, però són d'aplicació limitada. Un dels inconvenients és la pobra translocació nuclear. Presenten baixa toxicitat en concentracions eficaces.

\paragraph{Aplicació in vivo}
S'ha usat in vivo i a nivell preclínic i clínic. No obstant, en la majoria d'aquests usos la molècula de càrrega és un pèptid o proteïna.

L'aplicació més usada amb àcids nucleics es la de CPPs-ONs (oligonucleòtids) o CPPs- siRNAs per la inhibició de l'expressió gènica.

Entre diverses vies d'administració, l'aplicació dels CPPs en sistemes d'administració transdèrmica i tòpica han centrat l'atenció de la industria cosmètica i farmacèutica.

\subsubsection{Vectors proteics modulars multifuncionals}
\label{sec:vect-prot-modul}
Són proteïnes quimerètiques que inclouen dominis peptídics d'utilitat per la unió al DNA i a la cèl·lula diana i per facilitar el procés de captació cel·lular.

La proteïna es pot produir com a proteïna recombinant mitjançant la fusió directa de diversos dominis funcionals o bé es poden conjugar químicament dominis funcionals i la proteïna.

Els diferents dominis són:
\begin{itemize}
\item Unió i condensació d'àcid nucleic: pèptids bàsics o polications. \item Direccionament i unió cel·lular: dominis proteics reconeguts per un receptor a la membrana cel·lular.
\item Disrupció endocítica: polications o pèptids de penetració cel·lular, entre d'altres. Sovint, toxines d'origen bacterià.
\item Translocació al nucli: pèptids amb senyals de localització nuclear.
\end{itemize}

Està limitada pel pes, càrrega elèctrica i immunogenicitat de la molècula.

Un risc és la interacció dels vectors amb components de la sang i la inducció de trombes.

Ex: polilisina conjugada amb manosa, serà reconeguda per receptors de manosa que es troben als macròfags.

La síntesi es pot fer per expressió heteròloga de la proteïna de fusió en bacteris o bé per conjugació dels diferents dominis.


\subsection{Mètodes físics}
\label{sec:metodes-fisics}

Hi ha els següents:
\begin{itemize}
\item Hidroporació
\item Electroporació
\item Magnetofecció
\item Sonoporació
\item Làser
\item Pistola gènica
\item Microinjecció
\end{itemize}

\subsubsection{Hidroporació}
\label{sec:hidroporacio}

El sistema utilitza la força física generada per la injecció ràpida d'un gran volum de solució que conté l'àcid nucleic en un vas sanguini. La pressió hidrostàtica creada millora la permeabilitat de l'endoteli i de la membrana plasmàtica cel·lular i això permet que la cèl·lula capti els àcids nucleics presents al medi extracel·lular. Les pressions i forces tallants generades poden conduir al dany de la membrana transitori o permanent, i també desencadenar una sèrie d'esdeveniments cel·lulars per segellar o reorganitzar la membrana danyada.

El principal teixit on arribaran serà al fetge, però si la injecció es fa en un lloc determinat es pot concretar més la localització on es vol introduir el DNA.

Els principals avantatges del mètode són la simplicitat i manca de toxicitat addicional.

La transferència d'àcids nucleics hidrodinàmica es pot adaptar fàcilment a espècies animals petites i grans i a humans, només requereix l'optimització de les tècniques de col·locació de la caracterització i torniquet i dels paràmetres de lliurament (volum i velocitat d'injecció). Normalment s'injecten 1-1,5 mL.

El volum necessari per treballar en pacients fa la tècnica inviable.

Una solució de pDNA es perfon en una vena del membre (anterògrada i es col·loca un torniquet proximal, el qual permet un augment transitori de la pressió vascular després de la injecció. El pDNA només és extravasat a les zones de major pressió i això limita la transferència del pDNA a l'extremitat aïllada. El procediment complet es pot dur a terme en uns 5-10 minuts.

\subsubsection{Electroporació}
\label{sec:electroporacio}
Polsos elèctrics d'alta intensitat i curta duració (tensions 250-1250 V / cm, 20-100 mil·lisegons) són capaços de permeabilitzar la membrana plasmàtica cel·lular, de manera que els DNAs/RNAs presents en els medis extracel·lulars entren a la cèl·lula directament en el citosol. La recuperació de la membrana cel·lular per segellar-la o reorganitzar-la es produeix en minuts.

Eventualment, aquests polsos intensos poden induir la fusió de les cèl·lules que estan en contacte: electrofusió.

Polsos elèctrics de baixa intensitat i llarga duració poden dirigir el moviment de partícules carregades (que inclouen l'àcid nucleic) entre els elèctrodes, això pot forçar que les cèl·lules, quan es situen en mig dels elèctrodes, siguin creuades pel DNA.

\paragraph{Aplicació in vitro}
Els sistemes comercials estan dissenyats majoritàriament per cèl·lules en suspensió. Estan compostos de fonts d'alt voltatge i cubetes adients que incorporen els elèctrodes on es situa la suspensió de cèl·lules en un medi amb àcid nucleic. S'apliquen polsos intensos.

La composició en ions del tampó d'electroporació determina la resistència elèctrica. La composició de soluts, acostuma a ser isotònica o hipotònica, la qual afavoreix la captació de líquid i inflor cel·lular, i d'aquesta manera la disrupció de la membrana plasmàtica.

L'eficiència de la captació de molècules és variable (<50\%) i també el nombre de molècules captades. Les molècules lineals es capten millor.

Quan es vol induir la fusió cel·lular s'apliquen addicionalment procediments que afavoreixin el contacte cèl·lula-cèl·lula (ex. dielectroforesi o centrifugació).

\paragraph{Aplicació in vivo}
Cal injectar el DNA dissolt en solució salina en el teixit i sobre aquest teixit aplicar microelèctrodes que descarregaran el corrent generat per equips especialitzats. L'electroporació s'aconsegueix amb la combinació alternada de polsos d'alt-voltatge (ex. 800 volts/cm, 100 microsegons), els quals permeabilitzen la cèl·lula, i de baix voltatge (ex. 80 volts/cm, 100 mil·lisegons), els quals fan migrar el DNA (electroforesi) i ajuden a la seva distribució.

S'ha aplicat amb èxit per transfectar principalment múscul esquelètic, pell i tumors.

\subsubsection{Magnetofecció}
\label{sec:magnetofeccio}
Les nanopartícules magnètiques (NPMS) es poden guiar mitjançant un camp magnètic a les cèl·lules i teixits diana.

Diverses molècules es poden vincular a les NPMs per tal d'alliberar la seva càrrega molecular associada a la cèl·lula a través de la internalització per la via endocítica.

La composició de les NPMs és variada, però la més freqüent és la d'òxids de ferro estabilitzats amb un policatió que uneix l'àcid nucleic per atracció de càrrega.

\paragraph{Aplicació in vitro}
Les MNPs que transporten els àcids nucleics afegides al medi de cultiu es sedimenten a la superfície de les cèl·lules en minuts.

S'aplica un camp magnètic polsat per induir el moviment de les partícules a l'interior cel·lular.

\paragraph{Aplicació in vivo}
Les MNPS estan sent utilitzades clínicament com a agents de contrast
per ressonància magnètica nuclear (RMN).

La biodistribució de les MNPs depèn de les seves propietats fisicoquímiques, com la grandària, la hidrofília i càrrega superficial. No obstant, després de l'administració intravenosa (xeringa rosa), les MNPS principalment s'acumulen al fetge, la melsa i els ganglis limfàtics (zones de color rosa), encara que són presents en altres òrgans, com pulmons o cervell (punts de color rosa).

Si les MNPs són guiades per mitjà d'un camp magnètic, es concentren en l'àrea d'interès. Així, després de l'administració intratumoral i l'exposició a l'imant (xeringa verda), les MNPs es concentren al tumor (verd).

\subsubsection{Sonoporació}
\label{sec:sonoporacio}
Els ultrasons són ones sonores amb freqüències superiors al rang audible humà 20 kHz (el rang audible en humans és de 20 Hz a 20 kHz). Es propaguen a través de la matèria a partir d'un objecte que vibra com una ona de pressió i poden produir escalfament de la matèria. Les ones en polsos causen principalment vibració i les contínues escalfament.

Els ultrasons poden pertorbar les membranes cel·lulars i causar porus transitoris en la membrana plasmàtica, cosa que facilita l'entrada de molècules en la cèl·lula. L'energia necessària i freqüències (20-50 kHz) per aconseguir-ho aplicada de manera continua poden danyar extensament la cèl·lula.

Per això, els ultrasons es combinen amb micro(nano)bombolles, les quals han estat utilitzades com a agents de contrast d'ultrasons. Les micro(nano)bombolles es destrueixen per exposició a ultrasò (implosió) i generen microcorrents (de fins 700 m/s) que donen lloc a forces tallants en les cèl·lules i formen porus en les membranes cel·lulars. Això millora l'eficiència del lliurament de molècules. S'apliquen normalment ultrasons en mode de polsos de freqüències d'1 a 3 MHz i intensitats d'1 a 2 W/cm2. És possible usar, en lloc de nanobombolles, micro(nano)partícules degradables que també es destrueixen per l'ultrasò.

El dany de la cèl·lula depèn de la intensitat de l'ultrasò, la freqüència i temps d'aplicació, la concentració de micro(nano)bombolles i el tipus cel·lular. El dany cel·lular pot ser reversible.

Atès que l'ultrasò pot penetrar el teixit tou i es pot aplicar a una àrea específicament, es pot utilitzar per a la transferència gènica no invasiva en cèl·lules d'òrgans interns. Es combina amb les micro(nano)bombolles administrades via sanguínia o localment.

La limitació principal és la baixa eficiència del procés

\paragraph{Micro(nano)bombolles}
Les microbombolles mesuren entre 1 i 8 $\mu m$ i les nanobombolles són inferiors a 1 $\mu m$.

Els recobriments son fins i poden ser de diferent naturalesa (proteïnes, lípids (liposomes) o polímers). El contingut al nucli pot ser aire o gasos específics.

\paragraph{Micro(nano)partícules biodegradables}
L'àcid poli-làctic-co-glicòlic (PLGA) o les partícules de poliestirè són nanopartícules susceptibles a la destrucció per ultrasò, el qual, a més, les impulsa contra la cèl·lula i en millora el lliurament de molècules associades. PLGA ha estat aprovat per l'EUA Food and Drug Administration (FDA) per l'administració de fàrmacs.

\paragraph{Aplicació in vitro}
S'aplica a cèl·lules, en suspensió o cultivades sobre una superfície, immerses en un medi amb micro(nano)bombolles/micro(nano)partícules i DNA. El temps d'aplicació es de l'ordre de segons.

\paragraph{Aplicació in vivo}
Inicialment es va aplicar a òrgans i teixits dels quals s'han obtingut imatges amb l'ultrasò (cor, múscul esquelètic i ronyó) però també s'ha estès a d'altres òrgans.

S'injecten, en el vas sanguini apropiat que irriga un teixit o bé intrateixit, les micro/nanobombolles (o nanopartícules) amb el DNA i s'aplica l'ultrasò a una zona delimitada.

\subsubsection{Làser}
\label{sec:laser}

El làser és un dispositiu que emet llum (radiació electromagnètica) a través d'un procés òptic d'amplificació basat en l'emissió estimulada de fotons. La radiació electromagnètica pot ser de diverses freqüències, no només la llum visible, també làser infraroig, làser ultraviolat, entre d'altres.

Les interaccions làser-cèl·lules inclouen efectes fotoquímics, fototèrmics i fotomecànics.

\paragraph{Transfecció òptica (optoinjecció)}
Un raig làser altament concentrat es pot utilitzat com un microfeix per produir forats submicromètrics a la membrana cel·lular que facilitin la captació d'àcids nucleics exògens en cèl·lules cultivades.

S'aconsegueix amb una sola exposició a feixos de llum de la zona propera a l'infraroig (800 a 1000 nm) i una duració del pols de femtosegons o nanosegons a una energia determinada. Quan menor és la duració del pols, menor és el diàmetre del forat.

\paragraph{Aplicació in vitro}
Una sola cèl·lula és directament irradiada pel feix de làser enfocat i aquesta cèl·lula capta molècules exògenes. Es poden fer unes 1000 irradiacions/hora.

Les cèl·lules (-) són irradiades amb el feix de làser enfocat en presència del DNA contingut en el medi extracel·lular, el qual s'introduirà a través d'un orifici molt petit transitori fet a la membrana plasmàtica pel làser. Algunes d'aquestes cèl·lules irradiades quedaran transfectades (+).
