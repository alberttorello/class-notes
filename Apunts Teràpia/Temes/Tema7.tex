%------------------------------------------------------------------------------
% Tema 7. Introducció a la teràpia gènica
%------------------------------------------------------------------------------
\section{Introducció a la teràpia gènica}
\label{sec:intr-la-terap}

S'entén per teràpia gènica una teràpia que té per objectiu usar àcids nucleics per tractar malalties.

La teràpia gènica es pot utilitzar per:
\begin{itemize}
\item Malalties infeccioses
\item Malalties monogèniques hereditàries
\item Càncer
\item Malalties autoimmunitàries
\item Malalties multifactorials
\end{itemize}

L'addició gència intenta compensar una deficiència genètica introduint una versió correcta del gen. Els àcids nucleics es poden injectar directament a les cèl·lules o bé introduint-lo per mètodes fisicoquímcis i/o virals. Moltes estratègies buscant introduir el gen al nucli de les cèl·lules, on es pot expressar de forma estable o transitòria. Altres mètodes busquen la integració del gen als cromosomes. El gen es pot mantenir al nucli de forma episòmica.

Si el material genètic s'integra als cromosomes cel·lulars, s'introdueix a cèl·lules replicatives aquest passarà a cèl·lules filles. Si el material genètic roman en un episoma, quan la cèl·lula es divideixi només una cèl·lula mantindrà l'episoma.

A priori, la substitució del gen defectiu seria més desitjable que l'addició gènica. Mitjançant el reemplaçament del gen defectiu amb el gen corrector al mateix lloc del cromosoma, on el gen normal funciona, els científics esperen assolir la regulació natural del gen, és a dir, que el gen corrector s' expressi al moment i en la quantitat adient.

Les tècniques de teràpia gènica més emprades actualment no són del tot eficients. No poden assolir control sobre la integració del gen al lloc original cromosòmic, tot i que ja comencen a desenvolupar-se noves tècniques que permeten fer gene targeting. Encara menys es pot aconseguir que la informació genètica entri a totes les cèl·lules a les que va dirigida.

Les diferents estratègies de teràpia engloben:
\begin{itemize}
\item Teràpia gènica clàssica
  \begin{itemize}
  \item Producció d'un producte gènic que falta en el pacient
  \item Eliminació de cèl·lules (gens suïcides)
  \item Activació de les cèl·lules del sistema immunitari
  \end{itemize}

\item Teràpia gènica no convencional
  \begin{itemize}
  \item Inhibició de l'expressió d'un gen
  \item Correcció d'un defecte gènic per tal de restablir l'expressió normal del gen
  \end{itemize}
\end{itemize}

Les HSC poden ser manipulades \textit{ex vivo} i després tornar-les a implantar. Els hepatòcits també poden patir aquest procés i introduir els hepatòcits via vena porta. Es pot fer a la dermis.

No hi ha vectors vírics capaços de carregar un gen humà sencer (promotor + gen; exons i introns). El que es fa és introduir el cDNA; el que passa és que tenen un nivell d'expressió baix i la solució és col·locar un intró al 3'-UTR per tal que sigui reconegut per spliceosoma i exportat al citoplasma perquè sigui traduït. L'ús de promotors de retrovirus/lentivirus no és recomanable ja que la cèl·lula silencia els promotors (situats a les LTR).

Els virus tenen tropisme. Els miotubs accepten DNA nu. Els oligos poden ser captats directament per la cèl·lula i amb les modificacions pertinents escapar de les exo/endonucleases.