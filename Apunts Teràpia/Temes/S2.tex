%------------------------------------------------------------------------------
% S2. Bioimpressió en 3D
%------------------------------------------------------------------------------
\section{Bioimpressió en 3D}
La bioimpressió en 3D és el procés de dipositar materials vius en una estructura tridimensional.

Les matrius de suport poden ser de colàgen o de tipus gel. La matriu ha de desaparèixer amb el temps.

\begin{itemize}
\item Inkjet: gotes amb cèl·lules
\item Làser: 
\item Microextrusió: es dóna pressió al material amb pistons, aire a pressió o un cargol. La més freqüent és la d'aire a pressió.
\end{itemize}

Utilitat de la bioimpressió:
\begin{itemize}
\item Models cel·lulars més reals
\item Alternativa a l'experimentació animal
\end{itemize}

Tipus de cultius en 3D:
\begin{itemize}
\item Explants
\item Transwells
\item Esferoides
\item Scaffolds: Estructura polimèrica de material biocompatible sobre el qual es sembren cèl·lules.
\end{itemize}

Els problemes són la distribució de diferents cèl·lules i la vascularització del sistema.

Condicions per la bioimpressió:
\begin{itemize}
\item Disseny del model
\item Bioimpressora
\item Condicions estèrils
\item Hidrogels
\item Temps
\end{itemize}

% Nature Biotechnology, 2016 A 3D bioprinting system to produce large-scalce human organs

