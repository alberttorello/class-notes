%------------------------------------------------------------------------------
% Tema 14. Vectors virals
%------------------------------------------------------------------------------
\section{Vectors virals}
\label{sec:vectors-virals}

\subsection{Introducció}
\label{sec:introduccio}

Els vectors virals s'usen en:
\begin{itemize}
\item Investigació bàsica: transduir gens.
\item Teràpia gènica
\item Vacunes: Mateix concepte que vacunes de DNA.
\end{itemize}

S'han de tenir en compte diferents consideracions a l'hora de fer servir vectors virals:
\begin{itemize}
\item Seguretat: En general, han de ser virus no replicatius. S'usen virus replicatius condicionals en teràpies contra el càncer.
\item Toxicitat: Ha de ser mínima per disminuir l'afectació cel·lular.
\item Estabilitat: Expressió a llarg termini per evitar la inestabilitat genètica. En el cas de les malalties monogèniques, el que es vol és que el transgen s'expressi de manera estable i continuada.
\item Especificitat cel·lular:
\end{itemize}

Els virus més utilitzats són els adenovirus i els retrovirus.

% Taula característiques virus

Retrovirus: Cèl·lules en divisió
Lentivirus: Cèl·lules quiescents

És important tenir en compte la immunogenicitat dels virus.

Els lentivirus no provoquen tanta mutagènesi insercional. 

Virosoma: bicapa lipídica amb glicoproteïnes víriques.

Els virus usen receptors cel·lulars per entrar; a vegades moltes d'aquestes proteïnes es troben sobreexpressades en càncer.

Per evitar la neutralització del virus, es poden canviar les proteïnes de la càpside per les d'una altra espècie. També es poden canviar epítiops de proteïnes de superfície o intercanviar-les amb virus d'una altra família.

L'expressió del transgen es pot controlar amb l'ús de promotors específics, que estiguin sobreexpressats en càncer.

Els miRNA expressats en cèl·lules normals poden eliminar genoma viral i augmentar l'especificitat.

\subsection{Adenovirus}
\label{sec:adenovirus}

Són virus nus de forma icosaèdrica. % Estructura proteica

L'entrada és mitjançant el receptor CAR (\textit{coxsackie virus  and adenovirus receptor}) o el CD46. El domini knob interacciona amb CAR i les integrines interaccionen amb la base del pentó. Es forma una vesícula de clatrina. Disminueix el pH de l'endosoma i s'allibera el virus. El virus es transporta al nucli mitjançant microtúbuls i entra a l'espai nuclear. Fase early i late separades per la replicació del virus.

Un cop es produeixen les proteïnes late, es formen els virions i s'alliberen a l'exterior. Tot el procés dura entre 36 i 48h.

% Apunts VIRO

E1A i E1B s'eliminen quan s'usen adenovirus com a vectors per teràpia. Perd la capacitat de replicació. Es col·loca el cDNA al genoma de l'adenovirus i es posa tot sota el control d'un promotor fort. Això s'introdueix en cèl·lules empaquetadores: contenen gens de la replicació que s'han eliminat al vector.

