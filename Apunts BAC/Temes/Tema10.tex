%------------------------------------------------------------------------------
% Tema 10. El fetge
%------------------------------------------------------------------------------
\section{El fetge}
\label{sec:el-fetge}

\subsection{El fetge}
\label{sec:el-fetge-1}

\subsubsection{Circulació hepàtica}
\label{sec:circulacio-hepatica}
La sang entra el fetge per la via porta (intestí) i a través de
l'artèria hepàtica (sistèmica). Surt del fetge a través de la vena
hepàtica, que conflueix a la vena cava.

Els nutrients de la dieta arriben via vena porta, arriben als
hepatòcits. Al seu torn, els hepatòcits secreten productes als
canalicles, que van confluint fins que formen el conducte biliar que
va a  parar a la vesícula biliar.

\subsubsection{Funcions hepàtiques}
\label{sec:funcions-hepatiques}
La zona periportal i la zona perivenosa tenen diferències metabòliques
diferents. El fetge secreta albúmina, factors de coagulació...

El fetge té activitat detoxificadora d'alcohol, xenobiòtics, toxines... 

Hi ha un consens de 5-6 proves que determinen la funció
hepàtica. Aquestes proves són:
\begin{itemize}
\item Bilirubina plasmàtica: Informadora de la funció detoxificadora i
  excretora.

\item Enzims com ALT i AST, fosfatasa alcalina, la $\gamma$-glutamil
  transpeptidasa, LDH. ALT i AST informen de l'estat del
  parènquima. La fosfatasa alcalina i la $\gamma$-glutamil
  transpeptidasa informen de la ruta biliar. La LDH informa del
  parènquima.

\item Proteïna total plasmàtica

\item Albúmina (funció biosintètica a llarg termini). Vida mitjana
  llarga.
 
\item Temps de protrombina (funció biosintètica a curt termini). Vida
  mitjana molt curta.
\end{itemize}

\subsection{Estudi de la funció excretora}
\label{sec:estudi-de-la}

\subsubsection{Producció de bilirrubina - Icterícia}
\label{sec:prod-de-bilirr}
La bilirubina és el producte de la degradació dels grups
hemo. L'hemoglobina es destrueix al sistema reticuloendotelial. El
sistema reticuloendotelial es refereix a les cèl·lules de Kupffer del
fetge, macròfags de la melsa, de medul·la òssia, ganglis limfàtics...

El grup hemo consta d'un anell tetrapirròlic que s'ha de linearitzar,
ho fa la hemo-oxigenasa que allibera diòxid de carboni i Fe3+. S'obté
biliverdina, que a través de la biliverdina recutasa es transforma en
bilirubina.

Després la bilirubina va al fetge. La bilirubina que es genera és
lipofílica. L'acumulació de bilirubina pot causar toxicitat, sobretot
al SNC. El fetge aboca la bilirubina al fetge, però abans s'ha de fer
més hidrosoluble. Al fetge, la UDP-glucoroniltranferasa introdueix 2
glucoronats a la bilirubina. La UDP-glucoroniltranferasa és un enzim
microsomal (RE). Aquest és un procés comú a la via detoxificadora. La
bilirubina es transporta per sang lligada a albúmina. S'anomena també
bilirubina no esterificada. La bilirubina esterificada és la que ja
s'ha conjugat amb glucoronat.

\begin{itemize}
\item Lipofílica, no esterificada, no conjugada, indirecta.
\item Hidrosoluble, esterificada, conjugada, directa.
\end{itemize}

Un cop es formi la bilirubina conjugada, s'aboca a la bilis i després
al duodè. La flora microbiana transforma la bilirubina esterificada en
urobilinogen. L'urobilinogen pot tornar a entrar a l'organisme via
vena porta. Un 5\% va al ronyó i dóna coloració groguenca a la orina.

La bilirubina entra a l'hepatòcit per un transportador d'albúmina. Pot
quedar-se al citosol lligada a ligandina i després anar al RE o
directament entrar el RE. La UDP-glucoroniltranferasa modificarà la
bilirubina i serà diglucoronidada. La bilirubina conjugada s'aboca a
la membrana canalicular a la bilis.

Aquesta via també s'usa pel reciclatge dels grups hemo dels citocroms
hepàtics.

La bilirubina no conjugada es queda a l'organisme i es manifesta com a
icterícia. Les analítiques discriminen entra la bilirubina soluble i
la resta.

\paragraph{Determinació de bilirubina} \hfill \\
Es fa per la tècnica de Ehrlich, proposada el 1883. Es tracta la
mostra ambn àcid sulfanílic diazotizat. Si és bilirubina conjugada,
l'àcid atacarà la bilirubina i quedarà lligat a una de les 2 parts de
la molècula generades. Per això s'anomena bilirubina directa.

La bilirubina lipòfila és determina de manera indirecta a partir de la
bilirubina total. S'addiciona un dissolvent orgànic a la mostra i
després es fa la tècnica d'Ehrlich.

També es pot determinar l'urobilinogen.

\paragraph{Icterícia} \hfill \\
La icterícia neonatal es deu a que al moment del naixement no hi ha
activitat hepàtica completa. Es deu també al recanvi de la Hb fetal
per l'adulta. Incrementa la bilirubina liposoluble. La degradació de
la bilirubina es pot afavorir per l'exposició a la llum solar. Els
nens prematurs se'ls sotmet a rajos UV.

L'acumulació de bilirubina al cervell s'anomena kernicterus i pot
causar dany cerebral irreversible.

Una hemòlisi massiva pot causar icterícia ja que el sistema es satura.

% Taula causes icterícia

\begin{itemize}
\item Quan hi ha hepatitis es veu que l'entrada a bilirubina i la conjugació
no estan afectades però sí que la sortida de bilirubina està
alterada. La hepatitis vírica o provocada per fàrmacs indueix una
icterícia directa. Es detecta bilirubina en orina de color fosc i les
femtes blanques.

\item L'alteració hemolítica només fa canviar la bilirubina indirecta.

\item Síndrome de Gilbert: Problemes amb l'entrada de la bilirubina a
  l'hepàtòcit. A més, la glucoronil transferasa té baixa activitat.

\item Síndrome de Dubin-Johnson i Rotor són icterícies hepàtiques
  conjugades. 
  \begin{itemize}
  \item La síndrome de Dubin-Johnson s'observa un fetge amb una
  pigmentació molt fosca (\textit{black liver}). Defecte en el
  transportador d'anions orgànics de la membrana canalicular.
\item Síndrome de Rotor: Es creu que és un problema de transport de la
  bilirubina conjugada del RE a la membrana canalicular.
  \end{itemize}

\end{itemize}

La prova de la BST es basa en l'eliminació d'un colorant que passa pel
fetge i es pot distingir entre la síndrome de Dubin-Johnson i de Rotor.

\subsection{Estudi de la capacitat detoxificadora}
\label{sec:estudi-de-la-1}

\subsubsection{Eliminació de xenobiòtics}
\label{sec:elim-de-xenob}


\subsection{Estudi de les capacitats biosintètica i metabòlica}
\label{sec:estudi-de-les}


\subsection{Marcadors sèrics}
\label{sec:marcadors-serics}


\subsection{Alteracions hepàtiques}
\label{sec:alter-hepat}

