%------------------------------------------------------------------------------
% Tema 10. El fetge
%------------------------------------------------------------------------------
\section{El fetge}
\label{sec:el-fetge}

\subsection{El fetge}
\label{sec:el-fetge-1}

\subsubsection{Circulació hepàtica}
\label{sec:circulacio-hepatica}
La sang entra el fetge per la via porta (intestí) i a través de
l'artèria hepàtica (sistèmica). Surt del fetge a través de la vena
hepàtica, que conflueix a la vena cava.

Els nutrients de la dieta arriben via vena porta, arriben als
hepatòcits. Al seu torn, els hepatòcits secreten productes als
canalicles, que van confluint fins que formen el conducte biliar que
va a  parar a la vesícula biliar.

\subsubsection{Funcions hepàtiques}
\label{sec:funcions-hepatiques}
La zona periportal i la zona perivenosa tenen diferències metabòliques
diferents. El fetge secreta albúmina, factors de coagulació...

El fetge té activitat detoxificadora d'alcohol, xenobiòtics, toxines... 

Hi ha un consens de 5-6 proves que determinen la funció
hepàtica. Aquestes proves són:
\begin{itemize}
\item Bilirubina plasmàtica: Informadora de la funció detoxificadora i
  excretora.

\item Enzims com ALT i AST, fosfatasa alcalina, la $\gamma$-glutamil
  transpeptidasa, LDH. ALT i AST informen de l'estat del
  parènquima. La fosfatasa alcalina i la $\gamma$-glutamil
  transpeptidasa informen de la ruta biliar. La LDH informa del
  parènquima.

\item Proteïna total plasmàtica

\item Albúmina (funció biosintètica a llarg termini). Vida mitjana
  llarga.
 
\item Temps de protrombina (funció biosintètica a curt termini). Vida
  mitjana molt curta.
\end{itemize}

\subsection{Estudi de la funció excretora}
\label{sec:estudi-de-la}

\subsubsection{Producció de bilirrubina - Icterícia}
\label{sec:prod-de-bilirr}
La bilirubina és el producte de la degradació dels grups
hemo. L'hemoglobina es destrueix al sistema reticuloendotelial. El
sistema reticuloendotelial es refereix a les cèl·lules de Kupffer del
fetge, macròfags de la melsa, de medul·la òssia, ganglis limfàtics...

El grup hemo consta d'un anell tetrapirròlic que s'ha de linearitzar,
ho fa la hemo-oxigenasa que allibera diòxid de carboni i Fe3+. S'obté
biliverdina, que a través de la biliverdina recutasa es transforma en
bilirubina.

Després la bilirubina va al fetge. La bilirubina que es genera és
lipofílica. L'acumulació de bilirubina pot causar toxicitat, sobretot
al SNC. El fetge aboca la bilirubina al fetge, però abans s'ha de fer
més hidrosoluble. Al fetge, la UDP-glucoroniltranferasa introdueix 2
glucoronats a la bilirubina. La UDP-glucoroniltranferasa és un enzim
microsomal (RE). Aquest és un procés comú a la via detoxificadora. La
bilirubina es transporta per sang lligada a albúmina. S'anomena també
bilirubina no esterificada. La bilirubina esterificada és la que ja
s'ha conjugat amb glucoronat.

\begin{itemize}
\item Lipofílica, no esterificada, no conjugada, indirecta.
\item Hidrosoluble, esterificada, conjugada, directa.
\end{itemize}

Un cop es formi la bilirubina conjugada, s'aboca a la bilis i després
al duodè. La flora microbiana transforma la bilirubina esterificada en
urobilinogen. L'urobilinogen pot tornar a entrar a l'organisme via
vena porta. Un 5\% va al ronyó i dóna coloració groguenca a la orina.

La bilirubina entra a l'hepatòcit per un transportador d'albúmina. Pot
quedar-se al citosol lligada a ligandina i després anar al RE o
directament entrar el RE. La UDP-glucoroniltranferasa modificarà la
bilirubina i serà diglucoronidada. La bilirubina conjugada s'aboca a
la membrana canalicular a la bilis.

Aquesta via també s'usa pel reciclatge dels grups hemo dels citocroms
hepàtics.

La bilirubina no conjugada es queda a l'organisme i es manifesta com a
icterícia. Les analítiques discriminen entra la bilirubina soluble i
la resta.

\paragraph{Determinació de bilirubina} \hfill \\
Es fa per la tècnica de Ehrlich, proposada el 1883. Es tracta la
mostra ambn àcid sulfanílic diazotizat. Si és bilirubina conjugada,
l'àcid atacarà la bilirubina i quedarà lligat a una de les 2 parts de
la molècula generades. Per això s'anomena bilirubina directa.

La bilirubina lipòfila és determina de manera indirecta a partir de la
bilirubina total. S'addiciona un dissolvent orgànic a la mostra i
després es fa la tècnica d'Ehrlich.

També es pot determinar l'urobilinogen.

\paragraph{Icterícia} \hfill \\
La icterícia neonatal es deu a que al moment del naixement no hi ha
activitat hepàtica completa. Es deu també al recanvi de la Hb fetal
per l'adulta. Incrementa la bilirubina liposoluble. La degradació de
la bilirubina es pot afavorir per l'exposició a la llum solar. Els
nens prematurs se'ls sotmet a rajos UV.

L'acumulació de bilirubina al cervell s'anomena kernicterus i pot
causar dany cerebral irreversible.

Una hemòlisi massiva pot causar icterícia ja que el sistema es satura.

% Taula causes icterícia

\begin{itemize}
\item Quan hi ha hepatitis es veu que l'entrada a bilirubina i la conjugació
no estan afectades però sí que la sortida de bilirubina està
alterada. La hepatitis vírica o provocada per fàrmacs indueix una
icterícia directa. Es detecta bilirubina en orina de color fosc i les
femtes blanques.

\item L'alteració hemolítica només fa canviar la bilirubina indirecta.

\item Síndrome de Gilbert: Problemes amb l'entrada de la bilirubina a
  l'hepàtòcit. A més, la glucoronil transferasa té baixa activitat.

\item Síndrome de Dubin-Johnson i Rotor són icterícies hepàtiques
  conjugades. 
  \begin{itemize}
  \item La síndrome de Dubin-Johnson s'observa un fetge amb una
  pigmentació molt fosca (\textit{black liver}). Defecte en el
  transportador d'anions orgànics de la membrana canalicular.
\item Síndrome de Rotor: Es creu que és un problema de transport de la
  bilirubina conjugada del RE a la membrana canalicular.
  \end{itemize}

\end{itemize}

La prova de la BST (bromosulftaleïna) es basa en l'eliminació d'un
colorant que passa pel fetge i es pot distingir entre la síndrome de
Dubin-Johnson i de Rotor.

\subsubsection{Àcids biliars}
\label{sec:acids-biliars}
La bilis està formada per:
\begin{itemize}
\item Pigments biliars (bilirubina i biliverdina)
\item Àcids biliars sintetitzats a partir de colesterol
\item Colesterol lliure
\item Fosfolípids
\item Ions inorgànics
\item Proteïnes
\end{itemize}

L'eliminació d'hidroxils dels àcids biliars primaris dóna lloc als
secundaris, que són produïts a l'intestí.

Quan hi ha una obstrucció renal, aquests passen a la circulació
sistèmica. Es poden trobar a orina. Els símptomes són una malabsorció
de lípids, esteatorrea, femtes clares... Temps de protrombina elevat,
icterícia. La obstrucció biliar es comprova administrant vitamina K
exògena i mirant si el temps de protrombina es normalitza. El
tractament és amb ultrasons, estatines o l'ablació de la
vesícula. Moltes vegades la osbtrucció biliar és per cristalls de
colesterol ja que hi ha un excés o per dèficit d'àcids biliars.

La determinació d'àcids biliars es fa per tècniquest cromatogràfiques
(GLC, HPLC), assajos enzimàtics o immunoassaigs.

\subsection{Estudi de la capacitat detoxificadora}
\label{sec:estudi-de-la-1}

\subsubsection{Eliminació de xenobiòtics}
\label{sec:elim-de-xenob}
L'eliminació de xenobiòtics o productes tòxics es basa en introduir
grups funcionals que facin més solubles aquests productes i facilitar
la seva excreció.
\begin{itemize}
\item Fase I: Introducció de grups polars: oxidacions, ...
\item Fase II: Conjugació amb glucorònic, glicina o taurina.
\end{itemize}

Les proves que es fan són funcionals:
\begin{itemize}
\item Eliminació de colorants: Han de ser colorants que no s'eliminin
  via orina sinó que tinguin pas hepàtic. Rosa de Bengala,
  bromosulftaleïna. L'aparició d'un pic secundari en la corba
  d'eliminació plasmàtica del colorant suposa la síndrome de
  Dubin-Johnson, si el perfil és normal és síndrome de Rotor. En
  Dubin-Johnson el colorant difon per la membrana i torna a sang.

\item Ús de metil-xantines (cafeïna -> citocrom P450): Aparició de
  productes metabòlics en orina. Es valora en sang els productes del
  metabolisme de metil-xantines exògenes.

\item Metabolisme de fàrmacs (aminopirina-14C): Velocitat de
  desaparició del fàrmac a la sang. Aparició de productes metebòlics
  en sang i/o orina. Anàlisi respiratori (14CO2).
\end{itemize}

La via detoxificadora pot produir productes tòxics per l'organisme.

\subsection{Estudi de les capacitats biosintètica i metabòlica}
\label{sec:estudi-de-les}

\subsubsection{Proteïnes}
\label{sec:proteines}
La producció d'albúmina hepàtica és un indicador de la capacitat
biosintètica. L'albúmina té una vida mitja bastant llarga, pel que la
hipoalbuminèmia reflecteix alteracions 2 o 3 setmanes abans de
l'estudi.

Molts factors de coagulació també es sintetitzen al fetge:
\begin{itemize}
\item Temps de protrombina (TP): FII, FV,FVII, FX i fibrinogen
\item Temps de tromboplastina (TPT): precalicreïna, FII, FV, FVIII,
  FIX, FX, FXII i fibrinogen.
\end{itemize}

% Taula altres proteïnes

\subsubsection{Lípids i lipoproteïnes}
\label{sec:lipids-i-lipopr}
Hi ha 2 enzims:
\begin{itemize}
\item LCAT: Si disminueix LCAT, augmenten els TAG en plasma.
\item Lipasa hepàtica: Si disminueix, disminueixen els èsters de colesterol.
\end{itemize}

En el cas de les lipoproteïnes:
\begin{itemize}
\item Hepatitis: Absència de HDL i VLDL, augment de LDL i IDL.
\item Hepatitis crònica: Augment de LDL pobres en èsters de
  colesterol.
\item Icterícia obstructiva: Lipoproteïna X
\end{itemize}

\subsubsection{Aminoàcids i nitrogen amínic}
\label{sec:amin-i-nitr}
Quan hi ha una davallada del cicle de la urea (glutamina sintetasa) es
produeix hiperamonèmia.

En una cirrosi o hepatitis apareixen aminoàcids a la orina.

\subsubsection{Glúcids}
\label{sec:glucids}
Defectes en el metabolisme de la galactosa provoquen galactossèmia.

La hipoglucèmia es pot produir quan hi ha esteatosi hepàtica (síndrome
de Reye per l'administarció d'aspirina, alcoholisme).

\subsection{Marcadors sèrics}
\label{sec:marcadors-serics}

\subsubsection{Transaminases}
\label{sec:transaminases}
La determinació d'activitat enzimàtica es fa per espectrofotometria a
340 nm. 

Indiquen afectació del parènquima hepàtic, si és greu apareix la forma
mitocondrial de la GOT.

Quan les transaminases hepàtiques augmenten més de 10x fa sospitar de
dany hepàtic.

El criteri de consens:
\begin{itemize}
\item \textbf{hepatitis} quan l'increment de
transaminases sigui fins a 100 vegades. L'increment de les
transaminases serà proporcional al dany hepàtic (ALT > AST).

\item \textbf{cirrosi} hi ha molta variabilitat interindividual i AST > ALT.
\end{itemize}

\subsubsection{Fosfatasa alcalina}
\label{sec:fosfatasa-alcalina}
No és exclusiu de fetge i s'ha de determinar la isoforma que causa
l'augment de fosfatasa alcalina. El que es fa és usar inhibidors
específics a l'hora de determinar-ne l'activitat o bé fer una
electroforesi, ja que la fosfatasa alcalina es glicosila de forma
diferencial segons el teixit. La que es retarda en el gel està
glicosilada.

La isoforma òssia i hepàtica són les més freqüents en circulació. La
isoforma hepàtica està a la membrana canalicular. 

Els nens tenen més fosfatasa alcalina en sang (isoforma òssia) degut a
la remodelació òssia; que és més intensa en nens que en adults. Per
descartar l'alteració hepàtica es determina la 5'-nucleotidasa. Si
augmenta en paral·lel la 5'-nucleotidasa, vol dir que hi ha un
problema hepàtic.

\subsubsection{Gamma-glutamil transpeptidasa}
\label{sec:gamma-glut-transp}
Marcador de toxicitat hepàtica, sobretot deguda a la ingesta d'alcohol.


\subsection{Alteracions hepàtiques}
\label{sec:alter-hepat}
\begin{itemize}
\item \textbf{Hepatitis:} Inflamació del fetge per virus, bacteris, paràsits,
  tòxics...  El més habitual és una inflamació aguda molt important
  que normalment s'acaba curant; a vegades la lesió no es regenera i
  s'omplen de fibrina generant nòduls hepàtics fibròtics. A vegades,
  les hepatitis poden ser fulminants. Altres casos, la inflamació es
  pot cronificar que alterna pics d'inflamació amb estats de repòs; a
  poc a poc es genera un parènquima ric en nòduls fibròtics.

\item \textbf{Carcinoma hepatocel·lular}

\item \textbf{Colestasi:} Bloqueig de les vies biliars per una
  infecció, pedra o neoplàsica. Es manifesta mab icterícia,
  inflamació... La colestasi és habitual quan hi ha tumors al cap del
  pàncrees, que obstrueixen la part final de la via biliar.
\end{itemize}

\subsubsection{Hepatitis víriques}
\label{sec:hepatitis-viriques}
Les infeccions víriques al fetge són molt freqüents. Les hepatitis es
poden diferenciar per la presència d'anticossos específics en sèrum.

Quan en la hepatitis A es detecten els anticossos, comencen els
símptomes. Augmenten la bilirubina i la ALT.

En el cas de l'hepatitis B, es detecten antígens dies abans que
comencin els símptomes. L'anticòs de l'antigen de superfície i del
nucli es mantenen en sèrum i generen immunitat a llarg termini.

El virus de l'hepatitis D és un virus satèl·lit que replica en una
co-infecció. 