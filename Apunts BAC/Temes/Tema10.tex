%------------------------------------------------------------------------------
% Tema 10. El fetge
%------------------------------------------------------------------------------
\section{El fetge}
\label{sec:el-fetge}

\subsection{El fetge}
\label{sec:el-fetge-1}

\subsubsection{Circulació hepàtica}
\label{sec:circulacio-hepatica}
La sang entra el fetge per la via porta (intestí) i a través de l'artèria hepàtica (sistèmica). Surt del fetge a través de la vena hepàtica, que conflueix a la vena cava.

Els nutrients de la dieta arriben via vena porta, arriben als hepatòcits. Al seu torn, els hepatòcits secreten productes als canalicles, que van confluint fins que formen el conducte biliar que va a  parar a la vesícula biliar.

\subsubsection{Funcions hepàtiques}
\label{sec:funcions-hepatiques}
La zona periportal i la zona perivenosa tenen diferències metabòliques diferents. El fetge secreta albúmina, factors de coagulació...

El fetge té activitat detoxificadora d'alcohol, xenobiòtics, toxines... 

Hi ha un consens de 5-6 proves que determinen la funció hepàtica. Aquestes proves són:
\begin{itemize}
\item Bilirubina plasmàtica: Informadora de la funció detoxificadora i excretora.

\item Enzims com ALT i AST, fosfatasa alcalina, la $\gamma$-glutamil transpeptidasa, LDH. ALT i AST informen de l'estat del parènquima. La fosfatasa alcalina i la $\gamma$-glutamil transpeptidasa informen de la ruta biliar. La LDH informa del parènquima.

\item Proteïna total plasmàtica

\item Albúmina (funció biosintètica a llarg termini). Vida mitjana llarga.
 
\item Temps de protrombina (funció biosintètica a curt termini). Vida mitjana molt curta.
\end{itemize}

\subsection{Estudi de la funció excretora}
\label{sec:estudi-de-la}

\subsubsection{Producció de bilirrubina - Icterícia}
\label{sec:prod-de-bilirr}


\subsubsection{Eliminació d'àcids biliars - Obstrucció biliar}
\label{sec:elim-dacids-bili}


\subsection{Estudi de la capacitat detoxificadora}
\label{sec:estudi-de-la-1}

\subsubsection{Eliminació de xenobiòtics}
\label{sec:elim-de-xenob}


\subsection{Estudi de les capacitats biosintètica i metabòlica}
\label{sec:estudi-de-les}


\subsection{Marcadors sèrics}
\label{sec:marcadors-serics}


\subsection{Alteracions hepàtiques}
\label{sec:alter-hepat}

