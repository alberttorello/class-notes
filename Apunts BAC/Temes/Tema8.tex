%------------------------------------------------------------------------------
% Tema 8. Disfuncions del metabolisme glucídic
%------------------------------------------------------------------------------
\section{Disfuncions del metabolisme glucídic}
\label{sec:disf-del-metab}

\subsection{Regulació de la concentració de la glucosa sanguínia}
\label{sec:regulacio-de-la}
% Recordatori insulina/glucagó

Els nivells normals de glucosa en sang són de 3,89-5,83 mmols/L
(70-105 mg/dL).

\subsection{Hiperglucèmies}
\label{sec:hiperglucemies}
Hi ha 4 tipus d'hiperglucèmia:
\begin{itemize}
\item Diabetis tipus 1: Hiperglucèmia de manera abrupta. Els pacients
  necessiten insulina per sobreviure.
\item Diabetis tipus 2: Progressió gradual i la insulina no acostuma a
  ser necessària pel tractament.
\item Altres tipus específics
\item Diabetis gestacional: Es produeix en l'últim terç de la
  gestació. Normalment acaba després del part però en alguns casos el
  quadre hiperglucèmic continua.
\end{itemize}

Hi ha 2 subgrups poblacionals en risc de diabetis:
\begin{itemize}
\item Intolerància a la glucosa: 2h després de la ingesta de 75 de
  glucosa, els nivells de glucosa es troben entre 140 i 199 mg/dL.

\item Disfunció de la glucosa en dejuni: La glucosa es troba entre 100
  i 125 mg/dL.
\end{itemize}

Hi ha un marcador circulant usat per monitoritzar els diabètics que és
la hemoglobina $A_{1c}$. L'hemoglobina incorpora glucosa
espontàniament a la seva estructura i són un reflex dels nivells
circulants de glucosa. Aquesta hemoglobina té una vida mitja llarga.

Els nous criteris diagnòstics de la diabetis són:
\begin{itemize}
\item Símptomes de diabetis més una concentració casual (no en dejuni)
  de glucosa superior a 200 mg/dL.
\item Glucosa en plasma després d'un dejuni mínim de 8 hores superiors
  a 126 mg/dL.
\item Glucèmia superior a 200 mg/dL després de 2 hores del test de
  tolerància oral de glucosa.
\item Nivells Hb$A_{1c}$ superiors a 6,5\%.
\end{itemize}

% Taula comparativa

\subsubsection{Alteracions metabòliques}
\label{sec:alter-metab}
Augmenten els nivells de TAG ja que la LPL és sensible a
insulina. Cetoacidosi.

El fetge activa la gluconeogènesi i la glicogenòlisi, fet que augmenta
la glucosa circulant.

També augmenta la cetogènesi, degut a la beta oxidació dels àcids
grassos alliberats del teixit adipós, que està fent lipòlisi. Els
acetilCoA no es poden utilitzar al cicle de Krebs ja que els
intermediaris estan dirigits a la GNG. Els cossos cetònics
proporcionen energia a altres teixits però provoquen cetoacidosi
(acidosi metabòlica).

\subsubsection{Complicacions}
\label{sec:complicacions}
Alteracions de la microvasculatura i la macrovasculatura. Els vasos es
malmeten i generen gangrena sobretot als peus.

Hi ha neuropatia.

També hi pot haver retinopatia diabètica.

Els ronyons també pateixen un dany. La glucosa és un osmòlit, i com
que la concentració és tant alta no es pot reabsorbir a nivell
tubular. La glucosa malmet els glomèruls renals. L'epiteli fenestrat
presenta càrregues negatives i impedeix el pas de proteïnes. La
glucosa alta fa que els glomèruls siguin més permeables i que a la
orina hi hagi més proteïnes, majoritàriament albúmina perquè és la
proteïna més abundant del plasma i té un pes de 65 kDa, límit de mida
per filtrar als glomèruls (60 kDa).

També hi ha més risc de patir aterosclerosi. Sembla que estar
relacionada amb un procés de glicosilació de la superfície de les
lipoproteïnes.

\subsection{Paràmetres clínics}
\label{sec:parametres-clinics}

\subsubsection{Glucosa}
\label{sec:glucosa}
El mètode de l'ortotoluïdina consisteix que en medi àcid, la glucosa
reacciona amb amines aromàtiques i genera color. Un altre mètode, el
de Benedict està basat en la reducció del Cu (usat en tires reactives per la orina).

El mètode de referència és el mètode que han de tenir tots els
laboratoris i poder avaluar la qualitat inter-laboratoris. Per la
glucosa, és el mètode basat en la hexoquinasa i la
glucosa-6-P-deshidrogenasa. La reacció genera NADPH, que es pot
mesurar a 340 nm.

Hi pot haver diferents fonts de variabilitat:
\begin{itemize}
\item Sang/Sèrum o plasma. Normalment no s'usa sang total, ja que els
  eritròcits consumeixen glucosa i els valors sortirien més baixos
  dels reals.
\item La concentració arterial és superior a la venosa.
\item L'edat, els fàrmacs (corticoides i diürètics) augmenten la
  glucosa.
\end{itemize}

Diferents precaucions que s'han de prendre:
\begin{itemize}
\item No ha de ser sang total (eritròcits contenen < [glc])
\item Cal ser ràpids per evitar metabolisme eritrocitari
\item Desproteinització prèvia per evitar interferències
\item Estabilitat 8h a 25ºC – 72h a 4ºC
\end{itemize}

\subsubsection{Sobrecàrrega oral de glucosa (SOG)}
\label{sec:sobrecarrega-oral-de}
Si surt un valor alt de glucèmia, es repeteix l'analítica. Si torna a
sortir hiperglucèmia es practica un test de sobrecàrrega oral de
glucosa o test de tolerància oral a la glucosa. Aquest test es
realitza en 4 situacions:
\begin{enumerate}
\item Diabetis gestacional
\item Intolerància a la glucosa (IGT)
\item Neuropaties, nefropaties o retinopaties. En casos que la
  glucèmia sigui inferior a 140 mg/dL (7,7 mM).
\item Epidemiologia
\end{enumerate}

El test consisteix en ingerir 75g de glucosa pels adults i 1,75 g de
glucosa/kg en nadons i nens. S'extreu sang a $t_0$ en dejuni i en
situació normal. Cada cert temps es mesura la glucèmia i s'avalua
l'evolució de la glucèmia. Els resultats poden ser:
\begin{enumerate}
\item Normal: Pic abrupte a 30 minuts fins que a les 2h es normalitza
  la glucèmia.
\item Diabetis: Fa un pic abrupte a 30 min i després un  plateau més o
  menys estable.
\item Intolerant: Perfil intermig.
\end{enumerate}

En el cas de la diabetis gestacional, es fa un \textit{screening} cap
a les 24 setmanes de gestació. El test de O'Sullivan no requereix que
la gestant estigui en dejuni. S'administren 50 g de glucosa i es
mesura la glucèmia al cap d'1h. Si surt la glucèmia superior a 7,7 mM
es practica un test de SOG però amb 100 g de glucosa i es mira si la
glucosa al cap de 2h és inferior a 9,4 mM o bé si hi ha 2 punts de la
corba per sobre del llindar.

\subsubsection{Insulina}
\label{sec:insulina}
En el SOG, en paral·lel a la glucosa es pot determinar la insulina. No
es fa normalment. En situació normal, el perfil d'insulina és
paral·lel al de la glucosa. 

% Secreció d'insulina

Hi ha diferents situacions on és interessant determinar la insulina:
\begin{enumerate}
\item Hipoglucèmia
\item Insulinoma (tumor que secreta insulina)
\item Diabètic amb sobredosi
\end{enumerate}

El pèptid C és resultat de la proteòlisi de la pro-insulina. La vida
mitja del pèptid C circulant és més alta que la de la insulina. Al
fetge s'elimina el 75\% de la insulina circulant. La relació pèptid
C/insulina és equimolar. Normalment, hi ha 5-10 vegades més pèptid C
que insulina.
\begin{enumerate}
\item Si hi ha nivells elevats d'insulina i pèptid C, és símptoma
  d'insulinoma.
\item Si hi ha més insulina que pèptid C, vol dir que hi hagut una
  intoxicació per insulina.
\end{enumerate}

\subsubsection{Glicohemoglobines}
\label{sec:glicohemoglobines}


\subsubsection{Fructosamines}
\label{sec:fructosamines}


\subsubsection{Cossos cetònics}
\label{sec:cossos-cetonics}


\subsection{Hipoglucèmies}
\label{sec:hipoglucemies}


\subsection{Intoleràncies}
\label{sec:intolerancies}

