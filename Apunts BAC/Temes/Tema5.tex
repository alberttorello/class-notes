%------------------------------------------------------------------------
% Tema 5. Hemostàsia i coagulació
%------------------------------------------------------------------------
\section{Hemostàsia i coagulació}

\subsection{Introducció}
\subsubsection{Hemostàsia}
L'hemostàsia és l'equilibri entre factors anticoagulants i
procoagulants. Els mecanismes de control són:
\begin{enumerate}
\item Vasconstricció
\item Hemostàsia primària o formació del trombe plaquetari (blanc)
\item Hemostàsia secundària o coagulació (formació del coàgul de
  fibrina)
\item Fibrinòlisi o trencament del coàgul de fibrina
\end{enumerate}

Les característiques d'aquesta resposta són:
\begin{itemize}
\item Resposta ràpida i temporal
\item No ha d'ocluir la llum del vas lesionat
\item Localitzada al lloc de la ferida
\item Resistent al flux sanguini per tal d'evitar tromboembolismes
\end{itemize}

\subsubsection{Paper de la cèl·lula endotelial}
Les cèl·lules endotelials no pertorbades tenen efectes anticoagulants
i fibrinolítics:
\begin{itemize}
\item Anticoagulants: per la producció de:
  \begin{itemize}
  \item \textit{Tissue factor patway inhibitot} que forma part de la
    via extrínseca.
  \item Heparansulfats que activen l'antitrombina III.
  \item Trombomodulina: la qual unida a la trombina activa les
    proteïnes inhibidores de la coagulació, la proteïna C i la
    proteïna S.
  \end{itemize}

\item Fibrinolítics: degut a la protecció de \textit{tissue
    plasminogen activator} (tPA).
\end{itemize}

Quan es pertorben les cèl·lules endotelials es produeixen efectes
proadherents de plaquetes i procoagulants. Quan es trenca la paret
d'un vas, s'exposa el subendoteli el qual expressa el factor de von
Williebrand.
\begin{itemize}
\item Proadherents de plaquetes: per exposició del factor de von
  Willebrand (FvW) subendotelial.
\item Procoagulants: per contacte del factor VIIa de coagulació
  circulant amb el factor III de coagulació tissular, el qual és
  produït per cèl·lules endotelials estimulades per endotoxines o
  citoquines i per cèl·lules subendotelials del múscul llis
  constitutivament.
\end{itemize}

\subsection{Hemostàsia primària}
Quan l'endoteli es danya, s'exposa el teixit connectiu subjacent i
s'hi adhereixen les plaquetes i es forma un trombe blanc. En paral·lel
s'activa la hemostàsia secundària. S'alliberen factors de coagulació
de plaquetes, cèl·lules danyades i del plasma (calci, vitamina K).

Les plaquetes són fragments citoplasmàtics dels megacariòcits,
anucleats i rics en vesícules secretores, els quals es formen a la
mèdul·la òssia i passen a circulació. Presenten glicoproteïnes (GP)
inserides en la membrana que els permeten establir contactes entre
elles i amb el subendoteli. A mesura que les plaquetes envelleixen, es
fan més petites. 

Típicament, les plaquetes tenen una forma arrodonida. Quan les
plaquetes s'activen, adquireixen una forma estrellada.

Per tal que la plaqueta s'uneixi al factor de von Willebrand,
expressa glicoproteïnes tipus integrina i altres. El complex
GPIb–GPIX–GPV s'expressa en plaquetes i uneix FvW. Un cop les
plaquetes s'uneixen al FvW, les integrines plaquetàries GPIIb/IIIa
s'uneixen al fibrinogen circulant. El fibrinogen pot fer de pont entre
2 integrines que estan en plaquetes diferents.

Els factors més importants que alliberen les plaquetes són ADP,
serotonina, TXA2.

Els inhibidors de la funció plaquetària:
\begin{itemize}
\item Aspirina: l’àcid acetilsalicílic inhibeix la producció de TXA2
  a partir d’àcid araquidònic en inactivar irreversiblement l’enzim
  ciclooxigenasa (COX). Després que el tractament amb aspirina
  s'atura, l’activitat de l’enzim retorna a mida que noves plaquetes
  s‘incorporen a la circulació (la taxa diària de regeneració és
  d'aproximadament un 10\%). Inhibeix l'agregació plaquetària.

 
\item Els àcids grassos omega 3 de la dieta o suplements dietètics:
  competeixen amb l’àcid araquidònic com a substrats per la COX i
  produeixen TXA3 que és biològicament inert.
\end{itemize}

La trombina també contribueix a l'activació de les plaquetes.

\subsubsection{Patologies relacionades amb l'hemostàsia primària}
Poden ser:
\begin{itemize}
\item Origen vascular: La paret vascular pot estar afectada per falta
  de resistència o impermeabilitat o manca de constricció. La vitamina
  C és crucial pel bon manteniment. Un símptoma és el sagnat de les
  genives.

\item Origen plaquetari:
  \begin{itemize}
  \item En el nombre:
    \begin{itemize}
    \item Baixa producció
    \item Distribució patològica: acumulació a la melsa causant
      esplenomegàlia
    \item Destrucció per malalties autoimmunes
    \end{itemize}
  \item En la funcionalitat
    \begin{itemize}
    \item Adquirides: intoxicació per aspirina
    \item Hereditàries:
      \begin{itemize}
      \item Deficiència al factor de von Willebrand
      \item Deficiència en glicoproteïnes de membrana plaquetària:
        \begin{itemize}
        \item Gens de GPIbalpha, GPIX, GPIbbeta: Síndrome de Bernanrd
        \end{itemize}
      \end{itemize}
    \end{itemize}
  \end{itemize}
\end{itemize}

% Malalties hereditàries

\paragraph{Trombocitosi}
Hi ha un elevat nombre de plaquetes, que pot ser per causes:
\begin{itemize}
\item Primàries: Per trastorns mieloproliferatius.

\item Secundàries: deguda a:
  \begin{itemize}
  \item Infeccions
  \item Malalties inflamatòries
  \item Pèrdua de sang (la medul·la òssia respon incrementant la
    producció cel·lular)
  \item Dany tissular per trauma o cirurgia
  \item Alguns fàrmacs
  \item Melsa inactiva o esplenectomia
  \end{itemize}
\end{itemize}

\subsubsection{Tests per avaluar la hemostàsia primària}
\label{sec:tests-per-avaluar}

\paragraph{Fragilitat capil·lar} \hfill \\
Es fa servir el mànec de l'aparell per mesurar la pressió arterial o
esfigmomanòmetre i es col·loca al voltant de la part superior del
braç. S'infla fins a una pressió més o menys entre la sistòlica i la
diastòlica de la persona (potser 100 mm Hg) i es deixa posat durant 4
a 6 minuts.

En una prova positiva, apareixen nombrosos punts vermells petits
(petèquies) a la pell per sota del mànec. Aquestes petèquies són
conseqüència de la fragilitat capil·lar.

\paragraph{Temps de sagnia} \hfill \\
Es practica un tall de dimensions estandarditzades en el teixit
subcutani (afecta a capil·lars) i es mesura el temps que tarda en
bloquejar-se l’hemorràgia.

\begin{itemize}
\item Tècnica de Duke: tall a l'orella (t < 3 minuts).
\item Tècnica d’Ivy: tall en l’avantbraç (t< 6 minuts).
\end{itemize}

El temps de sagnia depèn del número i funció de les plaquetes, la
concentració de fibrinogen i la funció del vas, a part de les
característiques del tall.

\paragraph{Recompte de plaquetes} \hfill \\
Recompte de plaquetes al microscopi en cambra de recompte o en
comptador automatitzat.
\begin{itemize}
\item < 150 x 109/ L Trombocitopènia assimptomàtica 
\item < 50x109/L Hemorràgiesespontànies
\end{itemize}

\paragraph{Mida de les plaquetes} \hfill \\
La grandària de les plaquetes es pot observar en una pel·lícula de
sang perifèrica.
\begin{itemize}
\item  Volum normal: 7 - 10 fL
\item  Diàmetre normal: 2 μm.
\end{itemize}

Un increment en el recanvi està associat a major volum, perquè les
plaquetes recent formades a la medul·la òssia són més grans i la seva
mida disminueix mentre envelleixen circulant en sang.

Les plaquetes grans també s’observen en algunes malalties genètiques
com la síndrome de Bernard-Soulier.

A la síndrome de Wiskott-Aldrich les plaquetes són més petites del
normal.

% Taula diapo 11

\paragraph{Temps d'agregació plaquetària} \hfill \\
Es mesura la capacitat d’agents per induir l’activació i agregació de
les plaquetes: la velocitat i magnitud.

Es tracta la sang amb citrat, el citrat quelarà el Ca. Es centrifuga
la sang a baixes revolucions. Al sobrenedant quedaran les plaquetes i
s'afegeixen diferents elements que indueixen l'agregació plaquetària i
es monitora l'absorbància. A mida que les plaquetes s’agregen,
la intensitat de llum que travessa la mostra és superior. Es deixa una
alíquota sense addició d’agent activador per valorar l’agregació
espontània.

\begin{itemize}
\item  L’agregació està disminuïda quan les plaquetes són disfuncionals.
\item La resposta als agents agregants és diferent segons l’origen de la disfunció.
\end{itemize}

Una resposta deficient a ristocetin i normal a altres agonistes indica
deficiència en el complex GPIb-IX-V (síndrome de Bernard-Soulier) a
les plaquetes o en el factor de von Willebrand (FvW) a plasma. La
resposta a araquidonat està disminuïda en la intoxicació per aspirina.

\subsection{Hemostàsia secundària/Coagulació proteica}
El trombe blanc és inestable. Intervenen factors proteics, Ca,
vitamina K. 

\subsubsection{Proteïnes i factors de coagulació}
\label{sec:proteines-i-factors}
Els factors proteics poden ser zimògens activats per
Serina proteases. La major part dels factors de coagulació es
sintetitzen al fetge. 
\begin{itemize}
\item Factor III, \textit{tissue factor} o tromboplastina es
produeix al teixit danyat i a les plaquetes activades. 
\item El Factor IV o Ca ve dels ossos, de la dieta, de les plaquetes.
\item Factor VIII: plaquetes i cèl·lules endotelials. La deficiència
  produeix hemofília.
\item Factor XVI o de von Willebrand: plaquetes i subendoteli.
\end{itemize}

Els co-factors són el III (VIIa), el V (pro-trombinasa), el VIII
(tenasa), el factor XV (XI, XII) i la proteïna S.

Factor IX dóna hemofília B, factor XI dóna hemofília C.

\subsubsection{Cascada de coagulació}
\label{sec:casc-de-coag}
Hi ha 2 vies d'activació de la trombina:
\begin{itemize}
\item Intrínseca: El dany del vas activa zimògens, que convergeixen al
  factor X.
\item Extrínseca: El trauma activarà el tissue factor, i s'activarà el
  factor X. 
\end{itemize}

Intervenen Ca i fosfolípids de la superfície de les plaquetes.

El factor Xa i Va, que juntament amb fosfolípids i Ca formen el
complex de la protrombinasa (activa la protrombina a la trombina).

El factor X s'activa per la tenasa. Hi ha dos complexes de tenasa:
\begin{enumerate}
\item VIIa i TF formen el complex de la tenasa extrínseca
\item IXa i VIIIa formen el complex de la tenasa intrínseca. Intervé
  el vWF, que estabilitza el factor VIII.
\end{enumerate}

La via principal és l'extrínseca. La via intrínseca no és la més
important en la iniciació de la coagulació. El complex de la tenasa
extrínseca activa el factor IX, que activa la tenasa intrínseca.

\subsubsection{Tipus de factors de coagulació i mecanismes
  d'activació}
\label{sec:tipus-de-factors}
% Taula diapo 3

\paragraph{Vitamina K} \hfill \\
La vitamina K és liposoluble. Hi ha dues formes naturals, la vitamina
K1 (filoquinona) que és abundant en vegetals verds i cereals i la
vitamina K2 (menaquinona) produïda per bacteris en aliments fermentats
i flora intestinal. Les formes sintètiques són les vitamines K3, K4 i
K5.

Al fetge hi ha la epòxid reductasa que recicla la vitamina K oxidada i
la redueix. La reducció és necessària per carboxilar factors de
coagulació. El Ca s'uneix a aquests factors gamma-carboxilats. El Ca
fa de pont entre fosfolípids i factors de coagulació.

La vitamina K (forma reduïda) és necessària per l’activitat de l’enzim
gamma-glutamil carboxilasa que transforma els residus de glutamat en
gamma-carboxiglutàmic que és un lloc d’unió per calci i necessari per
la correcta funcionalitat de diversos factors de coagulació.

La deficiència en vitamina K és estranya, té lloc en malalties
intestinals que impedeixen l’absorció o tractaments llargs amb
antibiòtics. La vitamina K a dosis farmacològiques s’utilitza per
tractar les deficiències i té també efectes beneficiosos en
l’osteoporosi i càncer.

Diversos anticoagulants orals com la warfarina o el Sintrom actuen per
bloqueig de la reducció i per tant reciclatge i disponibilitat de la vitamina K.

\subsubsection{Visió actual de la coagulació}
\label{sec:visio-actual-de}
Hi ha circuits d'activació i inhibició per feedback. El factor XIII
s'entrecreua amb la fibrina i estabilitza el coàgul.

Es sintetitza com a prepro-protromnina. Hi ha una proteòlisi que
elimina el pèptid senyal. Després, la pro-protrombina és dependent de
vitamina K. S'elimina el propèptid i finalment la protrombina pateix
proteòlisi i genera la trombina.

Després de la iniciació de la cascada, el senyal es propaga i
s'amplifica.

\subsubsection{Formació del coàgul proteic}
\label{sec:formacio-del-coagul}
El fibrinogen és una proteïna soluble per la presència de pèptids rics
en aminoàcids ionitzables que creen repulsió intermolecular. La
hidròlisi d’aquests pèptids (per trombina) genera la fibrina amb alta
capacitat d’agregació.

El procés finalitza amb l’establiment d’enllaços intermoleculars en la fibrina per acció del
factor XIIIa (FSF: Fibrin-Stabilizing Factor).

La coagulació proteica consolida el trombe plaquetari.

\subsubsection{Inhibició fisiològica de la coagulació}
\label{sec:inhib-fisi-de}
\begin{enumerate}
\item L'antitrombina (AT) s'uneix a la trombina o al factor Xa per un
  enllaç covalent i les inhibeix irreversiblement. Els complexos són
  eliminats pel fetge. AT és capaç d'inhibir totes les proteases
  activades durant la coagulació, encara que a un menor grau. L’AT és
  d’origen hepàtic.

\item La unió de la trombina al seu receptor trombomodulina (TM)
  permet la conversió de la proteïna C en proteïna C activada. La
  proteïna C activada s’uneix a la proteïna S i formen un complex que
  degrada els factors Va i VIIIa. Les proteïnes C i S són produïdes
  pel fetge.

\item Els productes de degradació de la fibrina (PDFs) inhibeixen la
  trombina.

\item TFPI (tissue factor pathway inhibitor) produït per les cèl·lules
  endotelials, inhibeix al factor Xa i la trombina.
\end{enumerate}

\subsection{Fibrinòlisi}
