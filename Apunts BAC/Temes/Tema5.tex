%------------------------------------------------------------------------
% Tema 5. Hemostàsia i coagulació
%------------------------------------------------------------------------
\section{Hemostàsia i coagulació}

\subsection{Introducció}
\subsubsection{Hemostàsia}
L'hemostàsia és l'equilibri entre factors anticoagulants i
procoagulants. Els mecanismes de control són:
\begin{enumerate}
\item Vasconstricció
\item Hemostàsia primària o formació del trombe plaquetari (blanc)
\item Hemostàsia secundària o coagulació (formació del coàgul de
  fibrina)
\item Fibrinòlisi o trencament del coàgul de fibrina
\end{enumerate}

Les característiques d'aquesta resposta són:
\begin{itemize}
\item Resposta ràpida i temporal
\item No ha d'ocluir la llum del vas lesionat
\item Localitzada al lloc de la ferida
\item Resistent al flux sanguini per tal d'evitar tromboembolismes
\end{itemize}

\subsubsection{Paper de la cèl·lula endotelial}
Les cèl·lules endotelials no pertorbades tenen efectes anticoagulants
i fibrinolítics:
\begin{itemize}
\item Anticoagulants: per la producció de:
  \begin{itemize}
  \item \textit{Tissue factor patway inhibitot} que forma part de la
    via extrínseca.
  \item Heparansulfats que activen l'antitrombina III.
  \item Trombomodulina: la qual unida a la trombina activa les
    proteïnes inhibidores de la coagulació, la proteïna C i la
    proteïna S.
  \end{itemize}

\item Fibrinolítics: degut a la protecció de \textit{tissue
    plasminogen activator} (tPA).
\end{itemize}

Quan es pertorben les cèl·lules endotelials es produeixen efectes
proadherents de plaquetes i procoagulants. Quan es trenca la paret
d'un vas, s'exposa el subendoteli el qual expressa el factor de von
Williebrand.
\begin{itemize}
\item Proadherents de plaquetes: per exposició del factor de von
  Willebrand (FvW) subendotelial.
\item Procoagulants: per contacte del factor VIIa de coagulació
  circulant amb el factor III de coagulació tissular, el qual és
  produït per cèl·lules endotelials estimulades per endotoxines o
  citoquines i per cèl·lules subendotelials del múscul llis
  constitutivament.
\end{itemize}

\subsection{Hemostàsia primària}
Quan l'endoteli es danya, s'exposa el teixit connectiu subjacent i
s'hi adhereixen les plaquetes i es forma un trombe blanc. En paral·lel
s'activa la hemostàsia secundària. S'alliberen factors de coagulació
de plaquetes, cèl·lules danyades i del plasma (calci, vitamina K).

Les plaquetes són fragments citoplasmàtics dels megacariòcits,
anucleats i rics en vesícules secretores, els quals es formen a la
mèdul·la òssia i passen a circulació. Presenten glicoproteïnes (GP)
inserides en la membrana que els permeten establir contactes entre
elles i amb el subendoteli. A mesura que les plaquetes envelleixen, es
fan més petites. 

Típicament, les plaquetes tenen una forma arrodonida. Quan les
plaquetes s'activen, adquireixen una forma estrellada.

Per tal que la plaqueta s'uneixi al factor de von Willebrand,
expressa glicoproteïnes tipus integrina i altres. El complex
GPIb–GPIX–GPV s'expressa en plaquetes i uneix FvW. Un cop les
plaquetes s'uneixen al FvW, les integrines plaquetàries GPIIb/IIIa
s'uneixen al fibrinogen circulant. El fibrinogen pot fer de pont entre
2 integrines que estan en plaquetes diferents.

Els factors més importants que alliberen les plaquetes són ADP,
serotonina, TXA2.

Els inhibidors de la funció plaquetària:
\begin{itemize}
\item Aspirina: l’àcid acetilsalicílic inhibeix la producció de TXA2
  a partir d’àcid araquidònic en inactivar irreversiblement l’enzim
  ciclooxigenasa (COX). Després que el tractament amb aspirina
  s'atura, l’activitat de l’enzim retorna a mida que noves plaquetes
  s‘incorporen a la circulació (la taxa diària de regeneració és
  d'aproximadament un 10\%). Inhibeix l'agregació plaquetària.

 
\item Els àcids grassos omega 3 de la dieta o suplements dietètics:
  competeixen amb l’àcid araquidònic com a substrats per la COX i
  produeixen TXA3 que és biològicament inert.
\end{itemize}

La trombina també contribueix a l'activació de les plaquetes.

\subsubsection{Patologies relacionades amb l'hemostàsia primària}
Poden ser:
\begin{itemize}
\item Origen vascular: La paret vascular pot estar afectada per falta
  de resistència o impermeabilitat o manca de constricció. La vitamina
  C és crucial pel bon manteniment. Un símptoma és el sagnat de les
  genives.

\item Origen plaquetari:
  \begin{itemize}
  \item En el nombre:
    \begin{itemize}
    \item Baixa producció
    \item Distribució patològica: acumulació a la melsa causant
      esplenomegàlia
    \item Destrucció per malalties autoimmunes
    \end{itemize}
  \item En la funcionalitat
    \begin{itemize}
    \item Adquirides: intoxicació per aspirina
    \item Hereditàries:
      \begin{itemize}
      \item Deficiència al factor de von Willebrand
      \item Deficiència en glicoproteïnes de membrana plaquetària:
        \begin{itemize}
        \item Gens de GPIbalpha, GPIX, GPIbbeta: Síndrome de Bernanrd
        \end{itemize}
      \end{itemize}
    \end{itemize}
  \end{itemize}
\end{itemize}

% Malalties hereditàries

\paragraph{Trombocitosi}
Hi ha un elevat nombre de plaquetes, que pot ser per causes:
\begin{itemize}
\item Primàries: Per trastorns mieloproliferatius.

\item Secundàries: deguda a:
  \begin{itemize}
  \item Infeccions
  \item Malalties inflamatòries
  \item Pèrdua de sang (la medul·la òssia respon incrementant la
    producció cel·lular)
  \item Dany tissular per trauma o cirurgia
  \item Alguns fàrmacs
  \item Melsa inactiva o esplenectomia
  \end{itemize}
\end{itemize}

\subsection{Hemostàsia secundària/Coagulació proteica}


\subsection{Fibrinòlisi}
