%------------------------------------------------------------------------------
% Tema 12. Alteracions del metabolisme de nucleòtids
%------------------------------------------------------------------------------
\section{Alteracions del metabolisme de nucleòtids}
\label{sec:alter-del-metab}

\subsection{Recanvi de nucleòtids}
\label{sec:recanvi-de-nucl}
La ribosa-5-fosfat passa a PRPP (5-fosforibosil-1-pirofosfat) per la PRPP sintetasa. Els nucleòtids púrics inhibeixen la PRPP sintetasa.

La via de recuperació sintetitza nucleòsids a partir de nucleòtids de la dieta i productes de degradació.

La degradació de nucleòtids púrics genera xantina i finalment àcid úric. La xantina oxidasa pren com a substrat la hipoxantina o la xantina. La xantina és el punt de convergència de les vies de degradació de les purines. L'al·lopurinol inhibeix la xantina oxidasa. 

L'àcid úric, a pH fisiològic s'ionitza i l'urat es pot acoblar a Na i Ca i precipitar com a cristalls d'urat. Els urats tenen molt poca solubilitat. Precipita als llocs corporals on la temperatura sigui més baixa: a les extremitats, articulacions (líquid sinovial).

Quan s'inhibeix la xantina oxidasa, s'acumulen AMP/GMP/IMP que retroinhibeixen la síntesi \textit{de novo} de nucleòtids púrics.

L'excreció renal final d'àcid úric és entre el 6-12\% del filtrat inicial. L'àcid úric també es pot degradar a l'intestí, on la uricasa bacteriana el degradarà.

\subsection{Hiperuricèmia}
\label{sec:hiperuricemia}
2 causes d'hiperuricèmia:
\begin{itemize}
\item Excés de producció: Dieta rica en marisc, caça genera un excés de purines i producció d'àcid úric. El dèficit enzimàtic associat a hiperuricèmia provoca el síndroma de Lesch-Nyhan afecta el HGPRT transferasa. Les alteracions més notables són les neurològiques, on hi ha retard mental i autolesió; els nivells d'àcid úric estan elevats. El gen està al cromosoma X. En adults, la hiperuricèmia genera càlculs urinaris i inflamacions articulars. La deficiència de glucosa-6-fosfat també genera hiperuricèmia ja que hi ha un sobreflux de la via de les pentoses fosfat i síntesi de nucleòtids.

\item Baixa excreció renal: Pot ser degut a una inflamació renal. També en acidosi metabòlica, ja que es prioritza la sortida de protons de l'acidosi i es bloqueja la sortida d'àcid úric.
\end{itemize}

La determinació d'àcid úric es fa amb la reacció amb uricasa acoblada a una peroxidasa. En homes, els nivells normals són més alts que en homes. El significat diagnòstic és:
\begin{itemize}
\item Insuficiència renal
\item Excés de producció d'urat
\item Indicador no específic de la filtració glomerular.
\end{itemize}