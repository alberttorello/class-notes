%------------------------------------------------------------------------------
% Tema 18. Metabolisme ossi
%------------------------------------------------------------------------------

\section{Metabolisme ossi}
\label{sec:metabolisme-ossi}

El Ca prové de la ingesta.

La cèl·lula és més positiva fora que dintre. Les vesícules tenen més
càrregues positives a dins. El Ca afavoreix la fusió de membranes
cel·lular i vesicular i evita la repulsió electrostàtica.

El Ca va unit a albúmina per interacció electrostàtica.

PTH frena la resorció òssia.  La vitamina D afavoreix l'absorció
intestinal del Ca i actua amb sinergia amb la PTH.

Quan baixa el Ca plasmàtic, la PTH i la vitamina D afavoreixen la
resorció òssia per augmentar el Ca plasmàtic. El fosfat càlcic
tornaria a precipitar per una qüestió de equlibri de solubilitat. PTH
inhibeix la reabsorció tubular de fosfat.

La osteocalcina en plasma tant indica activitat osteoblàstica com
osteoclàstica.

Quan es crea os es troba colàgen i procolàgen. Quan es resorbeix es
troba només colàgen en plasma.

Si hi ha malaltia de Paget puja la fosfatasa àcida i la fosfatasa
alcalina.

La hidroxiprolina es detecta en orina quan es degrada colàgen==>
degradació òssia.

\subsection{Patologies}
\label{sec:patologies}

\subsubsection{Hipercalcèmia}
\label{sec:hipercalcemia}

En el cas agut és degut a un focus de necrosi tubular. 

En el cas de ser crònica, les causes són:

El que és important de determinar és el calci lliure en plasma (no
lligat a proteïnes de transport). Després es mira si hi ha
deshidratació.

Si el Ca està alt i el fosfat està baix, la causa és
hiperparatiroïdisme (PTH). La PTH augmeta el Ca i augmenta la
filtració glomerular de fosfat.

Hiperparatiroïdisme terciari: Si hi ha hipocalcèmia, es secreta
PTH. El que passa és que la PTH no aconsegueix normalitzar el Ca i la
glàndula s'acaba hipertrofiant. Això és degut a un dèficit de vitamina
D. La hipercalcàmia apareix quan s'arregla el dèficit de vitamina D.

Si els nivells de fosfat són normals o alts, vol dir que hi ha una
hipersensibilitat a vitamina D.

% Hipocalcèmia
Si el fosfat es manté elevat molt de temps inhibeix la 1-hidroxilasa
renal de la vitamina D. Això provoca un augment de PTH.

Si la urea està alta i la PTH també, hi ha una resistència a la PTH.

% Tècniques radioquímiques
El detector es basa en el centelleig líquid (detecta gamma i electrons).