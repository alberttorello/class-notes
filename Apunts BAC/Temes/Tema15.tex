%------------------------------------------------------------------------------
% Tema 15. Control del pH
%------------------------------------------------------------------------------

\section{Control del pH}
\label{sec:control-del-ph}

El control del pH és important per mantenir:
\begin{itemize}
\item Reaccions enzimàtiques cel·lulars
\item Transport de membrana
\item Receptors hormonals
\item Conformació de proteïnes de transport plasmàtic.
\end{itemize}

El pH de la sang és de 7,4 (6,8-7,8).

El CO2 o àcid carbònic s'elimina per exhalació pulmonar.

El CO2 pot provenir del metabolisme incomplet de compostos orgànics per estats fisiològics transitoris o estats patològics, com en l'exercici anaeròbic.

\subsection{Reaccions que alteren el pH}
\label{sec:reacc-que-alter}

\subsubsection{Fonts d'àcids}
\label{sec:fonts-dacids}


\subsubsection{Fonts d'àlcalis i consum de protons}
\label{sec:fonts-dalcalis}
Metabolisme de lactat i citrat de la dieta. Aquests elements es troben sobretot en fruita o verdura. Les dietes vegetarianes són alcalinitzants.

Els protons es perden quan hi ha vòmits (es perd HCl gàstric) i per deficiències en el cicle de la urea. La pèrdua d'amoni és, en definitiva, una pèrdua de protons.

\subsection{Sistemes tamponadors fisiològics}
\label{sec:sist-tamp-fisi}

La meitat dels àcids metabòlics són neutralitzats per les bases de la dieta. La resta d'àcids són neutralitzats per tampons intracel·lulars i sanguinis.

% tampons

El tampó ha de ser una parella d'àcid/base conjugada amb un pK proper al pH sanguini. El tampó fosfat és intracel·lular i el sanguini és el de bicarbonat.

Gràcies a la orina i a la respiració, es controlen el bicarbonat i el CO2.