%------------------------------------------------------------------------------
% Tema 15. Control del pH
%------------------------------------------------------------------------------

\section{Control del pH}
\label{sec:control-del-ph}

El control del pH és important per mantenir:
\begin{itemize}
\item Reaccions enzimàtiques cel·lulars
\item Transport de membrana
\item Receptors hormonals
\item Conformació de proteïnes de transport plasmàtic.
\end{itemize}

El pH de la sang és de 7,4 (6,8-7,8).

El CO2 o àcid carbònic s'elimina per exhalació pulmonar.

El CO2 pot provenir del metabolisme incomplet de compostos orgànics
per estats fisiològics transitoris o estats patològics, com en
l'exercici anaeròbic.

\subsection{Reaccions que alteren el pH}
\label{sec:reacc-que-alter}

\subsubsection{Fonts d'àcids}
\label{sec:fonts-dacids}


\subsubsection{Fonts d'àlcalis i consum de protons}
\label{sec:fonts-dalcalis}
Metabolisme de lactat i citrat de la dieta. Aquests elements es troben
sobretot en fruita o verdura. Les dietes vegetarianes són
alcalinitzants.

Els protons es perden quan hi ha vòmits (es perd HCl gàstric) i per
deficiències en el cicle de la urea. La pèrdua d'amoni és, en
definitiva, una pèrdua de protons.

\subsection{Sistemes tamponadors fisiològics}
\label{sec:sist-tamp-fisi}

La meitat dels àcids metabòlics són neutralitzats per les bases de la
dieta. La resta d'àcids són neutralitzats per tampons intracel·lulars
i sanguinis.

% tampons

El tampó ha de ser una parella d'àcid/base conjugada amb un pK proper
al pH sanguini. El tampó fosfat és intracel·lular i el sanguini és el
de bicarbonat.

Gràcies a la orina i a la respiració, es controlen el bicarbonat i el
CO2.

\subsection{Mecanismes de compensació del pH}
\label{sec:mecan-de-comp}

Hi ha diferents mecanismes de control:
\begin{enumerate}
\item Metabòlics

\item Respiratoris

\item Renals: 
\end{enumerate}

\subsection{Alteracions de l'equilibri àcid-base}
\label{sec:alter-de-lequ}

Si els trastorns afecten:
\begin{itemize}
\item Bicarbonat i altres: Trastorns metabòlics
\item CO2: Respiratoris
\end{itemize}

Segons el resultat de l'equilibri:
\begin{itemize}
\item Acidosi:
\item Alcalosi:
\end{itemize}

El GAP o dèficit d'anions és la diferència entre la concentració al
sèrum dels cations o anions mesurats sistemàticament. El GAP d'anions
és Na + K - Cl - bicarbonat. Inclou els fosfats i sulfats del metabolisme
cel·lular, el lactat i els cetoàicds de la oxidació incompleta dels
hidrats de carboni i àcids grassos i les proteïnes (sobretot
albúmina).

En principi, el GAP hauria de donar 0 però de manera natural no és
igual a 0.

\subsubsection{Acidosi metabòlica}
\label{sec:acidosi-metabolica}
El pH és menor a 7,35. 

\begin{itemize}
\item Amb GAP normal
  \begin{enumerate}
  \item Pèrdua de bicarbonat
    \begin{itemize}
    \item Renal: El túbul proximal no reabsorbeix el bicarbonat
      filtrat.
      \begin{itemize}
      \item Acidosi tubular proximal aïllada o associada al síndrome
        de Fanconi
      \item Acidosi per dilució (augment del volum plasmàtic)
      \item Inhbidors de l'anhidrasa carbònica
      \item Hiperparatiroïdisme 
      \end{itemize}
    \item Digestiva:
      \begin{itemize}
      \item Augmenten secrecions de bicarbonat al pàncrees
      \item Diarrea
      \item Drenatge de l'intestí prim
      \end{itemize}
    \end{itemize}
  \item Insuficient regeneració de bicarbonat al túbul distal
    \begin{itemize}
    \item Acidosi tubular distal: Les cèl·lules del tubul distal
      s'acidifiquen.
    \item Hipoaldosteronisme
    \item Diürètics (espironolactona)
    \end{itemize}
  \item Substàncies productores de protons a la dieta: Sals
    acidificants com el NH4Cl, Lys, Arg, Hys.
  \end{enumerate}

\item Amb GAP augmentat
  \begin{enumerate}
  \item Excreció reduïda d'àcids inorgànics (insuficiència renal)
  \item Gran producció d'àcids orgànics
    \begin{itemize}
    \item Acidosi làctica
      \begin{itemize}
      \item Hipòxia tissular: xoc, anèmia greu, intoxicació amb CO,
        exercici físic
      \item Ingesta d'etanol
      \item Leucèmia, tumors
      \item Dèficit congènit d'enzims del metabolisme de glúcids
      \end{itemize}
    \item Cetoacidosi: alcohòlica, diabètica, dejuni
    \end{itemize}
  \item Ingesta de certs productes com: etilenglicol, metanol, salicilats
  \end{enumerate}
\end{itemize}

Els mecanismes compensadors són:
\begin{itemize}
\item Respiratoris: Hiperventilació per eliminar CO2.
\item Renals: Eliminació d'àcids per la bomba Na+/H+, formació
  d'amoni, reabsorció de bicarbonat.
\end{itemize}

\subsubsection{Alcalosi metabòlica}
\label{sec:alcalosi-metabolica}
\begin{enumerate}
\item Administració excessiva de bases: carbonat sòdic, citrat en
  transfusions, antiàcids gàstrics.

\item Contracció del vòlim plamsàtic amb pèrdua intestinal de protons:
  pèrdua d'àcid gàstric per vòmits, aspiració nasogàstrica o drentage
  gàstroc. També alcalosi congènita amb diarrea, els transportadors de
  Cl- del budell no funcionen i s'hiperactiven les secrecions
  gàstriques i pancreàtiques fet que ocasiona una diarrea per pèrdua
  de líquids. També s'anomena clororrea.

\item Contracció del volum plasmàtic amb pèrdua renal de protons
  \begin{itemize}
  \item Tractament crònic amb diürètics
  \item Posthipercàpnia: Situació de xoc, ofec...
  \item Hipopotassèmia greu 
  \end{itemize}

\item Hiperfunció suprarenal
  \begin{itemize}
  \item Hiperaldosteronisme: Activa al transportador de Na/K/H. Capta
    molt Na i expulsa K i H.
  \item Síndromde Cushing
  \item Síndrome de Bartter: Defecte en transportadors de Na a la
    nansa ascendent de Henle i es perd sodi a la orina. Llavors, al
    tubul distal la bomba Na/K/H capta aquest sodi i expulsa K i H.
  \item Administració exògena de mineralocorticoides
  \end{itemize}

\item Altres: Transfusions sanguínies consecutives.
\end{enumerate}

Pel que fa a la respiració, s'hipoventila. El problema de la
hipoventilació és que causa una caiguda d'oxigen arterial que pot
acabar en una hiperventilació.

L'alcalosi prolongada necessita l'administració de potassi ja que la
bomba llança K. La orina és àcida ja que es llencen protons (acidúria
paradoxal).

\subsubsection{Acidosi respiratòria}
\label{sec:acidosi-respiratoria}

Els pulmons no eliminen CO2. Hi ha hipercàpnia.

\begin{enumerate}
\item Factors que inhibeixen el centre respiratori:
  \begin{itemize}
  \item Fàrmacs: narcòtics o barbitúrics
  \item Infeccions: encefalitis o meningitis
  \item Traumatismes i tumors al SNC
  \item Coma
  \end{itemize}
\item Factors que afecten els pulmons:
  \begin{itemize}
  \item Obstrucció pulmonar crònica
  \item Fibrosi pulmomar
  \item Asma
  \item Infeccions pulmonars greus
  \item Pneumotòrax i vessament pleural
  \end{itemize}
\item Altres
  \begin{itemize}
  \item Laringoespasmes
  \item Tumors en vies respiratòries superiors
  \item Greus distensions abdominals
  \item Obesitat molt greu
  \item Apnea nocturna
  \end{itemize}
\end{enumerate}

Gairebé mai es pot compensar a nivell respiratori. Al ronyó augmenta
l'eliminació d'àcids, augmenta l'eliminació d'amoni i la reabsorció de
bicarbonat.

\subsubsection{Alcalosi respiratòria}
\label{sec:alcal-resp}

Hiperventilació.

\begin{enumerate}
\item Factors que estimulen el centre respiratori
  \begin{itemize}
  \item 
  \end{itemize}
\item Fàrmacs
  \begin{itemize}
  \item 
  \end{itemize}
\item Ventilació mecànica excessiva
\end{enumerate}

Es compensa per tampons tissulars i renals. 

% Taula resum