%------------------------------------------------------------------------------
% Tema 11. Aparell digestiu
%------------------------------------------------------------------------------

\section{Aparell digestiu}
\label{sec:aparell-digestiu}

\subsection{Malabsorció}
\label{sec:malabsorcio}

\subsection{Secrecions bucals}
\label{sec:secrecions-bucals}

\subsection{Secrecions gàstriques}
\label{sec:secr-gastr}

\subsection{Secrecions intestinals}
\label{sec:secr-intest}

\subsubsection{Estudi de la funció pancreàtica}
\label{sec:estudi-de-la}

El marcador de la pancreatitis és la lipasa. També pot donar
informació sobre una obstrucció biliar, que pot desencadenar una
pancreatitis.

La determinació es fa amb l'addició d'1,2-diglicèrid de cadena llarga,
i a través de reaccions acoblades. L'altre mètode és amb una suspensió
de toleïna i mesurant la terbolesa.

La tripsina és un marcador de pancreatitis crònica i fibrosi cística.

La quimiotripsina es detecta en cordó umbilical ja que en neonats amb
fibrosi cística està augmentada. En adults és marcador de pancreatitis
aguda.

Es detecta amb una digestió de la gelatina sobre un film de rajos X.

Les secrecions es poden analitzar directament (tant el seu volum com
la composició). S'ha de fer una intubació nasoduodenal. Les secrecions
pancreàtiques s'estimulen per ingesta o infusió duodenal d'aminoàcids
(test de Lundh). També es pot fer una administració intravenosa de
secretina-CCK per estimular el pàncrees exocrí.

Es mesuren:
\begin{itemize}
\item Volum total del suc pancreàtic (150 mL/h)
\item Bicarbonat 90 nM
\item Activitats enzimàtiques
\end{itemize}

De manera no invasiva es poden mesurar:
\begin{itemize}
\item Femta: Aliments no absorbits o enzims pancreàtics
\item Orina/Plasma/Respiració:
  \begin{itemize}
  \item Quimiotripsina: S'administra oralment
    N-benzoil-L-tirosil-p-aminobenzoic i s'analitza el PABA en orina.
  \item Colesterol esterasa: Detecció de fluoresceïna en orina
  \item Productes de la digestió, marcats amb isòtops radioactius
  \item Hormones, aminoàcids i enzims
  \end{itemize}
\end{itemize}

\subsubsection{Estudi de la funció intestinal}
\label{sec:estudi-de-la-1}

Les principals alteracions són:
\begin{itemize}
\item Diarrea
\item Celiaquia
\item Malaltia de Crohn: Malaltia inflamatòria crònica. Pot afectar
  qualsevol part del tracte gastrointestinal però afecta sobretot el
  colon dret i l'ili distal.
\item Colitis ulcerosa: Malaltia inflamatòria recurrent de l'intestí
  gruixut. Malabsorció i dolors abdominals.
\item Fibrosi cística
\item Adenocarcinoma de colon
\item Adenocarcinoma rectal
\end{itemize}

La xilosa s'usa com a marcador de monosacàrids. S'absorbeix  a
l'intestí però no es pot metabolitzar. S'agafen mostres de sang/orina
a diferents temps.

Les alteracions més importants en absorció de glúcids són en els
disacàrids (intolerància a la lactosa). S'administra lactosa i
s'analitzen glucosa i galactosa en 1h.

L'hidrogen exhalat correlacions negativament amb els glúcids ingerits.

Els aminoàcids s'analitzen en orina. La idea és que si els
transportadors d'aminoàcids intestinals estan afectats, també estan
afectats al ronyó.

El test d'absorció de lípids es basa en Negre Sudan ingesta de lípids
i Negre Sudan. Es comencen a recollir les femtes negres, i fins que
torni a sortir el color negre.

Els lípids en femtes s'analitzen per gravimetria. S'extreuen els
lípids amb dissolvents orgànics i es deixa evaporar, s'han de pesar
abans i després de la evaporació.

Es pot usar marcatge radioactiu.

Femtes:
\begin{itemize}
\item Mala olor: malabsorció de proteïnes
\item Líquides-àcides: glúcids
\item Greixoses: lípids
\end{itemize}