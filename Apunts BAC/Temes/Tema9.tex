%------------------------------------------------------------------------------
% Tema 9. Disfuncions del metabolisme lipídic
%------------------------------------------------------------------------------
\section{Disfuncions del metabolisme lipídic}
\label{sec:disf-del-metab}

\subsection{Conceptes generals}
\label{sec:conceptes-generals}

\subsubsection{Tipus de lipoproteïnes, composició i estructura}
\label{sec:tipus-de-lipopr}

% Posar taula de composició

Les VLDL i els QM es buiden de la mateixa manera, amb la LPL. Les LDL
i QM romanents estan enriquits en colesterol. El colesterol de les LDL
és endogen i s'exporta als teixits. Els QM romanents van al fetge.

Les apoproteïnes tenen funcions estructurals, metabòliques,
reconeixement de receptors cel·lulars.

L'embolcall està format per apoproteïnes, fosfolípids en forma de
monocapa i colesterol lliure. El nucli està format per èsters de
colesterol, TAG.

% Taula apoproteïnes

ApoB està codificat per 1 gen que expressa 2 transcrits: apoB48
s'expressa a intestí (proteïna curta generada per splicing alternatiu)
i apoB100 expressa a fetge. ApoB100 s'uneix al receptor de LDL per la
part C-terminal.


\subsubsection{Metabolisme de les lipoproteïnes}
\label{sec:metabolisme-de-les}



\subsection{Paràmetres analítics}
\label{sec:parametres-analitics}
En cas d'una dislipèmia (alteració dels nivells de lípids circulants),
en primer lloc s'ha de confirmar amb 2 analítiques separades per un
interval de 2 setmanes. Es descarten possibles causes secundàries,
com:
\begin{itemize}
\item Diabetis (glucèmia)
\item Hepatopaties (transaminases)
\item Hipotiroïdisme (TSH)
\item Síndrome nefròtic (proteïnúria)
\item Paraproteïnes (electroforesi del plasma)
\end{itemize}

Si no hi ha causes secundàries, s'ha d'establir el diagnòstic
fenotípic i genètic de les diferents dislipèmies primàries.

La separació de les lipoproteïnes es pot fer mitjançant:
\begin{itemize}
\item Centrifugació seqüencial o en gradient de densitat. La via més
  ràpida és fer una ultracentrifugació en gradient de densitat de
  sacarosa. Permet obtenir les fraccions i analitzar-les per separat.

\item Electroforesi: Dóna una imatge general del perfil de
  lipoproteïnes. Es fa en agarosa o acetat de cel·lulosa. La mobilitat
  de les lipoproteïnes ve determinada per la quanitat de proteïna. Les
  que migren menys són les més grans i amb menys proteïna (QM).
\end{itemize}

També es mira l'aspecte del plasma reposat 12h a 4ºC:
\begin{itemize}
\item 1r tub dislipèmia: Les partícules que suren són poc
  denses, són QM. Coloració blanquinosa associada a
  TAG. Hipertriacilgliceridèmia d'origen exogen.

\item 2n tub dislipèmia: Tenen TAG lligats a
  VLDL. Hipertriacilgliceridèmia d'origen endogen.

\item 3r tub displipèmia: Hipertriacilgliceridèmia combinada.
\end{itemize}
 
\subsubsection{Lipoproteïnes}
\label{sec:lipoproteines}
Al sèrum del pacient s'afegeix PEG, concavalina o dextrà per
precipitar les proteïnes que continguin apoA i apoB. Es poden usar
anticossos. Després de centrifugar, al sobrenedant queda el colesterol
HDL i al pellet les LDL i VLDL. 

Normalment, es determina el colesterol HDL i al LDL es calcula amb una
fórmula basada en les determinacions anteriors. El mètode de
Friedewald no es pot aplicar si hi ha QM presents en sang. Es
considera que l'error en determinar el colesterol de VLDL és superior
al comès per una relació matemàtica. Per tant, es quantifiquen els TAG
en VLDL i es calcula la quantitat de colesterol en VLDL. Friedewald no
es pot aplicar quan hi ha QM en plasma i quan es supera el rang normal
de TAG (300 mg/dL) o en disbetalipoproteïnèmies.

\subsubsection{Colesterol}
\label{sec:colesterol}
Informa de la quantitat de colesterol total del plasma. Els valors
normals estan al voltant de 200-220 nm. Hi ha diferents mètodes:
\begin{itemize}
\item \textbf{Mètode de Liebermann-Burchard (químic):} És el mètode de
  referència. És una reacció del colesterol amb àcid acètic i
  sulfúric. Es produeix un cromogen degut a la modificació de
  l'hidroxil i es pot llegir a 620 nm (groc).

\item \textbf{Mètode enzimàtic (reacció de Trinder):} És el més usat en
  clínica. Dóna un cromogen vermell a 500 nm. Es reacciona el
  colesterol amb colesterol esterasa per generar colesterol lliure,
  colesterol oxidasa que genera peròxid d'hidrogen i finalment amb
  peroxidasa. La reacció amb peroxidasa és comuna a moltes
  determinacions.

\item \textbf{Química seca}
\end{itemize}

\subsubsection{Triacilglicerols}
\label{sec:triacilglicerols}
Hi ha 2 mètodes de determinació:
\begin{itemize}
\item \textbf{Mètode químic:} Extracció de lípids del plasma amb cloroform,
  saponificació i quantificació del glicerol alliberat (s'oxida fins a
  formiat amb iodat sòdic i després es fa reaccionar amb àcid
  cromotròpic – coloració rosa – i lectura a 570 nm).

\item \textbf{Mètode enzimàtic:}
\end{itemize}

% Posar valors de referència

\subsection{Dislipèmies}
\label{sec:dislipemies}
Segons el patró electroforètic i l'aspecte del plasma es poden
classificar les difents hiperlipèmies. La OMS ha establert 5 grans
classes d'hiperlipèmies:
\begin{enumerate}[label=\itembolasrojas{\arabic*}]
\item Tipus I. Quilomicronèmia familiar. Hu ha una banda en
  origen. Alteracions en TAG d'origen exogen.

\item Tipus II. Hipercolesterolèmia:
  \begin{enumerate}[a)]
  \item Tipus IIa: Presenta plasma normal, l'electroforesi mostra una
    banda beta mol marcada, LDL i colesterol. Risc cardiovasular
    associat.
  \item Tipus IIb: Plasma blanquinós. TAG augmentats d'origen
    endogen. Bandes beta i pre-beta (LDL i VLDL). Hipertrigliceridèmia
    i hipercolesterolèmia.
  \end{enumerate}

\item Tipus III: Plasma tèrbol i amb un sobrenedant
  blanquinós. L'electroforesi no diferencia la beta i la
  pre-beta. S'anomena disbetalipoproteïnèmia. Expressió en homozigosi
  d'un al·lel de l'apoE (apoE2) tenen predisposició a
  disbetalipoproteïnèmia, només un 1\% dels homozigots generen
  disbetalipoproteïnèmia. S'acumulen IDL i QMr, això explica la
  continuïtat de bandes pre beta io beta.

\item Tipus IV: Plasma tèrbol i increment de VLDL.

\item Tipus V: Plasma tèrbol i sobrenedant blanquinós. Apareixen molts
  QM i VLDL.
\end{enumerate}

Les tipus II i IV són bastant comuns en la població.

\subsubsection{Hiperlipoproteïnèmies primàries}
\label{sec:hiperl-prim}
%% Taula

\begin{itemize}
\item \textbf{Quilomicronèmia familiar:} Els nens amb dèficit en LPL no podran
  processar QM. Es manifesta a la lactància ja que la llet materna té
  un contingut important en TAG. S'han de donar àcids grassos de
  cadena mitja i curta (no van per QM) i àcids grassos essencials. El
  dèficit en apoC2 es tracta amb infusions de plasma ric en
  apoC2. apoC2 està implicada en el reconeixement de la LPL. Pot
  ocasionar pancreatitis. No hi ha risc cardiovascular.

\item \textbf{Hipercolesterolèmia familiar:} Mutacions al receptor de LDL o
  apoB. El receptor de LDL és un gen al qual s'han descrit moltes
  mutacions. Arg3500->Gln3500 és endèmica a Suïssa. També hi pot haver
  un augment de la síntesi d'apoB100. Es manifesten cutàniament amb
  xantomes...

\item \textbf{Disbetalipoproteïnèmia familiar:} 1/10.000 manifesta
  aquesta dislipèmia. Hi ha factors ambientals que influeixen en la
  seva manifestació. Està afectada la unió amb LRP, és a dir la
  retirada de QMr, IDL, beta-VLDL. Les beta-VLDL tenen menys TAG però
  més contingut en colesterol. Hi ha hipertrigliceridèmia. Es pot
  mirar la relació $\dfrac{TAG}{colesterol}$, si és 5 els resultats
  són normals però si és inferior a 3,33 es sospita de
  disbetalipoproteïnèmia familiar. Presenten xantomes, sobrepès,
  intolerància a la glucosa.

\item Hipertrigliceridèmia familiar: No s'associa a un genotip
  concret. La de tipus IV és molt comú. Es manifesta com a obesitat i
  intolerància a la glucosa.
\end{itemize}

El dèficit de LCAT (LCAT esterifica colesterol lliure dels teixits i
el carrega a HDL). Es manifesta com un increment de colesterol lliure
i aterosclerosi prematura.

Aquests fenotips evolucionen amb el temps.

\subsubsection{Hiperlipoproteïnèmies secundàries}
\label{sec:hiperl-secund}
Hiperlipoproteïnèmies provocades per altres malalties.

% Taula

\begin{itemize}
\item La colestasi és una obstrucció de les vies biliars. Hi ha un reflux de
bilis cap al fetge i després cap a circulació sistèmica. La bilis a la
sang s'associa a lipoproteïnes. Apareixen lipoproteïnes X riques en
colesterol, pobres en apo, riques en albúmina, migren cap al
càtode. La lipoproteïna X és la bilis empaquetada com a
lipoproteïna. També apareix colesterol lliure. La lipoproteïna X està
formada per un 90\% per lípids i colesterol lliure i la resta és 
albúmina i apoC i migren cap al càtode (negatiu).

\item L'hipotiroïdisme es manifesta amb un increment de colesterol circulant
ja que el LDLR està regulat per hormones tiroïdals.

\item El síndrome nefròtic genera un increment de colesterol.

\item La lipoproteïna a és una lipoproteïna normal que es troba
circulant. La composició és molt similar a la LDL. apoB100 presenta un
pont disulfur amb una serina proteasa de dominis KRINGLE 4 i 5. Són
molt similars als del plasminogen. La lipoproteïna a no té activitat
serina proteasa activa però pot competir amb el plasminogen i inhibir
la seva funció (deficiència de fibrinòlisi). L'excés de lipoproteïna a
provoca problemes a la fibrinòlisi i hipercolesterolèmia ja que no és
reconeguda pel LDLR. La teràpia és la convencional per
hipercolesterolèmia.
\end{itemize}


\subsubsection{Hipolipoproteïnèmies}
\label{sec:hipolipoproteinemies}
Dins de les hipolipoproteïnèmies primàries:
\begin{itemize}
\item \textbf{HDL afectades:}
  \begin{itemize}
  \item Analfalipoproteïnèmia (manca de HDL): No es retira colesterol
    dels teixits ni pas a LDL. Síntesi defectuosa d'apoAI.
 
  \item Malaltia de Tangier: Hipercolesterolèmia amb dipòsits de
    colesterol als teixits. Defecte a ABCA1, transportador de
    colesterol cap a fora de les cèl·lules. Les HDL interaccionen amb
    ABCA1 per recollir el colesterol sobrant. Es va descobrir a l'illa
    de Tangier degut a l'efecte fundador.
  \end{itemize}

\item \textbf{LDL afectades:}
  \begin{itemize}
  \item Abetalipoproteïnèmia: Mutació en el gen MTP autosòmica
    recessiva. MTP és el \textit{Microsomal TAG transfer protein}. No
    es formen QM ni VLDL. Hi ha un dèficit de vitamines
    liposolubles. Hi ha hipercolesterolèmia i absència total d'apoB100
    en plasma.

  \item Hipobetalipoproteïnèmia: La mutació al gen apoB és autosòmica
    dominant que genera una forma truncada d'apoB100. La síntesi de
    LDL està molt afectada, la de VLDL i QM no tant.
  \end{itemize}

\item Malaltia d'Anderson: Malaltia de retenció de QM. Es detecta en
  nadons ja que la llet materna és rica en lípids. Hi ha un defecte en
  la sortida de QM a limfa.
\end{itemize}

\subsubsection{Tractament}
\label{sec:tractament}

Diverses estratègies:
\begin{itemize}
\item Estatines: Inhibidor dl'HMG-CoA-reductasa
\item Resines segrestadores d'àcids biliars
\item Fibrats: Activen la LPL i afavoreixen la oxidació d'àcids
  grassos al fetge.
\item Nicotina: Inhibeix la lipòlisi al teixit adipós i la sortida de
  TAG al fetge.
\item Omega 3: Actuen sobre l'ACAT (esterifica colesterol a les
  cèl·lules) i activa SREBP permetent la síntesi de nous LDLR.
\end{itemize}