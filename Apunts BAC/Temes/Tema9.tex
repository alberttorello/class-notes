%------------------------------------------------------------------------------
% Tema 9. Disfuncions del metabolisme lipídic
%------------------------------------------------------------------------------
\section{Disfuncions del metabolisme lipídic}
\label{sec:disf-del-metab}

\subsection{Conceptes generals}
\label{sec:conceptes-generals}

\subsubsection{Tipus de lipoproteïnes, composició i estructura}
\label{sec:tipus-de-lipopr}

% Posar taula de composició

Les VLDL i els QM es buiden de la mateixa manera, amb la LPL. Les LDL i QM romanents estan enriquits en colesterol. El colesterol de les LDL és endogen i s'exporta als teixits. Els QM romanents van al fetge.

Les apoproteïnes tenen funcions estructurals, metabòliques, reconeixement de receptors cel·lulars.

L'embolcall està format per apoproteïnes, fosfolípids en forma de monocapa i colesterol lliure. El nucli està format per èsters de colesterol, TAG.

% Taula apoproteïnes

ApoB està codificat per 1 gen que expressa 2 transcrits: apoB48 s'expressa a intestí (proteïna curta generada per splicing alternatiu) i apoB100 expressa a fetge. ApoB100 s'uneix al receptor de LDL per la part C-terminal.


\subsubsection{Metabolisme de les lipoproteïnes}
\label{sec:metabolisme-de-les}



\subsection{Paràmetres analítics}
\label{sec:parametres-analitics}


\subsection{Lipoproteïnes}
\label{sec:lipoproteines}


\subsubsection{Colesterol}
\label{sec:colesterol}


\subsubsection{Triacilglicerols}
\label{sec:triacilglicerols}



\subsection{Dislipèmies}
\label{sec:dislipemies}

\subsubsection{Hiperlipoproteïnèmies primàries}
\label{sec:hiperl-prim}


\subsubsection{Hiperlipoproteïnèmies secundàries}
\label{sec:hiperl-secund}


\subsubsection{Hipolipoproteïnèmies}
\label{sec:hipolipoproteinemies}

