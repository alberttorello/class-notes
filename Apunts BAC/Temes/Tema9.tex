%------------------------------------------------------------------------------
% Tema 9. Disfuncions del metabolisme lipídic
%------------------------------------------------------------------------------
\section{Disfuncions del metabolisme lipídic}
\label{sec:disf-del-metab}

\subsection{Conceptes generals}
\label{sec:conceptes-generals}

\subsubsection{Tipus de lipoproteïnes, composició i estructura}
\label{sec:tipus-de-lipopr}

% Posar taula de composició

Les VLDL i els QM es buiden de la mateixa manera, amb la LPL. Les LDL
i QM romanents estan enriquits en colesterol. El colesterol de les LDL
és endogen i s'exporta als teixits. Els QM romanents van al fetge.

Les apoproteïnes tenen funcions estructurals, metabòliques,
reconeixement de receptors cel·lulars.

L'embolcall està format per apoproteïnes, fosfolípids en forma de
monocapa i colesterol lliure. El nucli està format per èsters de
colesterol, TAG.

% Taula apoproteïnes

ApoB està codificat per 1 gen que expressa 2 transcrits: apoB48
s'expressa a intestí (proteïna curta generada per splicing alternatiu)
i apoB100 expressa a fetge. ApoB100 s'uneix al receptor de LDL per la
part C-terminal.


\subsubsection{Metabolisme de les lipoproteïnes}
\label{sec:metabolisme-de-les}



\subsection{Paràmetres analítics}
\label{sec:parametres-analitics}
En cas d'una dislipèmia (alteració dels nivells de lípids circulants),
en primer lloc s'ha de confirmar amb 2 analítiques separades per un
interval de 2 setmanes. Es descarten possibles causes secundàries,
com:
\begin{itemize}
\item Diabetis (glucèmia)
\item Hepatopaties (transaminases)
\item Hipotiroïdisme (TSH)
\item Síndrome nefròtic (proteïnúria)
\item Paraproteïnes (electroforesi del plasma)
\end{itemize}

Si no hi ha causes secundàries, s'ha d'establir el diagnòstic
fenotípic i genètic de les diferents dislipèmies primàries.

La separació de les lipoproteïnes es pot fer mitjançant:
\begin{itemize}
\item Centrifugació seqüencial o en gradient de densitat. La via més
  ràpida és fer una ultracentrifugació en gradient de densitat de
  sacarosa. Permet obtenir les fraccions i analitzar-les per separat.

\item Electroforesi: Dóna una imatge general del perfil de
  lipoproteïnes. Es fa en agarosa o acetat de cel·lulosa. La mobilitat
  de les lipoproteïnes ve determinada per la quanitat de proteïna. Les
  que migren menys són les més grans i amb menys proteïna (QM).
\end{itemize}

\subsubsection{Lipoproteïnes}
\label{sec:lipoproteines}
Al sèrum del pacient s'afegeix PEG, concavalina o dextrà per
precipitar les proteïnes que continguin apoA i apoB. Es poden usar
anticossos. Després de centrifugar, al sobrenedant queda el colesterol
HDL i al pellet les LDL i VLDL. 

Normalment, es determina el colesterol HDL i al LDL es calcula amb una
fórmula basada en les determinacions anteriors. El mètode de
Friedewald no es pot aplicar si hi ha QM presents en sang. Es
considera que l'error en determinar el colesterol de VLDL és superior
al comès per una relació matemàtica. Per tant, es quantifiquen els TAG
en VLDL i es calcula la quantitat de colesterol en VLDL. Friedewald no
es pot aplicar quan hi ha QM en plasma i quan es supera el rang normal
de TAG (300 mg/dL) o en disbetalipoproteïnèmies.

\subsubsection{Colesterol}
\label{sec:colesterol}
Informa de la quantitat de colesterol total del plasma. Els valors
normals estan al voltant de 200-220 nm. Hi ha diferents mètodes:
\begin{itemize}
\item \textbf{Mètode de Liebermann-Burchard (químic):} És el mètode de
  referència. És una reacció del colesterol amb àcid acètic i
  sulfúric. Es produeix un cromogen degut a la modificació de
  l'hidroxil i es pot llegir a 620 nm (groc).

\item \textbf{Mètode enzimàtic (reacció de Trinder):} És el més usat en
  clínica. Dóna un cromogen vermell a 500 nm. Es reacciona el
  colesterol amb colesterol esterasa per generar colesterol lliure,
  colesterol oxidasa que genera peròxid d'hidrogen i finalment amb
  peroxidasa. La reacció amb peroxidasa és comuna a moltes
  determinacions.

\item \textbf{Química seca}
\end{itemize}

\subsubsection{Triacilglicerols}
\label{sec:triacilglicerols}
Hi ha 2 mètodes de determinació:
\begin{itemize}
\item \textbf{Mètode químic:} Extracció de lípids del plasma amb cloroform,
  saponificació i quantificació del glicerol alliberat (s'oxida fins a
  formiat amb iodat sòdic i després es fa reaccionar amb àcid
  cromotròpic – coloració rosa – i lectura a 570 nm).

\item \textbf{Mètode enzimàtic:}
\end{itemize}

% Posar valors de referència

\subsection{Dislipèmies}
\label{sec:dislipemies}

\subsubsection{Hiperlipoproteïnèmies primàries}
\label{sec:hiperl-prim}


\subsubsection{Hiperlipoproteïnèmies secundàries}
\label{sec:hiperl-secund}


\subsubsection{Hipolipoproteïnèmies}
\label{sec:hipolipoproteinemies}

