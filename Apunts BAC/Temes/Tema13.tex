%------------------------------------------------------------------------------
% Tema 13. Funció tiroïdal
%------------------------------------------------------------------------------

\section{Funció tiroïdal}
\label{sec:funcio-tiroidal}

% Regulació per hipotàlem-hipòfisi

\subsection{Funció tiroïdal}
\label{sec:funcio-tiroida}

L'hipotàlem secreta TRH (factor positiu) i la somatostatina (no
hipotalàmica, factor negatiu), van a la hipòfisi i es secreta TSH, el
TSH estimula la producció d'hormones tiroïdals. La T3 i T4 inhibeixen
la producció de TSH i TRH (\textit{feedback} negatiu).

En un dèficit de iode en la dieta; baixa la producció de T4 i es manté
la de T3, incrementen els nivells de TSH en resposta a la baixada de
T4 i llavors hi ha una hipertròfia del tiroide; es capta millor el
iode però pot produir goll.


Les hormones tiroïdals tenen accions fisiològiques molt variats:

\begin{table}[H]
\centering
\begin{tabular}{>{\centering}m{8cm}>{\centering}m{8cm}}
\hline 
\textbf{Efectes sobre diferenciació i creixement} & \textbf{Efectes metabòlics}\tabularnewline
\hline 
Diferenciació del sistema nerviós en el desenvolupament (T3) & Increment de la taxa metabòica basal\tabularnewline
Formació de surfactants pulmonars & Regulació de balanç mineral\tabularnewline
Factor permissiu per l'acció de la GH: creixement & Metabolisme de carbohidrats, lípids i proteïnes\tabularnewline
Metamofosi d'amfibis & Sinèrgies o antagonismes amb altres hormones (insulina, esteroïdals,
catecolamines)\tabularnewline
\hline 
\end{tabular}
\end{table}

\subsubsection{Síntesi d'hormones tiroïdals}
\label{sec:sint-dhorm-tiro}

La tiroides es troba entre el cartílag d'Adam i la tràquea i està
formada per dos lòbuls. Presenta una sèrie de fol·licles format per
una única capa de cèl·lules fol·liculars polaritzades amb la zona
basal i apical. Al centre hi ha el col·loide.  Al voltant de la capa
fol·licular hi ha cèl·lules C, que produeixen calcitonina.

El col·loide conté la tiroglobulina; una glicoproteïna iodada dimèrica
i amb 300 kDa per subunitat. A la seva seqüència d'aminoàcids hi ha
n(Tyr). En aquestes Tyr s'incorpora el iode concretament al grup OH:
aquestes estructures s'anomenen tironines. La TPO (peroxidasa
tiroïdal) transloca una tironina a una tironina propera. Per obtenir
T4 hi ha una proteòlisi específica. Es produeix un 90\% de T4 i un
10\% de T3.

En aquest procés es pot produir un únic residu de tironina anomenat
DIT (diiodotirosina) o també es pot generar MIT (monoiodotirosina). El
tiroides desioda aquests compostos i recupera el iode per
reincorporar-lo a la tiroglobulina. La tiroglobulina s'emmagatzema al col·loide.

La pèrdua d'un iode a l'anell extern en T3 canvia l'afinitat de la
molècula pel receptor nuclear: aquest reconeix de forma molt
específica i afí a T3. Així, en parlar de les accions de les hormones
tiroïdals, estarem parlant bàsicament de les accions de T3.

Les cèl·lules tiroïdals tenen 2 circuits de tràfic: cèl·lula ->
col·loide; col·loide -> cèl·lula. La tiroglobulina es transloca de la
cèl·lula cap al col·loide; i a fora de la cèl·lula actua la TPO. En
primer lloc, la TPO fa la organificació del iode. El iode s'ha de
transportar des de la sang cap al col·loide: a nivell basal hi ha NIS,
co-transportador amb sodi; per fer-la sortir al col·loide actua la
pendrina. Llavors la TPO genera T3 o T4. Quan hi ha necessitat
d'aquestes hormones, es vesiculen des del col·loide i es secreten a la
sang.

\subsubsection{Transport per sèrum}
\label{sec:transport-per-serum}
Les hormones tiroïdals són liposolubles; en sèrum aniran lligades a
proteïnes plasmàtiques. Es necessiten sistemes de transport cel·lular
específics (un per T3 i l'altre per T4). 

\begin{table}[H]
\centering
\begin{tabular}{>{\raggedright}m{5.2cm}>{\raggedright}m{5cm}}
\hline 
\textbf{Nivells en sèrum (/mL) en humans} & \textbf{Proteïnes de transport T3 i T4: \% de T4 transportada}\tabularnewline
\hline 
T4: 74 ng & TBG/TBP: 70\%\tabularnewline
T4 lliure: 23 pg (0.003\% T4 total) & TBPA (pre-albúmina): 5\%\tabularnewline
T3 total: 1.3 ng & Albúmina: 25\%\tabularnewline
T3 lliure: 3.2 pg (0.25\% T3 total) & \tabularnewline
\hline 
\end{tabular}
\end{table}

La TBG es sintetitza al fetge i en funció dels nivells circulants, es
regula la quantitat de T4 lliure en sèrum. TBG és sensible a
estrògens, els estrògens estan alts durant la gestació i disminueixen
la T4 lliure en sang. La gestació afavoreix un estat
pseudohipotiroïdal.

Els nivells en sèrum són:
\begin{itemize}
\item T4 sèrum: 5,5-12,5 ug/100 mL (72-163 nM)
\item T3 sèrum: 100-200 ng/100 mL (1,5-3,4 nM)
\item T4 lliure en sèrum: 1,5 pM
\item T3 lliure en sèrum: 5,0 pM
\end{itemize}

T3 té una elevada afinitat pel receptor nuclear, així que formarà amb
molta facilitat els heterodímers RXR-T3R. La T4 es desioda i es
transforma a T3, i així augmenta la T3 disponible dins la
cèl·lula. Les desiodacions seran activadores o inhibidores depenent de
la posició del iode eliminat. Aquesta capacitat es limita a certs
teixits de l'organisme, i en cas de necessitat la poden exportar a la
sang per augmentar la quantitat disponible de T3.

\subsubsection{Desiodació}
\label{sec:desiodacio}
Hi ha 3 tipus de desiodases, que catalitzen desiodacions en diferents
posicions de les hormones tiroïdals:

\begin{table}[H]
\centering
\begin{tabular}{c>{\centering}m{3cm}>{\centering}m{5cm}}
\cline{2-3} 
 & \textbf{Tipus de desiodació} & \textbf{Localització}\tabularnewline
\hline 
\textbf{D1} & 5'-desiodació 5-desiodació & Fetge, ronyó\tabularnewline
\textbf{D2} & 5'-desiodació & Hipòfisi, SNC, teixit adipós marró, múscul (humans)\tabularnewline
\textbf{D3} & 5-desiodació & Pell, placenta\tabularnewline
\hline 
\end{tabular}
\end{table}

Les desiodases de fetge i ronyó són les més importants. La
triiodotironina inversa es genera durant la desiodació i es considera
inactiva.

\subsubsection{Hipertiroïdisme}
\label{sec:hipertiroidisme}
L'estat eutiroïdal és l'estat normal de la tiroides. L'hipertiroïdisme
es caracteritza per una intolerància a la calor, sudoració,
palpitacions, taquicàrdies, pèrdua de pes, debilitat, hipertensió i
exoftàlmia (efecte d'hormones tiroïdals sobre músculs oculars).

En aquest cas, hi ha T4  total alta i T3 i T4 lliures elevats. La TSH
està disminuïda en l'hipertiroïdisme establert. TSH pot estar elevada.

Si hi ha una hipersecreció de TSH, es produeix goll (hiperplàsia de
tiroides). És l'hipertiroïdisme primari.

L'hipertiroïdisme secundari pot ser degut a un tumor hipofisiari
secretor de TSH, a un carcinoma de tiroides o a la malaltia de
Graves. La malaltia de Graves és una malaltia autoimmune que apareix
cap als 40 anys més freqüent en dones. Es creen anticossos contra el
receptor de la TSH, que l'activen constitutivament.

En el diagnòstic, s'ingereix iode radioactiu (I-131) i es capta una
radioimatge. La captació del I pot ser uniforme, nodular...

El tractament es pot fer amb I radioactiu en cas d'un tumor. Es pot
fer la resecció d'una part o de tota la glàndula. La levotiroxina és
la formulació farmacèutica de la T4. Es monitoritza mirant que TSH
estigui dins els rangs fisiològics.

\subsubsection{Hipotiroïdisme}
\label{sec:hipotiroidisme}
L'hipotiroïdisme es caracteritza per goll, baixa taxa metabòlica
(fred, debilitat), sobrepès, sequedat de pell...

Els nivells circulants de T3 i T4 circulants són baixos i els nivells
de TSH estan elevats.

L'hipotiroïdisme primari pot ser degut a un dèficit de iode a la
dieta, per cretinisme (congènit i problemes de desenvolupament i
retard mental). 

L'hipotiroïdisme secundari (atròfia de la glàndula) pot ser degut a
causes iatrogèniques degut a la ingesta de mandioca mal cuinada ja que
té tiocianat que pot passar a cianur. Aquest cianur es confon amb el
iode i bloqueja la síntesi d'hormones tiroïdals. La malaltia de
Hashimoto és una malaltia autoimmune en què es generen anticossos
contra la tiroglobulina, la TPO o altres proteïnes tiroïdals
importants. La malaltia de Hashimoto és relativament freqüent.

S'administra T4 i es controla la TSH.

\subsubsection{Marcadors clínics}
\label{sec:marcadors-clinics}
\begin{itemize}
\item Tiroxina: Es valora amb mètodes immunològics com el RIA, ELISA...
  \begin{itemize}
  \item Total: Cal la separació prèvia de la hormona i les seves
    proteïnes de transport.
  \item Lliure: Tractament previ del plasma, captació per reïnes o
    diàlisi del plasma.
  \end{itemize}

\item Anticossos contra hormones tiroïdals, proteïnes de la glàndula:
  ELISA o aglutinació.

\item Tiroglobulina: Quan es fa una resecció de la tiroides, es
  monitoritza la tiroglobulina. Serveix pel seguiment dels carcinomes
  de tiroides.

\item TSH: Es determina per RIA/ELISA. Es pot fer un test dinàmic amb
  l'administració de TRH.

\item Anticossos contra el receptor de TSH: Es comprova l'estimulació
  de cèl·lules fol·liculars amb plasma del pacient: si té anticossos
  contra TSH es veurà un augment d'AMPc en aquestes cèl·lules. Es
  poden fer assajos de competició amb plasma del pacient i TSH marcada
  radioactivament.
\end{itemize}

% Recordatori immunodifusió en gel

Es valora el radi de difusió dels immunocomplexos.

Les proteïnes migren del càtode a l'ànode. En funció de la quantitat
d'antigen, la cua serà proporcionalment més llarga
(immunoelectroforesi en coet).

La immunoelectroforesi primer és activa i després passiva.

Prova de Coomps: per detectar el grup sanguini.