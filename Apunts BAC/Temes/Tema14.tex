%------------------------------------------------------------------------------
% Tema 14. Estudi de la funció renal
%------------------------------------------------------------------------------
\section{Estudi de la funció renal}
\label{sec:estudi-de-la}

Els ronyons participen en:
\begin{itemize}
\item Excreció de nitrogen (urea i àcid úric)
\item Manteniment de l'equilibri electrolític i de pH
\item Manteniment de l'equilibri d'aigua
\item Producció i secreció d'hormones, degradació d'insulina, glucagó i aldosterona
\end{itemize}

L'estudi i avaluació de la funció renal es pot fer a partir de diversos paràmetres indicatius:
\begin{itemize}
\item Excreció de nitrogen (urea i àcid úric)
  \begin{itemize}
  \item Creatinina sèrica
  \item Depuració de creatinina
  \item Urea sèrica
  \item Urat sèric
  \end{itemize}
\item Manteniment de l'equilibri electrolític i de pH i manteniment de l'equilibri d'aigua
  \begin{itemize}
  \item Osmolitat d'orina i sèrum
  \item Sodi, potassi, clorur en orina i sèrum
  \item pH sang
  \item Bicarbonat
  \item Calci en orina i sèrum
  \item Fosfat i magnesi en sèrum
  \end{itemize}
\item Producció i secreció d'hormones, degradació d'insulina, glucagó i aldosterona
  \begin{itemize}
  \item Renina sèrica
  \item Eritropoietina sèrica
  \item 1,25-dihidroxicolecalciferol
  \end{itemize}
\end{itemize}

La orina és un producte orgànic format per filtració del plasma als ronyons i la seva transformació per aconseguir l'eliminació de les substàncies indesitjables pel sistema urinari.

% Anatomia i fisiologia del ronyó

L'escorça renal és isotònica (igual pressió osmòtica que a l'interior cel·lular, NaCl 150 mM) i la medul·la renal és hipertònica (més pressió osmòtica que a l'interior). El que determina la pressió osmòtica són els ions, sucres, proteïnes. Els més importants són els ions ja que hi ha més nombre absolut de ions per unitat de volum. Els ions més abundants són el Na, acompanyats de Cl i bicarbonats. El principal determinant de la pressió osmòtica de la sang és el Na.

Al glomèrul hi ha 3 capes: l'endoteli vascular fenestrat, la membrana basal i l'epiteli de podòcits. El que no s'ha de filtrar són les cèl·lules i proteïnes plasmàtiques. Les fenestracions de l'endoteli deixen passar proteïnes de fins a 60 kDa. La filtració depèn de la pressió sanguínia, l'exercici físic... L'epiteli de podòcits té sialoproteïnes a la membrana, que dónen càrrega negativa. El pI de l'albúmina és de 5, pel que a pH 7,4 té càrrega negativa; és a dir que l'epiteli de podòcits evita que les proteïnes el traspassin per repulsió electrostàtica.

El túbul proximal té molts transportadors i es recupera tot el sodi i pràcticament tot el potassi, glucosa, aminoàcids i aigua. Els transportadors són actius ja que estan concentrant els productes cap a fora el túbul proximal. 

A la branca descendent de la nansa de Henle es captura aigua per transport passiu (hipertònic). La branca ascendent de la nansa de Henle és impermeable a l'aigua però sí que transporta sodi i clorur cap a fora; aquest és el motiu pel qual la medul·la renal és hipetònica; també surten calci i magnesi.

Al túbul distal té un transportador de Na/K/protons. Quan agafa sodi, treu un potassi o un protó en funció de les concentracions que hi hagi. L'aldosterona regula aquest transportador. L'aldosterona promou l'acumulació de sodi. Al túbul distal es degrada glutamina i s'allibera amoníac, que és un gas neutre i que traspassa les membranes. L'amoníac reacciona amb els protons per formar ions amoni, que ja no passa per les membranes.

Al túbul col·lector hi ha un altre transportador Na/K/protons. Té aquaporines, que expulsen aigua i estan regulats per ADH. L'ADH activa la captació d'aigua al túbul col·lector.

% Patologies

Acidúria tubular renal: les cèl·lules del túbul s'acidifiquen a causa del transportador Na/K/protons, que no expulsa protons.