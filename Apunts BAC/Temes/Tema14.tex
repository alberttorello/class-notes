%------------------------------------------------------------------------------
% Tema 14. Estudi de la funció renal
%------------------------------------------------------------------------------
\section{Estudi de la funció renal}
\label{sec:estudi-de-la}

Els ronyons participen en:
\begin{itemize}
\item Excreció de nitrogen (urea i àcid úric)
\item Manteniment de l'equilibri electrolític i de pH
\item Manteniment de l'equilibri d'aigua
\item Producció i secreció d'hormones, degradació d'insulina, glucagó
  i aldosterona
\end{itemize}

L'estudi i avaluació de la funció renal es pot fer a partir de
diversos paràmetres indicatius:
\begin{itemize}
\item Excreció de nitrogen (urea i àcid úric)
  \begin{itemize}
  \item Creatinina sèrica
  \item Depuració de creatinina
  \item Urea sèrica
  \item Urat sèric
  \end{itemize}
\item Manteniment de l'equilibri electrolític i de pH i manteniment de
  l'equilibri d'aigua
  \begin{itemize}
  \item Osmolitat d'orina i sèrum
  \item Sodi, potassi, clorur en orina i sèrum
  \item pH sang
  \item Bicarbonat
  \item Calci en orina i sèrum
  \item Fosfat i magnesi en sèrum
  \end{itemize}
\item Producció i secreció d'hormones, degradació d'insulina, glucagó
  i aldosterona
  \begin{itemize}
  \item Renina sèrica
  \item Eritropoietina sèrica
  \item 1,25-dihidroxicolecalciferol
  \end{itemize}
\end{itemize}

La orina és un producte orgànic format per filtració del plasma als
ronyons i la seva transformació per aconseguir l'eliminació de les
substàncies indesitjables pel sistema urinari.

% Anatomia i fisiologia del ronyó

L'escorça renal és isotònica (igual pressió osmòtica que a l'interior
cel·lular, NaCl 150 mM) i la medul·la renal és hipertònica (més
pressió osmòtica que a l'interior). El que determina la pressió
osmòtica són els ions, sucres, proteïnes. Els més importants són els
ions ja que hi ha més nombre absolut de ions per unitat de volum. Els
ions més abundants són el Na, acompanyats de Cl i bicarbonats. El
principal determinant de la pressió osmòtica de la sang és el Na.

Al glomèrul hi ha 3 capes: l'endoteli vascular fenestrat, la membrana
basal i l'epiteli de podòcits. El que no s'ha de filtrar són les
cèl·lules i proteïnes plasmàtiques. Les fenestracions de l'endoteli
deixen passar proteïnes de fins a 60 kDa. La filtració depèn de la
pressió sanguínia, l'exercici físic... L'epiteli de podòcits té
sialoproteïnes a la membrana, que dónen càrrega negativa. El pI de
l'albúmina és de 5, pel que a pH 7,4 té càrrega negativa; és a dir que
l'epiteli de podòcits evita que les proteïnes el traspassin per
repulsió electrostàtica.

El túbul proximal té molts transportadors i es recupera tot el sodi i
pràcticament tot el potassi, glucosa, aminoàcids i aigua. Els
transportadors són actius ja que estan concentrant els productes cap a
fora el túbul proximal.

A la branca descendent de la nansa de Henle es captura aigua per
transport passiu (hipertònic). La branca ascendent de la nansa de
Henle és impermeable a l'aigua però sí que transporta sodi i clorur
cap a fora; aquest és el motiu pel qual la medul·la renal és
hipetònica; també surten calci i magnesi.

Al túbul distal té un transportador de Na/K/protons. Quan agafa sodi,
treu un potassi o un protó en funció de les concentracions que hi
hagi. L'aldosterona regula aquest transportador. L'aldosterona promou
l'acumulació de sodi. Al túbul distal es degrada glutamina i
s'allibera amoníac, que és un gas neutre i que traspassa les
membranes. L'amoníac reacciona amb els protons per formar ions amoni,
que ja no passa per les membranes.

Al túbul col·lector hi ha un altre transportador Na/K/protons. Té
aquaporines, que expulsen aigua i estan regulats per ADH. L'ADH activa
la captació d'aigua al túbul col·lector.

% Patologies

Acidúria tubular renal: les cèl·lules del túbul s'acidifiquen a causa
del transportador Na/K/protons, que no expulsa protons.

% Fins diap 20.

\subsection{Proves químiques}
\label{sec:proves-quimiques}

\subsubsection{Hemoglobina}
\label{sec:hemoglobina}

Quan hi ha hemòlisi, la hemoglobina es fixa en proteïnes. Si la
hemòlisi és molt intensa, la hemoglobina passa el filtrat glomerular i
s'excreta en orina.

Es determina per química seca amb una tira reactiva. Es pot determinar
també per ortotoluïdina i un peròxid orgànic, també hi pot reaccionar
la mioglobina, cal tenir-ho en compte si hi ha miòlisi. La
precipitació per sulfat d'amoni només precipita hemoglobina i no
mioglobina.

\subsubsection{Bilirubina}
\label{sec:bilirubina}

Prové de l'hemoglobina. És sensible a la llum. Si no està conjugada,
es transporta via albúmina i no es filtra. Quan es conjuga amb
glucoronat, es fa més soluble, es transporta lliure en sang i es pot
filtrar a la orina.

La bilirubinúria es pot donar:
\begin{itemize}
\item Infecció o intoxicació hepàtica
\item Obstrucció biliar, part de la bilirubina es conjuga, no va a la
  bilis, passa a la sang i es filtra a la orina.
\end{itemize}

\subsubsection{Urobilinogen}
\label{sec:urobilinogen}

L'urobilinogen és el producte format de la descomposició de la
bilirubina a l'intestí per la microbiota intestinal.

\subsubsection{Àcid úric}
\label{sec:acid-uric}

Producte final del catabolisme de les purines. Es forma al fetge i a
l'intestí. L'àcid úric s'elimina al ronyó, amb reabsorció al túbul
proximal. 

Pot augmentar en sang (uricèmia) quan hi ha insuficiència renal,
tractament amb diürètics, quimioteràpia i consum de nucleoproteïnes.

Si augmenta en sang, també augmenta en orina.

\subsubsection{Urea}
\label{sec:urea}

Producte final del catabolisme d'aminoàcids. Es forma al fetge i
s'excreta al ronyó.

La urèmia pot ser:
\begin{itemize}
\item Causes renals: Quan hi ha deshidratació, disminueix el flux
  renal i per tant l'excreció d'urea. També quan hi ha obstrucció de
  vies urinàries.
\item Altres causes: Quan augmenta la ingesta proteica o el
  catabolisme (en cas de febre o estrès).
\end{itemize}

\subsubsection{Creatinina}
\label{sec:creatinina}

És el producte de condensació de la creatina muscular. S'excreta pel
ronyó i la seva producció no depèn de la dieta.

S'usa com a prova funcional de la filtració glomerular ja que és
endògena i de producció constant.

L'aclariment serà major quan més creatinina es filtri. La creatinina
no es reabsorbeix.

\subsection{Examen del sediment urinari}
\label{sec:examen-del-sediment}

Es recullen 10-15 mL d'orina de primera hora del matí i es
centrifuguen 5 min a 1500-2000 rpm. S'observa el sediment al
microscopi a 40x (10 camps).

La cistinúria és la presència de cristalls d'aminoàcids a la orina.