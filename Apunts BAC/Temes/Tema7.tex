%------------------------------------------------------------------------------
% Tema 7. Enzimologia clínica
%------------------------------------------------------------------------------
\section{Enzimologia clínica}

\subsection{Enzims}
L'enzimologia clínica és un conjunt de tècniques destinades a detectar la presència i a quantificar l'activitat d'enzims en mostres biològiques:
\begin{itemize}
\item Presència d'enzims que no es trobin normalement en concentracions significatives
\item Variacions en els nivells d'enzims que poden trobar-se normalment en mostres biològiques
\item Isoenzims (formes diferents d'un enzim)
\end{itemize}

Es poden utilitzar enzims com a reactius específics per quantificar la concentració de metabòlits.

Un enzim és un biocatalitzador proteic. Es localitzen en tots els teixits corporals. Es determinen mitjançant la reacció enzimàtica. Un holoenzim està format per un apoenzim (porció proteica) i coenzim (porció no proteica no sempre necessària). L'activitat enzimàtica es pot veure reduïda si no hi ha apoenzim o coenzim.

Els enzims tenen destrucció citoplasmàtica o mitocondrial. A la circulació es poden eliminar per catabolisme que alliberarà aminoàcids i grups prostètics per la síntesi de noves proteïnes i/o enzims.

Tots els enzims sèrics tenen un origen cel·lular. Apareixen al sèrum com a conseqüència d'una lesió cel·lular (en petites quantitats de la degradació cel·lular). Les activitats enzimàtiques del sèrum són útils pel diagnòstic de malalties particulars o anomalies fisiològiques.

Quan les cèl·lules estan en proliferació, augmenta la síntesi d'enzims i s'atura el seu catabolsime. En l'estat d'inactivació, hi ha un descens de la síntesi d'enzims i augmenta la seva degradació degut a la carència de cofactors. 

Els enzims difonen a la limfa, passen a la sang on s'inactiven (pèrdua de grups prostètics, canvis conformacionals) i es catabolitzen. Els enzims es destrueixen en fagòcits, melsa, endotelials, cèl·lules sanguínies.

L'ús de certs enzims per diagnosticar malalties és per circumstàncies històriques. No és normal que un enzim que funciona sigui substituït per un altre, a no ser que la utilitat diagnòstica sigui molt millor.

\subsubsection{Lesió cel·lular}
Hi ha enzims i metabòlits intracel·lulars i extracel·lulars en una situació normal. Si hi ha una lesió cel·lular, es detecten enzims i metabòlits intracel·lulars a la sang.

Els enzims intracel·lulars poden aparèixer al plasma degut als processos normals de recanvi cel·lular. L'increment dels enzims pot ser degut a la lesió cel·lular o a l'augment de la proliferació.

\subsubsection{Mesura de la concentració dels enzims}
\begin{itemize}
\item \textbf{Concentració de massa:} Quantitat de l'enzim.
\item \textbf{Concentració catalítica:} Activitat de l'enzim. És el més usat. Es pot expressar com a:
  \begin{itemize}
  \item Activitat enzimàtica: Quantitat d'enzim que transforma un micromol de substrat per minut a 25ºC. Es poden fer servir UI o katals.
  \item Activitat específica: Unitats de l'enzim per mil·ligram de proteïna
  \item Activitat molecular o molar: Número de molècules de substrat trasformades per minut per
una sola molècula d'enzim
  \end{itemize}
\item Concentració d'isoformes: Permet discriminar el teixit d'origen.
\end{itemize}

L'activitat enzimàtica pot variar degut a la temperatura, pH, concentració de substrat, força iònica... La prova s'ha d'adequar a les concentracions òptimes de l'enzim.

\subsubsection{Distribució cel·lular dels enzims}
% Taula
\begin{table}[H]
  \centering
  
  \caption{Localització subcel·lular d'enzims amb importància clínica}
  \label{tab:enzims}
\end{table}

La presència d'enzims mitocondrials en sèrum és indicaiu de dany greu.

