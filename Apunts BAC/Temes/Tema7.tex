%------------------------------------------------------------------------------
% Tema 7. Enzimologia clínica
%------------------------------------------------------------------------------
\section{Enzimologia clínica}

\subsection{Enzims}
L'enzimologia clínica és un conjunt de tècniques destinades a detectar
la presència i a quantificar l'activitat d'enzims en mostres biològiques:
\begin{itemize}
\item Presència d'enzims que no es trobin normalement en
  concentracions significatives
\item Variacions en els nivells d'enzims que poden trobar-se
  normalment en mostres biològiques
\item Isoenzims (formes diferents d'un enzim)
\end{itemize}

Es poden utilitzar enzims com a reactius específics per quantificar la
concentració de metabòlits.

Un enzim és un biocatalitzador proteic. Es localitzen en tots els
teixits corporals. Es determinen mitjançant la reacció enzimàtica. Un
holoenzim està format per un apoenzim (porció proteica) i coenzim
(porció no proteica no sempre necessària). L'activitat enzimàtica es
pot veure reduïda si no hi ha apoenzim o coenzim.

Els enzims tenen destrucció citoplasmàtica o mitocondrial. A la
circulació es poden eliminar per catabolisme que alliberarà aminoàcids
i grups prostètics per la síntesi de noves proteïnes i/o enzims.

Tots els enzims sèrics tenen un origen cel·lular. Apareixen al sèrum
com a conseqüència d'una lesió cel·lular (en petites quantitats de la
degradació cel·lular). Les activitats enzimàtiques del sèrum són útils
pel diagnòstic de malalties particulars o anomalies fisiològiques.

Quan les cèl·lules estan en proliferació, augmenta la síntesi d'enzims
i s'atura el seu catabolsime. En l'estat d'inactivació, hi ha un
descens de la síntesi d'enzims i augmenta la seva degradació degut a
la carència de cofactors. 

Els enzims difonen a la limfa, passen a la sang on s'inactiven (pèrdua
de grups prostètics, canvis conformacionals) i es catabolitzen. Els
enzims es destrueixen en fagòcits, melsa, endotelials, cèl·lules sanguínies.

L'ús de certs enzims per diagnosticar malalties és per circumstàncies
històriques. No és normal que un enzim que funciona sigui substituït
per un altre, a no ser que la utilitat diagnòstica sigui molt millor.

\subsubsection{Lesió cel·lular}
Hi ha enzims i metabòlits intracel·lulars i extracel·lulars en una
situació normal. Si hi ha una lesió cel·lular, es detecten enzims i
metabòlits intracel·lulars a la sang.

Els enzims intracel·lulars poden aparèixer al plasma degut als
processos normals de recanvi cel·lular. L'increment dels enzims pot
ser degut a la lesió cel·lular o a l'augment de la proliferació.

\subsubsection{Mesura de la concentració dels enzims}
\begin{itemize}
\item \textbf{Concentració de massa:} Quantitat de l'enzim.
\item \textbf{Concentració catalítica:} Activitat de l'enzim. És el
  més usat. Es pot expressar com a:
  \begin{itemize}
  \item Activitat enzimàtica: Quantitat d'enzim que transforma un
    micromol de substrat per minut a 25ºC. Es poden fer servir UI o katals.
  \item Activitat específica: Unitats de l'enzim per mil·ligram de proteïna
  \item Activitat molecular o molar: Número de molècules de substrat
    trasformades per minut per
una sola molècula d'enzim
  \end{itemize}
\item Concentració d'isoformes: Permet discriminar el teixit d'origen.
\end{itemize}

L'activitat enzimàtica pot variar degut a la temperatura, pH,
concentració de substrat, força iònica... La prova s'ha d'adequar a
les concentracions òptimes de l'enzim.

\subsubsection{Distribució cel·lular dels enzims}
% Taula
\begin{table}[H]
  \centering
  
  \caption{Localització subcel·lular d'enzims amb importància clínica}
  \label{tab:enzims}
\end{table}

La presència d'enzims mitocondrials en sèrum és indicatiu de dany greu.

La detecció d'una isoforma permet fer un diagnòstic més precís. Es
poden detectar per:
\begin{itemize}
\item Diferència de càrrega neta (cromatografia, electroforesi,
  isoelectroenfocament)
\item Acció selectiva de determinades substàncies (inhibició
  selectiva)
\item Tècniques immunològiques (immunoinhibició, enzimimmunoassaig)
\end{itemize}

Poden alliberar-se els enzims encara que no hi hagi necrosi hística
(augment de la permeabilitat de les membranes) als teixits. Exemple:
delirium tremens.

La fosfatasa alcalina és un marcador per malalties hepatobiliars. En
canvi, hi ha altres enzims més bons com la leucinaminopeptidasa o la
5'-nucleotidasa.

Un altre exemple és la GPT (hepatopatia) que no ha estat substituïda
per la ornitin-carbamoïl-transferasa o iditol-deshidrogenasa.

\subsubsection{Interès diagnòstic dels enzims}
Les anàlisis enzimàtiques representen fins un 20\% de les proves
bioquímiques. Els laboratoris de bioquímica clínica poden determinar
entre 12-15 enzims diferents.

Actualment s'han determinat més de 60 enzims en sèrum, dels quals:
\begin{itemize}
\item Alguns es determinen habitualment al laboratori
\item Alguns són reflex de diverses malalties, però no es determinen
  degut a la seva dificultat
\item Alguns són importants a nivell de recerca, i només es determinen
  en situacions clíniques especials
\end{itemize}

% posar taula

\subsection{Estratègies bioquímiques per l'estudi clínic del
  metabolisme}

Els enzims es poden estudiar de diferents maneres:
\begin{enumerate}
\item Determinació de la concentració de metabòlits en líquids i
  teixits biològics
\item Determinació d'activitats enzimàtiques en líquids i teixits
  biològics
\item Diferenciació de les formes isoenzimàtiques
\item Anàlisi de la resposta metabòlica en front proves diagnòstiques
  específiques
\end{enumerate}

\subsubsection{Cinètica de les reaccions enzimàtiques monosubstrat}
Si:
\begin{itemize}
\item L'espectrofotòmetre absorbeix el substrat, l'absorbància
  disminuirà en funció del temps.
\item L'espectrofotòmetre absorbeix el producte, l'absorbància
  augmentarà en funció del temps.
\end{itemize}

S'ha d'escollir la longitud d'ona on es distingeixi l'absorció del
substracte de la del producte.

\subsubsection{Procediments per a mesurar la velocitat de
  transformació}

Hi ha 2 tipus:
\begin{enumerate}
\item \textbf{Procediments discontinus:}
  \begin{enumerate}
  \item A un punt.- Es mesura l’absorbància de la mostra i de un blanc
    al cap d’un temps determinat.
    \begin{equation}
      \label{eq:1}
      v = \frac{A_t-A_{Blanc}}{t}
    \end{equation}
  \item A dos punts.- Es mesuren 2 absorbàncies (A1, A2) a dos temps
    (t1, t2).
    \begin{equation}
      \label{eq:2}
      v = \frac{A_2-A_1}{t_2-t_1}
    \end{equation}
  \item A tres o més punts.- Es mesura l’absorbància a diversos valors
    de temps (mesurar dos punts pot ser inexacte).
    \begin{equation}
      \label{eq:3}
      v = \frac{\Delta A}{\Delta t}
    \end{equation}
  \end{enumerate}

\item \textbf{Procediments continus:} Es mesura l’absorbància
  continuament durant un temps determinat.
  \begin{equation}
    \label{eq:4}
    v = \frac{d A}{d t} 
  \end{equation}
\end{enumerate}

\subsubsection{Aplicació de tècniques enzimàtiques combinades}
La concentració dels enzims és molt baixa (nmol) i per tant es mesura
la seva activitat (o bé s'aplica immunoanàlisi).

Hi ha 3 fases de l'activitat enzimàtica:
\begin{enumerate}
\item Periode de lantència. La velocitat va augmentant progressivament.
\item Estat estacionari. La velocitat és proporcional a la
  concentració de l’enzim. 
\item Fase final.- la velocitat disminueix progressivament és on es fa
  l'anàlisi)
\end{enumerate}

La malalt deshidrogenasa té concentracions molt baixes, per tant
s'aplica a reaccions acoblades. Es mira l'activitat de l'enzim
anterior o posterior.

% Exemple de diap 13.

\subsubsection{Procediment general d'anàlisi per espectrofotometria
  visible o UV}
L'espectroscòpia estudia els sistemes mitjançant la seva interacció
amb les radiacions electromagnètiques. Un sistema és un conjunt de
partícules materials (àtoms, molècules).
\begin{itemize}
\item Absorció
\item Emissió
\item Difracció
\end{itemize}

L'espectrometria d'absorció molecular UV-visible és molt utilitzada en
clínica. Estudia l'absorció de la radiació magnètica ultravioleta i
visible. Permet mesurar la concentració d'una substància.

L'espectre electromagnètic és el conjunt de totes les longituds d'ona.

En un laboratori clínic s'utilitza amb més freqüència les regions
visible i ultraviolada.

Els espectrofotòmetres poden ser de:
\begin{itemize}
\item Feix simple: Primer es posa el blanc com a referència i després
  la mostra.
\item Feix doble: Detecta el blanc i la mostra alhora.
\end{itemize}

\subsection{Enzims marcadors habituals en enzimologia clínica}
