%------------------------------------------------------------------------------
% Tema 17. Control hormonal: neurohipòfisi i suprarrenals
%------------------------------------------------------------------------------
\section{Control hormonal: neurohipòfisi i suprarrenals}
\label{sec:hormones-i-ronyo}

\subsection{Neurohipòfisi}
\label{sec:neurohipofisi}

L'hipotàlem sintetitza la vasopressina i la oxitocina. Circulen al
sistema nerviós amb proteïnes de transport (neurofisina II i I) i es
neurosecreten a la hipòfisi. Circulen a la sang com a hormones
lliures.

La vasopressina es secreta en funció de la osmolalitat. Quan augmenta
la osmolalitat del plasma, s'activa l'hipotàlem anterior i activa la
sensació de set. Es secreta també ADH, anirà al túbul col·lector i
allà afavoreix la reabsorció d'aigua sense soluts a través
d'aquaporines. Té un efecte antidiürètic. La orina serà
hiperosmòtica (1200-1400 mosm/kg).

Quan falta aigua, disminueix el volum sanguini (p.e per una pèrdua de
sang), disminueix la pressió arterial i s'activen els barorreceptors
de l'aurícula i els cossos carotidis. Activen l'hipotàlem per produir
l'ADH. La morfina, els opiacis, la por, l'ansietat, el part també poden
activar la secreció de l'ADH.

\subsubsection{Hipofunció neurohiposifiària}
\label{sec:hipof-neur}
Pot ser deguda a un dany de les cèl·lules, a una obstrucció de la
secreció... Es produeix un dèficit d'ADH i la orina no es pot
concentrar i es dóna poliúria, el que comporta una \textbf{diabetis
  insípida central}.

Quan s'alteren els osmorreceptors i barorreceptors, es produeix un
dèficit d'ADH ja que no es secreta tot i que l'hipotàlem està bé. Això
produeix una \textbf{diabetis insípida clínica}.

Quan estan alterats els receptors d'ADH a la nefrona, l'ADH no pot
actuar i es produeix \textbf{diabetis insípida nefrogènica}.

La diabetis insípida neurogènica comprèn la central i la clínica.

Per diagnosticar, es fa una restricció hídrica absoluta. Quan el
pacient no beu, hauria d'augmentar l'ADH i la osmolalitat de la
orina. S'agafen mostres de sang i orina periòdiques. Després s'injecta
ADH i es tornen a fer les determinacions. La interpretació és la
següent:
\begin{itemize}
\item 
\end{itemize}

\subsubsection{Hiperfunció neurohiposifiària}
\label{sec:hiperf}
També s'anomena síndrome de secreció inapropiada d'ADH (SSIADH). Pot
ser degut a:
\begin{itemize}
\item Alteracions del SNC: traumatisme cranial, trombosi
  cerebrovascular, meningoencefalitis, hemorràgia cerebral.
\item Estrès
\item Fàrmacs: barbitúrics, narcòtics, psicotròpics, nicotina, carbamacepina.
\item Tuberculosi, pneumònia
\end{itemize}

La SSIADH va acompanyada d'hiponatrèmia.

Quan es mesuren hormones en plasma, cal tenir en compte la situació
fisiològica del pacient per evitar diagnòstics erronis. L'augment o
disminució d'ADH, p.e, depèn de si el pacient ha begut poc o molta
aigua abans de l'analítica.

\subsection{Glàndules suprarrenals}
\label{sec:gland-supr}

Està regulada per l'eix hipotàlem-adenohipòfisi. 

%% Completar