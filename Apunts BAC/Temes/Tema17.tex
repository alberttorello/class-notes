%------------------------------------------------------------------------------
% Tema 17. Control hormonal: neurohipòfisi i suprarrenals
%------------------------------------------------------------------------------
\section{Control hormonal: neurohipòfisi i suprarrenals}
\label{sec:hormones-i-ronyo}

\subsection{Neurohipòfisi}
\label{sec:neurohipofisi}

L'hipotàlem sintetitza la vasopressina i la oxitocina. Circulen al
sistema nerviós amb proteïnes de transport (neurofisina II i I) i es
neurosecreten a la hipòfisi. Circulen a la sang com a hormones
lliures.

La vasopressina es secreta en funció de la osmolalitat. Quan augmenta
la osmolalitat del plasma, s'activa l'hipotàlem anterior i activa la
sensació de set. Es secreta també ADH, anirà al túbul col·lector i
allà afavoreix la reabsorció d'aigua sense soluts a través
d'aquaporines. Té un efecte antidiürètic. La orina serà
hiperosmòtica (1200-1400 mosm/kg).

Quan falta aigua, disminueix el volum sanguini (p.e per una pèrdua de
sang), disminueix la pressió arterial i s'activen els barorreceptors
de l'aurícula i els cossos carotidis. Activen l'hipotàlem per produir
l'ADH. La morfina, els opiacis, la por, l'ansietat, el part també poden
activar la secreció de l'ADH.

\subsubsection{Hipofunció neurohiposifiària}
\label{sec:hipof-neur}
Pot ser deguda a un dany de les cèl·lules, a una obstrucció de la
secreció... Es produeix un dèficit d'ADH i la orina no es pot
concentrar i es dóna poliúria, el que comporta una \textbf{diabetis
  insípida central}.

Quan s'alteren els osmorreceptors i barorreceptors, es produeix un
dèficit d'ADH ja que no es secreta tot i que l'hipotàlem està bé. Això
produeix una \textbf{diabetis insípida clínica}.

Quan estan alterats els receptors d'ADH a la nefrona, l'ADH no pot
actuar i es produeix \textbf{diabetis insípida nefrogènica}.

La diabetis insípida neurogènica comprèn la central i la clínica.

Per diagnosticar, es fa una restricció hídrica absoluta. Quan el
pacient no beu, hauria d'augmentar l'ADH i la osmolalitat de la
orina. S'agafen mostres de sang i orina periòdiques. Després s'injecta
ADH i es tornen a fer les determinacions. La interpretació és la
següent:
\begin{itemize}
\item 
\end{itemize}

\subsubsection{Hiperfunció neurohiposifiària}
\label{sec:hiperf}
També s'anomena síndrome de secreció inapropiada d'ADH (SSIADH). Pot
ser degut a:
\begin{itemize}
\item Alteracions del SNC: traumatisme cranial, trombosi
  cerebrovascular, meningoencefalitis, hemorràgia cerebral.
\item Estrès
\item Fàrmacs: barbitúrics, narcòtics, psicotròpics, nicotina, carbamacepina.
\item Tuberculosi, pneumònia
\end{itemize}

La SSIADH va acompanyada d'hiponatrèmia.

Quan es mesuren hormones en plasma, cal tenir en compte la situació
fisiològica del pacient per evitar diagnòstics erronis. L'augment o
disminució d'ADH, p.e, depèn de si el pacient ha begut poc o molta
aigua abans de l'analítica.

\subsection{Glàndules suprarrenals}
\label{sec:gland-supr}

Està regulada per l'eix hipotàlem-adenohipòfisi. 

%% Completar

\subsubsection{Mineralocorticoides}
\label{sec:mineralocorticoides}

La més important és l'aldosterona. La desoxicorticosterona i la
corticosterona tenen una activitat similar però no són tant
importants.

Els mineralocorticoides es sintetitzen a partir de colesterol. ACTH
actua a l'escorça adrenal activant la síntesi de pregnenolona a partir
de colesterol.

Quan disminueix la pressió de filtració renal, els baroreceptors de
l'arteriola aferent glomerular activen la secreció de renina, que
actua sobre angiotensinogen plasmàtic per formar angiotensina
I. L'angiotensina I es converteix a angiotensina II al pulmó. Als
eritròcits, l'angiotensinasa destrueix l'angiotensina II.

L'angiotensina II actua als vasos sanguinis augmentant la
pressió. També actua a l'hipotàlem, on indueix la secreció d'ADH i la
ingesta d'aigua per augmentar el volum plasmàtic.

A l'escorça suprarenal s'allibera aldosterona. L'aldosterona actua al
túbul distal on hi ha la bomba de Na/K/H, on es recupera Na i aigua.

Quan hi ha un excés de K, l'aldosterona ajusta el K del plasma
afavorint l'absorció del Na i la secreció de K.

ACTH al còrtex suprerenal activa la síntesi d'aldosterona per captar
Na. És un mecanisme poc important fisiològicament (produir aldosterona
per acció d'ACTH). S'activa quan hi ha molt poc Na. Normalment, la via
de la renina és més important.

Quan augmenta el LEC, augmenta la pressió auricular i s'allibera
ANF. Es perd Na i aigua i s'inhibeix l'aldosterona i la renina.

Es pot mesurar la renina i l'aldosterona. S'incuba el plasma entre
1.5-18h a 37ºC amb IECA i inhibidors de l'angiotensinasa.

L'aldosterona s'extreu amb dissolvents o per cromatografia i es
determina per immunoanàlisi.

Acidosi per dilució: dilució del volum plasmàtic i augmenta el volum
plasmàtic. S'inhibeix l'aldosterona i s'activa el PNA. Es retenen
protons.

Contracció del volum plasmàtic: S'activa l'aldosterona, es reabsorbeix
aigua i Na però es perden protons.

\paragraph{Hiperaldosteronisme}
Es classifiquen en:
\begin{itemize}
\item Primari: Independent del sistema
  renina-angiotensina-aldosterona.
  \begin{itemize}
  \item  Adenoma productor d'aldosterona
  \item Hiperplàsia nodular zona glomerulosa
  \item Carcinoma productor d'aldosterona
  \end{itemize}

\item Secundari: Hipersecreció per estímuls extrasuprarenals
  \begin{itemize}
  \item Fisiològic: Es per líquid sanguini, que es compensa amb LIC
    però es produeix hiponatrèmia. S'activa l'aldosterona per
    reabsorbir aigua i sodi.
    \begin{itemize}
    \item Postura erecta
    \item Sudoració
    \item Ingesta de molt K i poc K
    \item Embaràs
    \end{itemize}
  
\item Patològic
    \begin{itemize}
    \item Hipertensió vasculorenal
    \item Insuficiència cardíaca congestiva
    \item Síndrome nefròtic
    \item Hipoproteïnèmia
    \item Tumor productor de renina 
    \item Anticonceptius orals
    \end{itemize}
  \end{itemize}
\end{itemize}

S'observa hipertensió arterial, hipopotassèmia i alcalosi
metabòlica. Hi haurà augment de bicarbonat (els protons excretats per
la bomba reaccionen amb CO2 i es produeix bicarbonat).

El diagnòstic es fa amb una dieta de 5 dies rica en Na. Es mesura
l'activitat de la renina plasmàtica. La renina hauria d'estar baixa.
\begin{itemize}
\item 
\end{itemize}

\paragraph{Hipoaldosteronisme}
Pot ser:
\begin{itemize}
\item Aïllat: Dèficit enzimàtic de la via de síntesi d'aldosterona o
  hiposecreció de renina.

\item Insuficiència suprarenal crònica
  \begin{itemize}
  \item Primària: Malaltia d'Addison (hipofunció suprarenal). Tenen
    l'aldosterona, els glucocorticoides i els andrògens suprarenals
    baixos.

  \item Secundària: Baixa producció a l'hipòfisi d'ACTH, nivells
    baixos de glucocorticoides i andrògens suprarrenals.
  \end{itemize}
\end{itemize}

\subsubsection{Glucocorticoides}
\label{sec:glucocorticoides}
Es sintetitza a partir del colesterol. ACTH activa la síntesi de
colesterol i pregnenolona.

Presenten ritmes circadians. El cortisol és el més important. Circula
fixat a alpha-globulina, albúmina i lliure. Té una vida mitjana de
80-90 minuts.

S'uneix a receptors citoplasmàtics, que transloquen al nucli on actuen
com a TF.

% Accions del cortisol

És una hormona anabòlica i catabòlica. Actua en situació d'alarma i
després normalitza el metabolisme.

Es determina amb anticossos marcats: RIA o ELISA. Té una poca
especificitat ja que hi ha reaccions creuades amb altres esteroides. 

Els nivells de cortisol varien segons:
\begin{itemize}
\item Ritme circadià: El màxim s'assoleix entre les 6h últimes de son
  i fins a 4h després. El mínim s'assoleix 4h abans d'anar a dormir.

\item Ritme ultradià: La secreció es fa en forma de polsos, entre 5-10 diaris.
\end{itemize}

Per seguir els nivells de cortisol, s'hospitalitza el pacient ii
s'extreu sang a diferents moments. S'analitza la corba de secreció
segons els ritmes circadians. Si desapareix el ritme circadià, vol dir
que l'individu té síndrome de Cushing.

Els valors elevats suposen:
\begin{itemize}
\item Síndrome de Cushing
\item Malalties agudes
\item Estrès
\item Depressions
\item Alcoholisme
\item Diürètics
\item Ingesta de Na: Es secreta cortisol per augmentar la taxa de
  filtració glomerular.
\end{itemize}

L'ACTH és un pèptid de 4500 Da.

Si s'injecta ACTH, el cortisol puja fins a 550. Si el cortisol es
queda a 300 nM, el pacient té malaltia d'Addison.

Una injecció intravenosa de CRH: si provoca un augment de CTH i
cortisol hi ha síndrome de Cushing primari però si es queden igual i
ha síndrome de Cushing primari o ectòpic. Els augments són relatius al
moment abans de l'estimulació amb CRH.

L'administració d'insulina provocaria una pujada de cortisol. Si el
cortisol no augmenta, hi ha síndrome de Cushing.

La dexametasona és un glucocorticoide sintètic amb alta afinitat pels
receptors i una vida mitjana llarga. No interfereix en les mesures de
cortisol. S'administra DXM a la nit i al matí el cortisol baixa perquè
la DXM retroinhibeix la secreció de cortisol. Si el cortisol està alt,
suposa que l'individu té síndrome de Cushing.

\paragraph{Síndrome de Cushing}
Hiperfunció de les glàndules suprarenals amb hiperproducció de
glucocorticoides.
\begin{itemize}
\item Primari
  \begin{itemize}
  \item Adenoma
  \item Carcinoma
  \item Hiperplàsia nodular
  \end{itemize}
\item Secundari: Adenoma hipofisiari secretor d'ACTH: malaltia de
  Cushing.
  
\item Ectòpica: Neoplàsies secretores d'CTH i CRH.
\end{itemize}

El diagnòstic es basa en un augment de cortisol al plasma i a la
orina. Es perd el ritme circadià de cortisol. El cortisol tampoc
s'inhibeix per DXM. No augmenta el cortisol en resposta a insulina.

Si disminueix ACTH és el síndrome primari. Si ACTH es manté o augmenta
hi ha síndrome secundari o ectòpic.

% Galtes, parpelles i mans inflats. Hiponatrèmia, hipoosmolalitat. El
% pacient té SSIADH

\subsubsection{Catecolamines}
\label{sec:catecolamines}

