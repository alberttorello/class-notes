%------------------------------------------------------------------------------
% Tema 16. Control dels electròlits i l'aigua
%------------------------------------------------------------------------------

\section{Control dels electròlits i l'aigua}
\label{sec:contr-dels-electr}

%% Recordatori aigua i compartiments corporals

Els ions principals són:
\begin{itemize}
\item LEC: Na >> K i Cl >> bicarbonat
\item K i fosfit
\end{itemize}

Les membranes són permeables a l'aigua però no als ions. L'aigua difon
per compensar la pressió osmòtica.

La bomba de Na/K és un sistema de transport actiu que treu 3 Na per
entrar 2 K i així mantenir la diferència de potencial de la membrana
cel·lular.


La osmolalitat de la cèl·lula també està condicionada per la
concentració de proteïnes intracel·lulars.

\subsection{Regulació de l'osmolalitat}
\label{sec:regul-de-losm}

\subsubsection{Hipotàlem}
\label{sec:hipotalem}
Si falta líquid, l'hipotàlem ens genera la sensació de set. 

Controla la pèrdua d'aigua mitjançant la síntesi de vasopressina o
ADH. L'ADH viatja fins a la hipòfisi on surt a la sang. Al ronyó,
interaccionen amb el túbul col·lector per captar aigua.

Hi ha detectors d'osmolalitat, els osmoreceptors que activen
l'hipotàlem. 

Angiotensina II circulant estimula directament l'hipotàlem.

\subsubsection{Renina-angiotensina-aldosterona}
\label{sec:renina-angi-aldost}
Si falta aigua, baixa la pressió i això ho detecta el glomèrul; també
hi pot haver una estimulació simpàtica renal o que hi hagi poc Na al
túbul distal (al túbul proximal els transportadors no es saturen i per
això baixa tant el Na).

Per tot això, el ronyó allibera renina. Hi ha angiotensinogen
circulant al plasma sintetitzat pel fetge, i es fabrica angiotensina
II. L'angiotensina II té receptors als vasos, al ronyó i a
l'hipotàlem. Al còrtex adrenal provoca l'alliberament d'aldosterona i
afavoreix la captura de sodi i aigua. A l'hipotàlem actua al centre de
la set i secretant ADH.

\subsubsection{Pèptid natriurètic atrial}
\label{sec:pept-natr-atri}

Augmenta volum LEC, puja la pressió però disminueix la osmolalitat. El
PNA augmenta la velocitat de filtració glomerular i l'excreció renal
d'aigua i Na.

Les proteïnes del plasma compensen la pressió hidrostàtica generada
pel cor.

\subsection{Alteracions hidroelectrolítiques}
\label{sec:alter-hidr}

\subsubsection{Aigua}
\label{sec:aigua}

La retenció d'aigua és visible per l'aparició d'edemes.

La pèrdua d'aigua s'intueix per senyals de deshidratació.

En un cas d'hiponatrèmia o hipernatrèmia s'ha d'invetsigar si és degut
a l'aigua o al sodi.

\subsubsection{Sodi}
\label{sec:sodi}
Es determina fent servir elèctrodes selectius o espectrofotometria de
flama.

\paragraph{Hiponatrèmia}
Es determina també la osmolalitat de la mostra, i segons això:
\begin{enumerate}
\item Normal (isotònica)
  \begin{itemize}
  \item Hiperproteïnèmia o hiperlipidèmia: És una
    pseudohiponatrèmia. Les macromolècules no alteren gaire la pressió
    osmòtica però sí que alteren el volum extracel·lular.

  \item Perfusió isotònica sense Na:
  \end{itemize}

\item Alta (hipertònica): Si hi ha un excés de glucosa, provoca que
  augmenti la pressió osmòtica. Augmenta el LEC gràcies a l'aportació
  de LIC; i el LEC es dilueix parcialment.

\item Baixa (Isotònica): És la més freqüent i es pot classificar
  segons el volum extracel·lular:
  \begin{itemize}
  \item Hipervolèmia: Presència d'edemes. L'inidividu reté molta aigua
    i Na (menys que l'aigua que es reté). Causes:
    \begin{itemize}
    \item Insuficiència cardíaca congestiva
    \item Insuficiència renal crònica (nefrosi)
    \item Dany hepàtic amb hipoproteïnèmia
    \end{itemize}

  \item Isovolèmia: Normalment s'acompanya per pèrdues renals de
    Na. Pot ser deguda a:
    \begin{itemize}
    \item Alteració ADH
    \item Fallada en la resposta d'ADH
    \item Tractament amb diürètics (compensat amb aigua del LIC)
    \end{itemize}

  \item Hipovolèmia: Pèrdues d'aigua i Na. Deshidratació amb una
    disminució de Na total. Degut a:
    \begin{itemize}
    \item Pèrdues renals de Na
      \begin{itemize}
      \item Nefropatia (nefritis, acidosi tubular, necrosi tubular)
      \item Tractament amb diürètics, insulinam adrenalina
      \end{itemize}
    \item Pèrdues extarrenals de Na. Normalment en sèrum augmenta la
      urea, la proteïna i l'hematòcrit.
      \begin{itemize}
      \item Tracte gastrointestinal (vòmit, diarrea)
      \item Pell (cremades, suor)
      \end{itemize}
    \end{itemize}
  \end{itemize}
\end{enumerate}

\paragraph{Hipernatrèmia}
Augmenta el Na (concentració), i per tant la osmolalitat del plasma. Es classifica
segons el volum extracel·lular:
\begin{itemize}
\item \textbf{Hipovolèmica:} Pèrdua de líquids hipotònics (deshidratació). Es
  produeix hipernatrèmia, hipovolèmia, deshidratació i pèrdua de sodi
  total.

  \begin{itemize}
  \item Renal: La osmolalitat de la orina és inferior a la del
    plasma. Pot estar causada per:
    \begin{itemize}
    \item Diüresi osmòtica per hiperglucèmia
    \item Insuficiència renal amb poliúria: Dèficit d'ADH
    \item Insuficiència adrenal amb poliúria. Dèficit d'aldosterona,
      que actua sobre la bomba Na/K/H.
    \end{itemize}

  \item Extrarrenal: La osmolalitat de la orina és superior a la del
    plasma. Degut a pèrdues d'aigua per la pell i el tracte
    gastrointestinal. Es compensa amb l'excreció d'un menor volum
    d'orina hipertònica.
  \end{itemize}

\item \textbf{Euvolèmica:} Pèrdua moderada pura d'aigua. En part
  compensada pel LIC.
  \begin{itemize}
  \item Renal: La osmolalitat de la orina és inferior a la del
    plasma. Excreció f'un volum major d'orina hipotònica. Degut a la
    diabetis insípida. Pot ser:
    \begin{itemize}
    \item Neurogènica: Secreció disminuïda d'ADH.
    \item Nefrogènica: Resposta renal a l'ADH disminuïda.
    \end{itemize}
  \item Extrarrenal: La osmolalitat de la orina és més gran que la del
    plasma. Degut a la pèrdua inadequada d'aigua per via
    gastrointestinal, cutània o respiratòria.
  \end{itemize}

\item \textbf{Hipervolèmica:} Guany d'aigua i Na. Causat per
  hiperaldosteronisme crònic: augmenta la retenció de Na i per tant
  també d'aigua. Pot ser deguda a la perfusió amb excés de Na.
\end{itemize}

\subsubsection{Potassi}
\label{sec:potassi}
El K en plasma suposa un 2\% del potassi total. És més abundant a
l'interior de les cèl·lules. El potassi es filtra però es reabsorbeix
pràcticament tot al tubul proximal. Al tubul distal es secreta K quan
es reabsorbeix Na.

Si LEC és àcid, disminueix l'entrada de K al LIC.
Si LEC és bàsic, augmenta el K al LIC.

Les alteracions es classifiquen segons el pH de la sang.

\paragraph{Hipopotassèmia}

\begin{itemize}
\item \textbf{Bicarbonat baix:}
  \begin{itemize}
  \item Pèrdua renal: L'acidosi tubular és l'acidificació de les
    cèl·lules del túbul distal. La bomba de Na/K/H no treu protons i
    passen a la sang, llavors es secreta K a la orina.

L'alcalosi respiratòria és quan s'elimina CO2 per la respiració degut
a una hiperventilació. El ronyó secreta K per compensar.

  \item Pèrdua extrarrenal: Diarrea aguda o fístula pancreàtica.
  \end{itemize}

\item \textbf{Bicarbonat alt o normal:}
  \begin{itemize}
  \item Pèrdua renal:
    \begin{itemize}
    \item Alcalosi metabòlica: Excés de bicarbonat. Es recuperen
      protons al ronyó però es secreta K.
    \item Excés d'aldosterona.
    \end{itemize}

  \item Pèrdua extrarrenal: Diarrea crònica, abús de laxants, adenoma
    de colon...
  \end{itemize}
\end{itemize}

\paragraph{Hiperpotassèmia}

\begin{itemize}
\item \textbf{Bicarbonat normal:} Degut a un dèficit d'aldosterona. La
  distribució del K entre els compartiments pot ser errònia degut a un
  dèficit d'insulina, necrosi tissular, cremades, leucocitosi.

\item \textbf{Bicarbonat alt:} Acidosi respiratòria crònica. El ronyó
  treu protons per Na, i augmenta el K al plasma.

\item \textbf{Bicarbonat baix:} Acidosi metabòlica. Augmenta el GAP
  aniònic, es perden protons i es reté K.
\end{itemize}

\subsubsection{Clor}
\label{sec:clor}

Alteracions al LEC paral·leles al Na.

\paragraph{Hipoclorèmia}
Dèficit de Cl no associat a Na. Quan hi ha excés de secrecions
gàstriques (vòmits). Es produeix alcalosi metabòlica, i es llença
potassi, pel que va associada a hipopotassèmia.

La clororrea és una malaltia en que no funcionen els transportadors de
Cl de l'intestí.

\paragraph{Hiperclorèmia}
Produïda per:
\begin{itemize}
\item Acidosi metabòlica: Degut a insuficiència renal crònica, acidosi
  tubular o inhibidors de l'anhidrasa carbònica. Si es perden
  bicarbonats, es reabsorbeix Cl per compensar la pèrdua de càrregues
  negatives.

\item Alcalosi respiratòria: Es compensa capturant protons al ronyó i
  secretant K. Es perden bicarbonats.
\end{itemize}