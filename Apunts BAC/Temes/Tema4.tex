%------------------------------------------------------------------------------
% Tema 4. Prote�nes plasm�tiques
%------------------------------------------------------------------------------
\section{Prote�nes plasm�tiques}

Hi ha 4 grans grups de prote�nes plasm�tiques:
\begin{itemize}
\item Transport: Sobretot de mol�cules hidrof�biques. La majorit�ria �s l'alb�mina.
\item Reactants de fase aguda
\item Complement
\item Immunoglobulines
\end{itemize}

\subsection{Prote�nes plasm�tiques de transport}
Aquests sistemes estan en equilibri:
\begin{equation}
  \label{eq:1}
  P + L <=> P-L
\end{equation}

Per ser efectiu, el lligand ha d'anar lliure en plasma (sense uni� a prote�nes). Quan augmenta la concentraci� de prote�na transportadora, augmenta la concentraci� de complex prote�na-lligand i disminueix la quantitat de lligand lliure a la sang.

La tiroxina es transporta unida a alb�mina o a una prote�na espec�fica com la TBP.

Les prote�nes plasm�tiques s'analitzen segons el patr� d'electroforesi. Les prote�nes que tenen un pI proper a 7 tenen poca mobilitat. Si tenen un pI m�s baix (6) ja corren m�s al gel. L'alb�mina t� un pI molt baix i corre molt, la prealb�mina encara t� un pI m�s baix. La manera de distingir les prote�nes �s mitjan�ant la revelaci� amb antic�s. 

Per separar prote�nes segons el pI es fa electroforesi en un suport d'acetat de cel�lulosa i per separar per pes molecular es fa SDS-PAGE.

