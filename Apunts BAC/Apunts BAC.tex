%====================================================================
%Encap�alament
%====================================================================
\documentclass[a4paper,oneside,12pt]{article}
\usepackage[left=2.5cm,right=2cm,top=2cm,bottom=2cm]{geometry}
\usepackage[catalan]{babel}
\usepackage[T1]{fontenc}
\usepackage[latin9]{inputenc}
\usepackage{graphicx} % Required for including pictures
\usepackage{subfigure} % subfiguras
\usepackage{float} % Allows putting an [H] in \begin{figure} to specify the exact location of the figure
\usepackage{wrapfig} % Allows in-line images
\linespread{1.2}
\graphicspath{{Pictures/}}

\usepackage{subscript} %super�ndexs i sub�ndexs
\setcounter{secnumdepth}{5} %Numerar fins a subpar�grafs
\usepackage{pdfpages} %posar arxius pdf (com articles...)

\usepackage{amssymb, amsmath, amsbsy} % librerias ams
\usepackage{stackrel} %Escriure a sobre o sota de qualsevol fletxa
\usepackage{color}
\usepackage{multicol}
\usepackage{array}
\usepackage[full]{textcomp}
\usepackage{multirow}

% Enumeracions
\usepackage[shortlabels]{enumitem}
\usepackage{pstricks}

\usepackage{tikz}
\newcommand*{\itembolasazules}[1]{% bolas 3D 
	\footnotesize\protect\tikz[baseline=-3pt]% 
	\protect\node[scale=.7, circle, shade, ball
	color=blue]{\color{white}\Large\bf#1};}

\newcommand*{\itembolasverdes}[1]{% bolas 3D 
	\footnotesize\protect\tikz[baseline=-3pt]% 
	\protect\node[scale=.7, circle, shade, ball
	color=green]{\color{white}\Large\bf#1};}

\newcommand*{\itembolasrojas}[1]{% bolas 3D 
	\footnotesize\protect\tikz[baseline=-3pt]% 
	\protect\node[scale=.7, circle, shade, ball
	color=red]{\color{white}\Large\bf#1};}

% Lletres gregues
\DeclareRobustCommand{\greektext}{%
  \fontencoding{LGR}\selectfont\def\encodingdefault{LGR}}
\DeclareRobustCommand{\textgreek}[1]{\leavevmode{\greektext #1}}
\DeclareFontEncoding{LGR}{}{}
\DeclareTextSymbol{\~}{LGR}{126}

%Bibliografia
\usepackage[square]{natbib}

% Instruccions especials
\newcommand{\seny}{senyalitzaci�}
\newcommand{\aas}{amino�cids}
\newcommand{\TF}{factor de transcripci�}
\newcommand{\TFs}{factors de transcripci�}
\newcommand{\tgfb}{TGF\textgreek{b}}
\newcommand{\pparg}{PPAR\textgreek{g}}
\newcommand{\betaadren}{\textgreek{b}-adren�rgic}
\newcommand{\Rbeta}{R\textsubscript{\textgreek{b}}}
\newcommand{\iso}{isoproterenol}
\newcommand{\aden}{adenilat ciclasa}
\newcommand{\ca}{Ca\textsuperscript{2+}}
\newcommand{\betacat}{\textgreek{b}-catenina}
\newcommand{\fletsup}{$\uparrow$}
\newcommand{\fletinf}{$\downarrow$}
\newcommand{\gng}{gluconeog�nesi}
\newcommand{\tnfa}{TNF-\textgreek{a}}
\newcommand{\ikb}{I\textgreek{k}B\textgreek{a}}
\newcommand{\nfkb}{NF-\textgreek{k}B}
\newcommand{\pgc}{PGC-1\textgreek{a}}
\newcommand{\na}{Na\textsuperscript{+}}
\newcommand{\pot}{K\textsuperscript{+}}
\newcommand{\ags}{�cids grassos}
\newcommand{\ck}{cicle de Krebs}
\newcommand{\ag}{�cid gras}
\newcommand{\ppara}{PPAR\textgreek{a}}
\newcommand{\ppard}{PPAR\textgreek{d}}
\newcommand{\ifng}{IFN-\textgreek{g}}

% F�rmules qu�miques
\usepackage{chemformula}
%\usepackage[version=4,arrows=pgf]{mhchem}
\usepackage{texshade}
\usepackage{textopo}

%\newcommand{\powd}[1]{$10^{#1}$}
%\newenvironment{?}{?}{?}

\setcounter{tocdepth}{2} % numerar fins subseccions a TOC

% Encap�alament i peu de p�gina
\usepackage{fancyhdr}
\pagestyle{fancy}
%\renewcommand{\sectionmark}[1]{\markright{\textsc{\thesection. #1}}}
\lhead{}
\rhead{}
\lfoot{\sc Bioqu�mica Anal�tica i Cl�nica}
%\rfoot{\thepage}
\cfoot{}

\fancyfoot[R]{\thepage}

\renewcommand{\headrulewidth}{0.4pt}
\renewcommand{\footrulewidth}{0.4pt}

\raggedbottom
\providecommand{\tabularnewline}{\\}


% Format seccions
%\usepackage{sectsty}
%\sectionfont{\centering\nohang\LARGE\bfseries}
%\partfont{\centering\huge\red}
\usepackage{titlesec}

\titleformat{\section}[block]{\centering}{\bfseries\LARGE\thesection  . }{0em}{\LARGE\bfseries}{}

\titleformat{\subsection}[hang]{\flushleft}{\bfseries\Large\thesubsection }{0.2cm}{\Large\bfseries}

\titleformat{\subsubsection}[hang]{\flushleft}{\bfseries\large\thesubsubsection }{0.2cm}{\large\bfseries}

% Format parts
%\titleformat{\part}[hang]{\centering\vspace{-1.75cm}}{\bfseries\huge\thepart . }{0em}{\huge \bfseries}{}
\newcommand{\titline}{\titlerule[2pt]}
\titleformat{\part}[hang]
  {\vspace{-2cm}\sc\bfseries\huge\red}{\centering\huge\sc\bfseries
  	}
  {0ex}
  {\titline \\
   \vspace{1pt}%
   \centering\huge\sc\bfseries\red\thepart . }
   [%
 \titline]

\usepackage[hyperindex,linktocpage]{hyperref}

% Format �ndex
\usepackage{kpfonts}
\usepackage{titletoc}
\contentsmargin{0cm}
\titlecontents{part}[0pc]
{\addvspace{30pt}%
	\\\color{red}\large\sc\bfseries}%
{}
{}
{\;\titlerule\;\large\bfseries \thecontentspage}%
\titlecontents{section}[2.4pc]
{\addvspace{1pt}}
{\bfseries\contentslabel[\thecontentslabel]{2.4pc}}
{}
{\hfill\small\bfseries\thecontentspage}
[]
\titlecontents*{subsection}[4pc]
{\addvspace{-1pt}\small}
{}
{}
{\ --- \small\thecontentspage}
[ \textbullet\ ][]
%%--


% ====================ENTORNS ====================
\usepackage{tikz,tkz-tab}
\usepackage{tcolorbox, empheq}%

% Entorn per dades
\definecolor{colordadesfons}{RGB}{255,255,255}
\definecolor{colordadesmarc}{RGB}{108, 217, 0}

%\definecolor{colorrecfons}{RGB}{203,216,227}
\definecolor{colorrecfons}{RGB}{255,255,255}
\definecolor{colorrecmarc}{RGB}{128,177,221}

\tcbuselibrary{theorems}
\newtcbtheorem[number within=section]{dades}{An�lisi de dades} {colback=colordadesfons,colframe=colordadesmarc,fonttitle=\bfseries,separator sign={\ $\blacktriangleright$}}{theo}

\newtcbtheorem[number within=section]{rec}{Recordatori} {before=\begin{center},after=\end{center},colback=colorrecfons,colframe=colorrecmarc,fonttitle=\bfseries,separator sign={\ $\blacktriangleright$}}{theo}

\newtcbtheorem[number within=section]{metodes}{M�todes} {colback=red!5!white,colframe=red!75!black,fonttitle=\bfseries,separator sign={\ $\blacktriangleright$}}{theo}

% Equacions
\tcbuselibrary{skins,breakable}
\newcounter{example}
\colorlet{colexam}{red!75!black}

\newtcolorbox[]{myexample}[2][]{%
	empty,title={#1 \thetcbcounter},attach boxed title to top left,
	boxed title style={empty,size=minimal,toprule=2pt,top=4pt,
		overlay={\draw[colexam,line width=2pt]
			([yshift=-1pt]frame.north west)--([yshift=-1pt]frame.north east);}},
	coltitle=colexam,fonttitle=\Large\bfseries,
	before=\par\medskip\noindent,parbox=false,boxsep=0pt,left=0pt,right=3mm,top=4pt,
	breakable,pad at break*=0mm,vfill before first,
	overlay unbroken={\draw[colexam,line width=1pt]
		([yshift=-1pt]title.north east)--([xshift=-0.5pt,yshift=-1pt]title.north-|frame.east)
		--([xshift=-0.5pt]frame.south east)--(frame.south west); },
	overlay first={\draw[colexam,line width=1pt]
		([yshift=-1pt]title.north east)--([xshift=-0.5pt,yshift=-1pt]title.north-|frame.east)
		--([xshift=-0.5pt]frame.south east); },
	overlay middle={\draw[colexam,line width=1pt] ([xshift=-0.5pt]frame.north east)
		--([xshift=-0.5pt]frame.south east); },
	overlay last={\draw[colexam,line width=1pt] ([xshift=-0.5pt]frame.north east)
		--([xshift=-0.5pt]frame.south east)--(frame.south west);},% 
}

\begin{document}

%----------------------------------------------------------
%Portada
%----------------------------------------------------------
\begin{titlepage}

\newcommand{\HRule}{\rule{\linewidth}{0.5mm}} % Defines a new command for the horizontal lines, change thickness here

\center % Center everything on the page
\HRule \\[0.4cm]
{\Huge\bfseries\textsc{Bioqu�mica Anal�tica i Cl�nica}} \\[0.4cm] % Title of your document
\HRule \\[04cm]
\includegraphics[width=1\textwidth]{Logo} %Include a department/university logo

\vspace{1cm}
\textsc{\Large Ci�ncies Biom�diques UB - Primavera 2017}\\[0.5cm] % Minor heading such as course title
\vspace{0.5cm}
\textsc{\large Albert Torell� P�rez}
\vfill % Fill the rest of the page with whitespace
\end{titlepage}

%===================================
\renewcommand{\figurename}{\textsc{{Figura}}}
\renewcommand{\tablename}{\textsc{{Taula}}}
\renewcommand{\thefootnote}{\alph{footnote}}

%------------------------------------------------------------------------------
%Cos dels apunts 
%------------------------------------------------------------------------------
\pagenumbering{Roman} % para comenzar la numeraci�n de paginas en n�meros romanos

%----------------------------
% Taula de continguts
%----------------------------
\tableofcontents

%--------------------------------------------------------------------------------------------------
%BLOC 1. CONCEPTES B�SICS DE REGULACI�
%--------------------------------------------------------------------------------------------------
\newpage
\pagenumbering{arabic}
\part{Introducci�}

%------------------------------------------------------------------------------
% Tema 1. Variabilitat Cl�nica
%------------------------------------------------------------------------------
%------------------------------------------------------------------------------
% Tema 1. Variabilitat Cl�nica
%------------------------------------------------------------------------------
\section{Variabilitat Cl�nica}
En les proves anal�tiques hi ha un rang de variabilitat, ja sigui per
assumptes purament t�cnics o estil de vida dels pacients (fumadors,
ingesta de medicaments...).

Els par�metres que s'analitzen s'anomenen magnituds bioqu�miques d'un
sistema. El s�rum i la orina s�n els principals fluids fisiol�gics que
s'analitzen.

\subsection{Mesura o determinaci�}
Conjunt d'operacions que permeten donar un valor a una magnitud
bioqu�mica concreta d'un sistema.

La mostra s'anomena esp�cimen, que �s la porci� de material original
especialment seleccionada, provinent d'un sistema din�mic, i que en el
moment d'obtenir-lo s'assumeix que es representatiu del material
orginal.

La variabilitat d'una determinaci� pot provenir de diferents fonts:
\begin{itemize}
\item Preanal�tica o premetrol�gica: Presa de la mostra, conservaci�,
  transport. S�n extra-anal�tiques. 
\item Biol�gica
\item Iatrog�nica: Relacionada amb el seguiment d'un tractament m�dic.
\item Anal�tica i extraanal�tica: �s la que dep�n del proc�s purament
  anal�tic, de la t�cnica i instrument.
\item Postanal�tica: Interpretaci� del resultat, confondre les mostres...
\end{itemize}

\subsubsection{Variabilitat premetrol�gica}
\begin{itemize}
\item Postura: En una posici� erecta varia el volum de sang, ja que el
  l�quid passa a l'espai intersticial. Augmentar� la concentraci�
  d'algunes magnituds.
  \begin{itemize}
  \item Augmenta m�s un 5\% la concentraci� d'alb�mina, Fe,
    colesterol, fosfatasa �cida i fosfatasa alcalina.
  \end{itemize}
Els ions passen m�s f�cilment per l'espai intersticial.

\item Si el temps de torniquet �s molt gran es genera hip�xia (amb
  l'acidosi corresponent). Pot comportar dilataci� venosa, i augmenta
  l'alb�mina, colesterol, AST.

\item Hem�lisi: Alteraci� colorim�trica.

\item Temps fins a l'anal�tica.
\end{itemize}



%=================================================================
%=================================================================
\newpage
\phantomsection
\addcontentsline{toc}{part}{Refer�ncies}
\begin{multicols}{2}
\bibliography{bibbac} 
\bibliographystyle{authordate3}
\end{multicols}

\end{document}

Els de BIOMED fem un parcial. Val un 10\% extra.

El final �s 60 preguntes tipus test, 3 preguntes curtes de cada part +
10 preguntes de pr�ctiques.