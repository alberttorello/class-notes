
\section{Anomalies en ploïdies}
\label{sec:anom-en-ploid}

Els ratolins joves tenen hepatòcits 2n però a partir de 56 dies d'edat, comencen a aparèixer cèl·lules 4n.

Diferents maneres d'adquirir poliploïdia de manera natural:
\begin{itemize}
\item Fusió cel·lular
\item Endoreplicació: Només fase S, sense entrar en fase M.
\item Mitotic slippage: La mitosi no s'acaba de fer. Error en checkpoint de SAS-securina...
\item Error en la citocinesi: Manera preferencial en fetge d'augmentar la ploïdia.
\end{itemize}

Al fetge: primera mitosi donant cèl·lules 4n però binucleades. Llavors hi ha mitosi i es formen 2 cèl·lules 4n cadascuna amb 1 sol nucli per cèl·lula. Així successivament fins a 32n.

Perquè la poliploïdia té lloc després del naixement: Tenia lloc en el punt del deslletament. Això és perquè es passa d'una high fat diet a una dieta high carbohidrate. STZ destrueix específicament les cèl·lules $\beta$ pancreàtiques. Els ratolins tractats amb STZ tenien menys ploïdia i una citocinesi disminuïda.

Tractaven les cèl·lules amb inhibidors de PI3K, Akt i MEK. RhoA és una proteïna associada a microtúbuls i té un paper clau en citocinesi.

% Hi-C in poliploid cells??????

E2F són TF implicats en mitosi: 1 i 2 activadors i la resta inhibidors.

miR122 regulava poliploïdia. Ratolins KO presentaven menys hepatòcits binucleats.

Localització de cèl·lules poliploides: Resultats contradictoris

La ploïdia al fetge està molt controlada. Ploidy conveyor: mecanisme que reconeix la ploïdia al fetge i pot regular-la.

Aneuploïdies al fetge: 

Altres teixits quan adquireixen poliploïdies es produeixen tumors.

Per què serveix la poliploïdia al fetge: Augmenta la diversitat genètica del fetge i augmenta capacitat d'afrontar reptes metabòlics.