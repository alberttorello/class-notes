%------------------------------------------------------------------------------
% Tema 2. Adult Stem Cells
%------------------------------------------------------------------------------
\section{\textit{Adult Stem Cells}}

La potència de les ASC és més limitada, aquestes cèl·lules són multipotents. El
programa epigenètic és restrictiu i el cultiu és complicat.

El nínxol cel·lular es refereix a cèl·lules mare adultes. Primer es va
descriure al sistema nerviós. Es va veure que el nínxol és
extrapol·lable a tots els òrgans adults. El nínxol necessita cèl·lules
amb capacitat proliferativa, una matriu extracel·lular permissiva i angiogènesi a la zona:
\begin{itemize}
\item Aquestes cèl·lules presenten taxes de proliferació
  baixes. Moltes vegades estan quiescents. Elements que controlen
  entrada i sortida al cicle cel·lular. El cicle cel·lular pot durar fins a 3 dies.

\item La matriu extracel·lular permissiva permet que les cèl·lules
  migrin (cèl·lules de les criptes intestinals). La fibronectina és dels més importants.

\item Hi ha una molècula endotelial que controla la proliferació d'
  ASC. El vas sanguini aporta elements que forcen a la diferenciació
  de les ASC, a la sortida del cicle cel·lular...
\end{itemize}

Els ROS inhibeixen la capacitat proliferativa de la cèl·lula i indueixen la diferenciació.

Les ASC es troben en:
\begin{multicols}{2}
  \begin{itemize}
  \item Sang del cordó umbilical: Majoritàriament HSC. També hi ha MSC i endotelials.
  \item Placenta
  \item Medul·la òssia: Punció a cresta ilíaca (en persones joves) o
    en estèrnum (gent gran). Hi ha 2 poblacions: les HSC i MSC
    (mesenquimals de medul·la òssia).
  \item Lipoaspirat: S'obtenen MSC.
  \end{itemize}
\end{multicols}

Per aïllar ASC, es necessita molta quantitat de teixit. Trobem 1 ASC
per cada milió de cèl·lules.

\begin{enumerate}[\bf 1)]
\item Percentatge baix de la població cel·lular general: A la medul·la
  òssia i al cordó umbilical, però, hi ha moltes HSC.
\item Dificultat d'aïllament en el teixit adult: Actualment, els protocols són senzills.
\item Relativa dificultat de creixement \textit{in vitro}: Actualment, no és difícil.
\item Limitació amb relació al potencial i versatilitat: Una MSC no
  dura més de 3 setmanes i les neural stem cells uns 4 mesos abans de ser un glioma.
 \end{enumerate}

\subsection{Cèl·lules mare hematopoiètiques}
El marcador de les HSC és el CD34. La iferenciació d'aquestes
cèl·lules a línies eritroides i leucoides és senzilla però costa molt
obtenir plaquetes ja que els megacariòcits costen de mantenir en cultiu.

Les cèl·lules CD34 no funcionen per reparar la necrosi en un cor
infartat.

Un \textit{sorting} per CD34, c-kit i Sca-1 s'obté la població de
progenitors hematopoiètics. Són cèl·lules poc adherents a les plaques
de plàstic ja que són molt proliferatives.

Quan s'extreu la sang de cordó umbilical, les cèl·lules es cariotipen
i es guarden en un banc.

\subsection{Cèl·lules mare mesenquimals}
El CD44 és un marcador de cèl·lules mesenquimals. S'obtenen de
medul·la òssia, teixit adipós blanc, epidermis, musculatura i
perivasculatura.

El lipoaspirat és un mètode que dóna poques contaminacions. Per
degradar el teixit es fan servir col·lagenasa o dispasa. Es
centrifuga l'homogenat de teixit i s'obtenen 2 bandes: a la part
superior queda el greix i just per sota hi ha les MSC. Els primers
dies hi haurà molta mort cel·lular i després queda un cultiu
adherent. Són cèl·lules mononucleades.

Són cèl·lules multipotents. Exemple: no es poden diferenciar a
cèl·lules neuronals funcionals, ja que quan se'ls hi treu els factors
diferenciadors no mantenen el llinatge neuronal.

MSC de medul·la òssia es va veure que anaven bé per casos de
neurodegeneració. L'esclerosi múltiple presenta brots en què el
sistema va revertint els processos de desmielinització fins que la
malaltia cronifica. Quan es posaven MSC autòlogues, els pacients 
presentaven menys brots. 

Aquestes cèl·lules a part de la capacitat reparadora, poden modular el
sistema immunitari a la zona d'implantació.

Les cèl·lules mare mesenquimals es van identificar per primera vegada
en 1966 en els estudis realitzats per Friedenstein i Petrakova, que
van aïllar cèl·lules progenitores formadores de cartílag a partir de
cèl·lules de la medul·la òssia de rates amb una morfologia similar a
fibroblasts.

La font més estudiada i accessible de MSC és la medul·la òssia, encara
que MSC s'han aïllat a partir de més teixits, incloent el fetge, la
sang fetal, sang del cordó umbilical i el líquid amniòtic

Dins de la medul·la, les MSC comprenen 0,001 -0,1\% de la població
total de cèl·lules nucleades. Les MSC humanes s'aïllen de la medul·la
òssia, obtingudes principalment de la cresta ilíaca posterior de la
pelvis, i dels compartiments tibial i femoral.

Aquestes MSC es poden expandir àmpliament en cultius in vitro sense
pèrdua de funció o de fenotip.

MSCs se seleccionen a partir de cèl·lules mononuclears de moll d'os
(MNC) per la seva adherència a plàstic en cultiu i s'expandeixen en un
mitjà que consta d'Eagles Modificat Medi Dulbecco (baix nivell de
glucosa), suplementat amb 10\% de sèrum boví fetal (FBS) i L-glutamina
en 37ºC amb 5\% de CO2 a densitats d'1 x 104-0,4 x 106 cèl·lules /
cm2, segons el protocol estàndard.

Els cultius contenen cèl·lules de ràpida auto-renovació que es
mantenen en alts nivells durant diversos passatges si es mantenen els
cultius a baixa densitat, a més de les cèl·lules més grans i més
madures que predominen en passatges posteriors. Aquestes cèl·lules més
grans deixen la proliferació aproximadament al límit de Hayflick de 50
duplicacions de població. Hi ha moltes variacions de les condicions de
cultiu descrites, incloent el medi de cultiu privades de sèrum
suplementat amb citoquines específiques i factors de creixement
essencials per a l'expansió de MSC. En condicions de cultiu, les MSC
en el seu estat indiferenciat es pot observar per microscòpia de
contrast com una monocapa confluent adherent de cèl·lules en forma de
fus que tenen una aparença fibroblàstica.

Separaven les cèl·lules per gradient de densitat i les cultivaven. Hi
havia cèl·lules adherents (HSC) i en suspensió (cèl·lules
estromals). Creien que les estromals servien de suport a les HSC.

Aquests cultius costaven d'arrencar, però després presentaven una gran
proliferació. Aguanten fins a 20 \textit{passages}. Es va veure que el
responsable d'aquest fenomen era Dickkopf-1 (inhibidor de la
senyalització de Wnt). Quan un cultiu arriba a confluència, les
cèl·lules es comencen a tocar formant unions amb E-cadherina. Les
unions amb E-cadherina són dependents de beta-catenina. Si la
beta-catenina no està a les unions cel·lulars, es troba al nucli fent
de factor de transcripció. De forma natural, GSK-3beta fosforila a
beta-catenina i indueix la degradació al proteasoma i roman un flux
cap al nucli. Wnt inhibeix l'activitat GSK-3beta. Wnt funciona de
forma autocrina.

% Spees et al. PNAS, 2003

Els assajos de cicatrització es basen en lesionar una capa de
cèl·lules endotelials per avaluar \textit{in vitro} la capacitat de
regeneració i cicatrització. El KillerRedR emet fluorescència
vermella i genera ROS, que mata la cèl·lula. El KillerRed es basa en
transfectar un plàsmid que expressi aquest compost. El Rosa Bengala fa
el mateix però de manera massiva i s'emplea en experiments \textit{in
  vivo}.

Aquest paper feia lesions puntuals usant làser pulsat. Després
col·locaven MSC en aquesta lesió. Al cap d'un temps hi havia una
recuperació de cèl·lules La conclusió que van treure és que la MSC es
va transdiferenciar a cèl·lules endotelials. El que REALMENT va passar
és que la MSC es van fusionar amb cèl·lules endotelials (\textit{in vitro}).

\subsubsection{Aplicacions terapèutiques de les MSC}
Les aplicacions en teràpia són 2:
\begin{enumerate}
\item Immunomodulació


\item Reemplaçament cel·lular
\end{enumerate}

\paragraph{Immunomodulació} \hfill \\
Les MSC formen part de l'estroma de la medul·la òssia i donen suport
al microambient. SDF1 (CXCL12) és una citocina sintetizada per les
MSC. Aquesta molècula té 2 receptors: el CXCR4 i el CXCR7, els quals
competeixen per SDF1. Quan la cèl·lula expressa CXCR7, sempre lliga
SDF1. Si la cèl·lula expressa CXCR4 i CXCR7, CXCR7 fa les funcions de
dominant negatiu ja que té més afinitat. LA cèl·lula regula els
nivells de CXCR7 durant la migració. Provoca canvis al citosquelet,
proteïnes motores associades al citosquelet, integrines... CXCR4
s'expressa en HSC i fa que la HSC vagi al torrent sanguini. El binomi
SDF1/CXCR4 activen la via de les MAPK (diferenciació).

El tractament de MSC en pacients amb esclerosi múltiple anava molt bé,
els pacients milloraven símptomes. 

Les MSC disminueixen l'activació de
limfòcits i macròfags. Les MSC generen NO, PGE2, TGFbeta1 i la
kineurina disminueixen l'activació de limfòcits.

\paragraph{Reemplaçament cel·lular} \hfill \\
Pot regenerar tots els tipus cel·lulars derivats de mesènquima. A la
pràctica s'emplea per produir os, cartílag i teixits connectiu.

Si en un cultiu de MSC, es suplementa amb dexametasona, glicerol
fosfat, ascorbat i FBS; es comença a acumular calci i expressa colàgen
VIII. Després es poden trasplantar en os i tenen un percentatge d'èxit
alt.

Si en un cultiu de MSC, s'afegeix TFGbeta1, ascorbat i sense sèrum;
augmenta l'expressió de colàgen II i IX i agrecan i àcid hialurònic.