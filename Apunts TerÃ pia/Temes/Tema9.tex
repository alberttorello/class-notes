%------------------------------------------------------------------------------
% Tema 9. Estratègies de teràpia gènica
%------------------------------------------------------------------------------
\section{Estratègies de teràpia gènica}
\label{sec:estr-de-terap}

En general, hi ha 5 estratègies principals:
\begin{enumerate}
\item Augment dirigit d'expressió gènica
\item Expressió d'un gen suïcida
\item Eliminació de cèl·lules per estimulació del sistema immunitari
\item Inhibició dirigida d'expressió gènica
\item Correcció dirigida de la mutació
\end{enumerate}

\subsection{Increment d'expressió/addició gènica}
\label{sec:incr-dexpr-genica}
En una determinada malaltia en què hi ha una deficiència d'un producte
gènic concret, s'introdueix el gen correcte per tal que s'expressi.

\subsubsection{Fibrosi quística}
\label{sec:fibrosi-quistica}
És la malaltia monogènica més prevalent en la població europea. La
prevalença de portadors és de 1/25. En 1/2500 naixements es produeix
un cas de fibrosi quística.

La mutació al transportador de moc provoca una depleció de moc a les
mucoses. Es produeixen moltes infeccions i s'acumulen macròfags a la
zona, fet que provoca la necrosi del teixit.

El gen té 25 kb i codifica per un transportador actiu de clor. El gen
CFTR s'expressa a totes les glàndules exocrines.

Hi ha mutacions \textit{antisense} (de codons stop) que produeixen
proteïnes truncades, mutacions \textit{missense}... La $\Delta$F508 és
una deleció en pauta que provoca l'acumulació de la proteïna al RER,
on es destrueix i aquest és l'al·lel més freqüent a Europa.

Hi ha diverses estratègies de teràpia: afavorir el tràfic al RER,
provocar que el ribosoma no s'aturi al codó stop...

Els avantatges que hi ha per aplicar teràpia és que només s'ha de
dirgir a un teixit (el pulmonar, per ser el més afectat) i només que
s'aconsegueixi la transferència en poques cèl·lules, la condició
patològica millora considerablement. Pel que fa a la teràpia gènica,
no es poden utilitzar retrovirus ni lentivirus ja que el gen és molt
llarg i no l'admeten. Els adenovirus tenen un tropisme molt elevat per
cèl·lules tracte respiratori pero activen molt el sistema immunitari.

Els retrovirus només infecten cèl·lules en divisió. El delivery es pot
fer via sistèmica (sang) o via luminal (vies respiratòries).

Si la teràpia va a cèl·lules mare i és exitosa, n'hi pot haver prou
amb una administració. Si es transfecta cèl·lules quiescents, com que
es van renovant s'han de repetir les administracions.

Els assajos clínics han utilitzat liposomes de cèl·lules epitelials
amb nebulitzadors.

\subsubsection{Malaltia de Parkinson}
\label{sec:malalt-de-park}
És una malaltia neurodegenerativa progressiva en la que el sistema de
neurones dopaminèrgiques de la substància nigra degeneren, afectant
especialment al striatum. Causa tremolors involuntaris i disminució de
la velocitat i control dels moviments voluntaris.

La dopamina és convertida per la tirosina hidroxilasa (TH) en L-dopa,
que pot atravessar la barrera hematoencefàlica, on serà decarboxilada
a dopamina per la L-aminoàcid decarboxilasa (AADC). El tractament amb
dopamina no és possible, l'administració de L-dopa es va fent menys
efectiva amb el temps i té més efectes secundaris.

Els liposomes no atravessen eficientment la barrera
hematoencefàlica. Per a la teràpia gènica de la malaltia de Parkinson
tradicionalment s'han preferit els vectors vírics respecte els no
vírics, atès que no atravessen eficientment la barrera
hematoencefàlica, són poc eficients en l'entrega de gens i solen tenir
una expressió transitòria.

El vector ideal per teràpia gènica de la malaltia de Parkinson seria aquell que:
\begin{itemize}
\item Infecti cèl·lules no replicatives
\item No causi resposta immunitària
\item No sigui patogènic ni citotòxic
\item Pugui incorporar un gen com el GDNF
\item Permeti una expressió sostinguda del gen
\end{itemize}

% Gene therapy for Parkinson's disease: a step closer? The Lancet

% Long-term safety and tolerability of ProSavin, a lentiviral
% vector-based gene therapy for Parkinson's disease: a dose
% escalation, open label, phase 1/2 trial

\subsection{Eliminació de cèl·lules: gens suïcida i activació del
  sistema immunitari}
\label{sec:elim-de-cel.l}
Aquesta estratègia es focalitza en el tractament el càncer. Es pot fer
mitjançant la introducció específica de gens que indueixin la mort
tumoral o amb l'activació del sistema immunitari.

El propi gen pot codificar una toxina o bé un enzim que formi un
metabòlit tòxic. Aquest últim cas es pot controlar l'activació del
sistema mitjançant l'administració de pro-droga.

\subsubsection{Gens suïcida}
\label{sec:gens-suicida}
S'usa un enzim que participa en les vies de biosíntesi i reciclatge de
nucleòsids. Els exemples més coneguts són:
\begin{itemize}
\item El gen TK d'herpesvirus + acyclovir + ganaciclovir. El metabòlit
  normal no és tòxic per les cèl·lules que no expressin l'enzim
  d'herpesvirus. Es torna tòxic quan es fosforila.

\item Citosina deaminasa + 5-fluorocitosina
\end{itemize}

També es poden utilitzar enzims no relacionats amb metabolisme
cel·lular, com la linamarasa + linamarina.

En principi, interessa que presenti efecte de veïnatge (\textit{bystander
effect}), però l'abast i durada d'aquest efecte és imprevisible.

El glioblastoma multiforme és un tumor de cèl·lules glials molt greu. El
tractament amb cirurgia és molt conservador i normalment tenen lloc
recaigudes multifocals molt agressives. Per la teràpia, cal tenir en
compte que les cèl·lules tumorals es divideixen però les neuones no:
s'usen retrovirus per infectar cèl·lules tumorals ja que són
replicatives. El gen de la TK d'HSV es pot empaquetar en un
retrovirus. El retrovirus no s'administra via sistèmia, sinó que
s'injecta just després de l'exèresi del tumor per eliminar les
cèl·lules tumorals que quedin. L'eficiència no és total. Uns dies més
tard, s'administra per via intravenosa ganciclovir, que serà
fosforilat per la TK d'HSV  que expressin les cèl·lules tumorals. 

Els assajos van mostrar que hi havia més mort cel·lular de la que
esperaven segons l'eficiència de transfecció. Això és degut a que el
ganaciclovir trifosforilat passa per gap junctions i altres cèl·lules
adquireixin el ganaciclovir trifosforilat. S'han fet assajos en humans
però l'efecte va ser menor de l'esperat ja que els pacients
presentaven la malaltia ja multifocal.

El sistema de la limarinasa és un enzim de la planta de lli que la
protegeix dels paràsits. Aquest enzim produeix cianur a partir de
limarina i glucosa. El cianur difon cap a les cèl·lules veïnes ja que
és un gas. El punt clau és el control de la difusió del cianur
produït.

Es poden cultivar MSC del pacient i introduir-hi el gen terapèutic, ja
que s'infiltren a l'estroma tumoral. Mitjançant l'efecte de veïnatge
poden eliminar les cèl·lules tumorals properes.

Les estratègies de cavall de Troia han de ser molt dirigides.

\subsubsection{Immunomodulació}
\label{sec:immunomodulacio}
Intenta induir una resposta immunològica del propi pacient a les
lesions tumorals i metàstasis.

S'han explorat diverses alternatives:
\begin{itemize}
\item Que les cèl·lules tumorals expressin antígens específics de
  membrana no presents a les cèl·lules normals. Els neoantígens
  funcionen com a adjuvants. El que es fa és introduir un gen que
  codifiqui per antigen forani que activi el sistema immunitari. Una
  alternativa consisteix en transduir les cèl·lules tumorals amb
  molècules co-estimuladores, que milloren la presentació d'antígens
  o que incrementen la resposta immune contra tipus/antígens
  cel·lulars concrets.

\item Una de les estratègies consisteix en obtenir cèl·lules tumorals
  irradiades del pacient, prèviament transduïdes amb una citoquina
  per a estimular la resposta immune.
\end{itemize}

Molt sovint es combina la presentació de neoantígens i activació del
sistema immunitari.

S'extreu el tumor del pacient i s'aïllen els limfòcits infiltrats, es
cultiven amb IL-2. S'infecta el cultiu amb TNF i es seleccionen les
cèl·lules transfectades i s'injecten al tumor.

La contrapartida són la possible aparició de malalties autoimmunes.

PD-1 inhibeix els limfòcits NK. El melanoma expressa PD-1. Si el
melanoma travessa la làmina basal de la dermis de seguida fa
metàstasi. Els melanòcits s'originen a la cresta neural.

Els limfòcits T d'un pacient amb un tumor no són tant eficients en
reconèixer antígens com les d'una persona normal. S'aïllen els
antígens del tumor, s'insereixen en cèl·lules dendrítiques que es
cultiven amb limfòcits T d'un donant sa compatible. S'expandeixen els
limfòcits T i finalment s'injecten al pacient.

\subsection{Inhibició dirigida de l'expressió gènica}
\label{sec:inhib-dirig-de}


\subsection{Correcció de la mutació}
\label{sec:correccio-de-la}

