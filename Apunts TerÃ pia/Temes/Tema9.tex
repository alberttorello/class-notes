%------------------------------------------------------------------------------
% Tema 9. Estratègies de teràpia gènica
%------------------------------------------------------------------------------
\section{Estratègies de teràpia gènica}
\label{sec:estr-de-terap}

En general, hi ha 5 estratègies principals:
\begin{enumerate}
\item Augment dirigit d'expressió gènica
\item Expressió d'un gen suïcida
\item Eliminació de cèl·lules per estimulació del sistema immunitari
\item Inhibició dirigida d'expressió gènica
\item Correcció dirigida de la mutació
\end{enumerate}

\subsection{Increment d'expressió/addició gènica}
\label{sec:incr-dexpr-genica}
En una determinada malaltia en què hi ha una deficiència d'un producte
gènic concret, s'introdueix el gen correcte per tal que s'expressi.

\subsubsection{Fibrosi quística}
\label{sec:fibrosi-quistica}
És la malaltia monogènica més prevalent en la població europea. La
prevalença de portadors és de 1/25. En 1/2500 naixements es produeix
un cas de fibrosi quística.

La mutació al transportador de moc provoca una depleció de moc a les
mucoses. Es produeixen moltes infeccions i s'acumulen macròfags a la
zona, fet que provoca la necrosi del teixit.

El gen té 25 kb i codifica per un transportador actiu de clor. El gen
CFTR s'expressa a totes les glàndules exocrines.

Hi ha mutacions \textit{antisense} (de codons stop) que produeixen
proteïnes truncades, mutacions \textit{missense}... La $\Delta$F508 és
una deleció en pauta que provoca l'acumulació de la proteïna al RER,
on es destrueix i aquest és l'al·lel més freqüent a Europa.

Hi ha diverses estratègies de teràpia: afavorir el tràfic al RER,
provocar que el ribosoma no s'aturi al codó stop...

Els avantatges que hi ha per aplicar teràpia és que només s'ha de
dirgir a un teixit (el pulmonar, per ser el més afectat) i només que
s'aconsegueixi la transferència en poques cèl·lules, la condició
patològica millora considerablement. Pel que fa a la teràpia gènica,
no es poden utilitzar retrovirus ni lentivirus ja que el gen és molt
llarg i no l'admeten. Els adenovirus tenen un tropisme molt elevat per
cèl·lules tracte respiratori pero activen molt el sistema immunitari.

Els retrovirus només infecten cèl·lules en divisió. El delivery es pot
fer via sistèmica (sang) o via luminal (vies respiratòries).

Si la teràpia va a cèl·lules mare i és exitosa, n'hi pot haver prou
amb una administració. Si es transfecta cèl·lules quiescents, com que
es van renovant s'han de repetir les administracions.

Els assajos clínics han utilitzat liposomes de cèl·lules epitelials
amb nebulitzadors.

\subsubsection{Malaltia de Parkinson}
\label{sec:malalt-de-park}
És una malaltia neurodegenerativa progressiva en la que el sistema de
neurones dopaminèrgiques de la substància nigra degeneren, afectant
especialment al striatum. Causa tremolors involuntaris i disminució de
la velocitat i control dels moviments voluntaris.

La dopamina és convertida per la tirosina hidroxilasa (TH) en L-dopa,
que pot atravessar la barrera hematoencefàlica, on serà decarboxilada
a dopamina per la L-aminoàcid decarboxilasa (AADC). El tractament amb
dopamina no és possible, l'administració de L-dopa es va fent menys
efectiva amb el temps i té més efectes secundaris.

Els liposomes no atravessen eficientment la barrera
hematoencefàlica. Per a la teràpia gènica de la malaltia de Parkinson
tradicionalment s'han preferit els vectors vírics respecte els no
vírics, atès que no atravessen eficientment la barrera
hematoencefàlica, són poc eficients en l'entrega de gens i solen tenir
una expressió transitòria.

El vector ideal per teràpia gènica de la malaltia de Parkinson seria aquell que:
\begin{itemize}
\item Infecti cèl·lules no replicatives
\item No causi resposta immunitària
\item No sigui patogènic ni citotòxic
\item Pugui incorporar un gen com el GDNF
\item Permeti una expressió sostinguda del gen
\end{itemize}

% Gene therapy for Parkinson's disease: a step closer? The Lancet

% Long-term safety and tolerability of ProSavin, a lentiviral
% vector-based gene therapy for Parkinson's disease: a dose
% escalation, open label, phase 1/2 trial

\subsection{Eliminació de cèl·lules: gens suïcida i activació del
  sistema immunitari}
\label{sec:elim-de-cel.l}


\subsection{Inhibició dirigida de l'expressió gènica}
\label{sec:inhib-dirig-de}


\subsection{Correcció de la mutació}
\label{sec:correccio-de-la}

