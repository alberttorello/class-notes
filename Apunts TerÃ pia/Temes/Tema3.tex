%------------------------------------------------------------------------------
% Tema 3. Malalties coronàries
%------------------------------------------------------------------------------
\section{Malalties coronàries}

Posem el cas que hi ha el tamponament d'una artèria coronària. Si s'afecta una branca primària, tot el que queda per sota mor. Si afecta una terciària, lògicament l'afectació és menor. El gruix del ventricle disminueix i quan arriba a ser un 40\% menor de l'inicial és un problema greu, és una afectació a llarg termini degut a la mort cel·lular.

El cateterisme coronari es basa en netejar el trombe i la col·locació d'un stent (per mantenir l'artèria oberta). L'altre opció és un bypass coronari.

Al cor hi ha unes cèl·lules que estan en un nínxol quiescent que generen poques cèl·lules. La resposta endògena està dirigida a renovació i manteniment. La complicació és que les cèl·lules generades s'han d'incorporar a un sistema contràctil i amb activitat elèctrica. El pericardi reacciona i cobreix la zona de fibroblasts. La cicatriu no presenta activitat elèctrica i no la transmet, no es contrau. Llavors es generen arrítimia. El cor perd força d'ejecció i els nivells d'oxigen en sang baixen i es genera més músculatura. L'arrítimia es complica fins a una insuficiència coronària.

La injecció de cèl·lules directament al cor és un problema ja que el cor presenta moviment i les cèl·lules es perden. S'injecten llavors amb un hidrogel per augmentar la densitat de la solució. L'altra opció és col·locar un pegat amb cèl·lules sobre la zona lesionada. L'altre estratègia és induir la proliferació endògena al cor.

Els 2 ventricles del cor s'originen de progenitors diferents. L'aurícula té un altre origen. Hi ha 9 poblacions diferents descrites al cor. El ventricle esquerre és el més gran però és el que té menys cèl·lules.

S'han provat nombroses teràpies cel·lulars per reparar les lesions d'un infart; tant teràpies de recuperació  (citocines i quimioquines) o de regeneració.

El c-kit és un receptor tirosina quinasa que es presenta només en algunes cèl·lules. MDR-1 és un altre marcador. c-kit és el marcador més general.

Ús de cèl·lules de cor
% diapo 6: ninxol cel·lular

L'anàlisi clonal es basa en aïllar una cèl·lula i fer-la proliferar en cultiu. Si s'injecten en un cor, les cèl·lules s'integren. L'inconvenient és l'eficiència. Les cèl·lules s'integren més en un ventricle que en un altre.

% Ús de HSC
Es provoca una isquèmia en rates femella. Per altra banda, hi ha mascles que expressen GFP i se'ls fa una punció ilíaca. Es van barrejar les cèl·lules amb un hidrogel. Els hidrogels s'obtenen de matriu extracel·lular de sarcomes en cultiu.

Van injectar aquest hidrogel + aspirat medul·lar.  %diapo 12.

Més tard, es va demostrar que les HSC no servien per infart de miocardi.

Llavors es van fer estudis amb MSC i el cor recuperava el gruix del ventricle després del tractament amb MSC. El problema era que la diferenciació no generava múscul cardíac i les cèl·lules no anaven en sincronia amb la resta del cor. Les MSC no guanyaven potència d'ejecció i al cap d'un temps generaven arrítmies.