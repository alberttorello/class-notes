%------------------------------------------------------------------------------
% Tema 3. Malalties coronàries
%------------------------------------------------------------------------------
\section{Malalties coronàries}

Posem el cas que hi ha el tamponament d'una artèria coronària. Si s'afecta una branca primària, tot el que queda per sota mor. Si afecta una terciària, lògicament l'afectació és menor. El gruix del ventricle disminueix i quan arriba a ser un 40\% menor de l'inicial és un problema greu, és una afectació a llarg termini degut a la mort cel·lular.

El cateterisme coronari es basa en netejar el trombe i la col·locació d'un stent (per mantenir l'artèria oberta). L'altre opció és un bypass coronari.

Al cor hi ha unes cèl·lules que estan en un nínxol quiescent que generen poques cèl·lules. La resposta endògena està dirigida a renovació i manteniment. El pericardi reacciona i cobreix la zona de fibroblasts. La cicatriu no presenta activitat elèctrica i no la transmet, no es contrau. Llavors es generen arrítimia. El cor perd força d'ejecció i els nivells d'oxigen en sang baixen i es genera més músculatura. L'arrítimia es complica fins a una insuficiència coronària.

La injecció de cèl·lules directament al cor és un problema ja que el cor presenta moviment i les cèl·lules es perden. S'injecten llavors amb un hidrogel per augmentar la densitat de la solució. L'altra opció és col·locar un pegat amb cèl·lules sobre la zona lesionada. L'altre estratègia és induir la proliferació endògena al cor.

Els 2 ventricles del cor s'originen de progenitors diferents. L'aurícula té un altre origen. Hi ha 9 poblacions diferents descrites al cor. El ventricle esquerre és el més gran però és el que té menys cèl·lules.