%------------------------------------------------------------------------------
% Tema 13. Barreres cel·lulars
%------------------------------------------------------------------------------
\section{Barreres cel·lulars}
\label{sec:barreres-cel.lulars}

\subsection{Entrega sistèmica i barreres extracel·lulars}
\label{sec:entrega-sistemica-i}
Cal tenir en compte que l'entrega sistèmica de material genètic nu, o en vectors virals o físics presenta diferents barreres:
\begin{itemize}
\item \textbf{Metabolisme sistèmic}
  \begin{itemize}
  \item Endonucleases sèriques
  \item Interaccions inespecífiques amb proteïnes
  \item Interaccions específiques amb cèl·lules
  \item Fagocitosi
  \item Excreció renal
  \end{itemize}
\item \textbf{Barreres extracel·lulars:}
  \begin{itemize}
  \item Endoteli
  \item Membranes
  \item Mucus
  \end{itemize}
\item \textbf{Barreres cel·lulars:}
  \begin{itemize}
  \item Interacció amb la superfície cel·lular
  \item Mecanismes d'internalització
  \item Alliberament de les vesícules endocítiques
  \item Mobilitat i estabilitat dels àcids nucleics lliures al citoplasma
  \item Transport al nucli
  \end{itemize}
\end{itemize}

\subsubsection{Metabolisme sistèmic}
\label{sec:metabolisme-sistemic}

\paragraph{Endonucleases sèriques}
Les nucleases del plasma degraden el DNA nu i els oligonucleòtids.
El DNA plasmídic (nu o complexat) administrat per via intravenosa ràpidament es desenrotlla a formes circulars obertes, que estan exposades a les nucleases.

Els siRNAs (o similars) no modificats o sense protecció són degradats per ribonucleases de plasma.

\paragraph{Agregació amb proteïnes sèriques}
Els poliplexes o lipoplexes carregats positivament s'uneixen a les proteïnes del sèrum carregades negativament (com ara albúmina), reduint d'aquesta manera l'eficàcia de la transfecció.
La PEGilació (unió de molècules de polietilenglicol a la superfície de lipoplexes o polipolexes carregats positivament) pot apantallar les càrregues positives i evitar la interacció amb les proteïnes plasmàtiques.

Si la transfecció fos in vitro, no faria falta ja que interessa que el lipoplexe tingui càrrega positiva perquè així interacciona més fàcilment amb la cèl·lula.

La pegilació, és a dir la conjugació amb polietilenglicol (PEG), dels polímers catiònics permet una reducció de la unió no específica a les superfícies cel·lulars.

La pegilació es pot fer reversible via utilització de connectors que es poden trencar per diferents mecanismes.

\paragraph{Agregació amb cèl·lules sanguínies i fagocitosi}
Alguns vectors tipus molècules catiòniques poden unir-se a qualsevol tipus de cèl·lula in vivo incloent cèl·lules de la sang i macròfags.

La unió a cèl·lules sanguínies (riques en heparan sulfats) poden provocar la seva agregació i la formació de trombes.

La unió a macròfags resulta en un aclariment ràpid del torrent sanguini per fagocitosi.

Per exemple: liposomes catiònics convencionals s'eliminen en última instància, de la circulació sanguínia pel sistema dels macròfags.

La PEGilació dels vectors redueix el risc en l'administració intravascular de vectors carregats positivament

\paragraph{Excreció renal}
Molècules de vector neutres, feblement aniòniques o amb càrregues no exposades (vectors PEGilats) tenen poca afinitat per la superfície cel·lular i s'eliminen més lentament, en funció de la seva grandària molecular que determina el seu grau de filtració glomerular renal.

La filtració renal al glomèrul depèn de la mida molecular, les molècules <50 kDa són filtrades, poden ser reabsorbides en els túbuls renals i s'excreten en l'orina.

En conseqüència, les molècules petites per sota del llindar de la mida filtrada al glomèrul s'eliminen ràpidament del plasma per excreció renal.

Els oligonucleòtids i els siRNAs són excretats ràpidament en l'orina; quan es lliguen a vectors s'impedeix l'excreció renal i això és essencial per millorar la farmacocinètica després de l'administració sistemàtica.

\subsubsection{Barreres extracel·lulars}
\label{sec:barr-extr}

\paragraph{Endoteli}
Les grans molècules que circulen en sang només poden accedir a cèl·lules sanguínies, cèl·lules endotelials; i les cèl·lules del parènquima en el fetge i la melsa, que tenen un endoteli discontinu.

En el cas dels capil·lars continus no fenestrats, l'aigua i els soluts petits (radi molecular < 3 nm) passen entre les cèl·lules de l'endoteli, mentre que els soluts més grans passar a través de les cèl·lules de l'endoteli o bé a través de canals transendotelials o transcitosis, intervinguda principalment per cavèoles.

Els capilars continus fenestrats presenten una major permeabilitat a l'aigua i soluts petits, però els coeficients de reflexió similars a macromolècules grans (els diafragmes de la Llei de fenestras com filtres moleculars).

Els capil·lars discontinus i sinusoïdals tenen fenestracions (sense diafragmes), gaps, i una membrana basal mal organitzada. Aquestes cèl·lules endotelials contenen moltes depressions revestides de clatrina, les quals juguen un paper important en l'endocitosi mediada per receptor (encara que també poden prendre part en la transcitosis) que inclou compartiments endosomal i lisosomals.

\paragraph{Membranes}
Estructures de la matriu extracel·lular que separen les cèl·lules del teixit connectiu circumdant. Això es produeix en diversos teixits i tumors sòlids. Les membranes limiten la difusió de les molècules en el teixit.

\paragraph{Mucus}
Secreció viscosa que conté proteïnes que impedeix la difusió de molècules.

\subsection{Interacció cel·lular}
\label{sec:interaccio-cel.lular}
La interacció dependrà de si el vector té:
\begin{itemize}
\item Cobertura amb lligand: Adsorció/adhesió per interacció electrostàtica.
\item Cobertura amb lligand: Reconeixement per una proteïna cel·lular de proteïnes de la càpside viral, altres proteïnes, pèptids RGD o residus glucídics.
\end{itemize}

En el cas dels àcids nucleics nus, el mecanisme és força desconegut.

\subsubsection{Cobertura sense lligand}
\label{sec:cobert-sense-llig}
El vector és adsorbit en la membrana cel·lular depenent de les propietats de bioadhesió. L'abundància de llocs aniònics a la membrana cel·lular (proteoglicans de la superfície cel·lular sulfatats carregats negativament) implica que grups catiònics tenen una tendència a "enganxar-se".

Això ha estat aprofitat en els liposomes catiònics i polímers catiònics.

La PEGilació dels polímers catiònics permet una reducció de la unió no específica a les superfícies cel·lulars.

\subsubsection{Cobertura amb lligand}
\label{sec:cobert-amb-llig}
Les proteïnes lligands són proteïnes que interaccionen amb receptors o antígens de la membrana, lligands o anticossos, respectivament.

Moltes integrines de la superfície cel·lular s'uneixen a les seqüències de pèptids Arg-Gly-Asp (RGD) presents en les proteïnes de matriu extracel·lular i sèrum.

RGD s'utilitza per dirigir la interacció dels vectors amb la cèl·lula mitjançant les integrines. Algunes integrines intervenen a més en la internalització dels lligands RGD a través de la via d'endocitosi. S'usa molt per transfectar tumors.

La mannosa interacciona amb el receptor de mannosa. Aquest s'expressa preferentment a macròfags i cèl·lules dendrítiques. El receptor de mannosa uneix glicoproteïnes que contenen mannosa, glucosa, fucosa i N-acetilglucosamina.

Galactosa: els residus de galactosa acoblats a la polietilenimina (PEI) transfecten de forma selectiva hepatòcits a través del receptor de les asialoglicoproteïnes. Aquest receptor s'uneix naturalment a les asialoglicoproteïnes, això és, glicoproteïnes de les quals ha estat retirat l'àcid siàlic quedant exposats els residus de galactosa.

El receptor de les asialoglicoproteïnes s'expressa a les cèl·lules del fetge per eliminar glicoproteïnes diana de la circulació.

% Taula de lligands de càpsides virals


\subsection{Captació}
\label{sec:captacio}

\subsubsection{Fusió amb la membrana}
\label{sec:fusio-amb-la}
Les cobertures amb determinats lípids fusionen amb la membrana plasmàtica i d'aquesta manera les molècules associades són alliberades directament al citosol.

Els vectors que es fusionen amb la membrana són:
\begin{itemize}
\item Virus amb embolcall lipídic: retrocirus, herpes, vaccinia virus (cowpox).
\item Liposomes
\end{itemize}

\subsubsection{Endocitosi}
\label{sec:endocitosi}
La ruta endocítica per la qual una molècula o virus entra a la cèl·lula és altament dependent de la seva mida, la càrrega i la composició, així com en el tipus de cèl·lula en què està entrant.

\paragraph{Endocitosi mitjançada per receptor}
Procés pel qual les cèl·lules internalitzen molècules en vesícules que contenen les molècules de lligand, en aquest cas molècules del vector (proteïnes, carbohidrats, ..), i el receptor de membrana que reconeix el lligand.

\begin{itemize}
\item \textbf{Vesícules recobertes de clatrina:} la via endocítica implica la fusió d'endosomes amb els lisosomes (vesícules que contenen enzims per la degradació) i per tant la degradació dels vectors i DNA. La mida òptima del lligand és de 100 nm. És utilitzada per adenovirus 2/5, AAV, retrovirus, liposomes i polications petits.

\item \textbf{Caveoles:} la mida típica del lligand és de 500 nm. És utilitzada per polications grans. La captació de molècules també és mitjançada per receptors.

\item \textbf{Lípid rafts:} són microdominis de la membrana cel·lular enriquida en colesterol i esfingolípids. És utilitzada per retrovirus. Les vesícules no dependents de clatrina semblen no fusionar amb els lisosomes i dirigir-se directament al reticle endoplasmàtic.
\end{itemize}

\paragraph{Fagocitosi}
Els fagosomes són vesícules que encapsulen molècules grans. La fagocitosi es realitza principalment per cèl·lules especialitzades en eliminar els patògens de grans dimensions o restes: macròfags i neutròfils. La fagocitosi requereix d'una certa especificitat: la unió del lligand a receptors cel·lulars d'adherència . Els fagosomes maduren per fusionar amb els lisosomes.

\paragraph{Macropinocitosi}
Consisteix en la invaginació de la membrana plasmàtica i formació d'endosomes per captar proteïnes individuals i fluids.

La destinació intracel·lular dels macropinosomes depèn del tipus cel·lular: en els macròfags es fusionen posteriorment amb els lisosomes, mentre que altres cèl·lules, eventualment reciclen la major part dels continguts de tornada a la superfície cel·lular.

Els macropinosomes són inherentment vesícules que degoten en comparació amb altres tipus d'endosomes.

Sembla ser que aquest procés és molt important pels tumors per captar nutrients de l'exterior cel·lular.

\subsubsection{Pèptids de penetració cel·lular}
\label{sec:pept-de-penetr}
Els pèptids de penetració cel·lular (CPPs) presenten diferents mecanismes de translocació de la membrana cel·lular

Alguns CPPs i les seves càrregues poden penetrar directament a través de la membrana plasmàtica, segons els models: 
\begin{itemize}
\item Catifa, en el qual hi ha una desestabilització transitòria de la membrana cel·lular induïda per l'associació del pèptid i reorganització consegüent de fosfolípids.
\item Micel·la invertida, per la pertorbació de la membrana plasmàtica formant una estructura hexagonal en un procés reversible.
\item Porus transitoris produïts després de la inserció i oligomerització dels pèptids en una estructura d'anell de formes diferents.
\end{itemize}

No obstant, en general s'accepta que els CPPs usen sobretot les vies endocítiques per entrar en les cèl·lules. El mètode de captació endocítica varia segons la grandària i la naturalesa del complex COP-àcid nucleic. Poden ser vies dependents de clatrina, vies clatrina-independents i macropinocitosi.

\subsection{Sortida de l'endosoma i estabilitat al citoplasma}
\label{sec:sortida-de-lendosoma}

Estructures hexagonals de fusió: els liposomes catiònics multilaminars després de l'endocitosi formen estructures hexagonals que fusionen amb les membranes aniòniques endosòmiques per alliberar l'àcid nucleic al citosol. Els poliplexos dendrímers també semblen formar aquestes estructures.

Els polications tipus poliamines actuen com una esponja que atrapa H+, Cl- i aigua, infla els endosomes i fa que esclatin alliberant el seu contingut al citosol.

També tenen un efecte paraigües: la protonació de les amines al baix pH fa que el polímer passi d'una conformació condensada a un estesa.

Alguns àcids nucleics són efectius al citosol.
\begin{itemize}
\item mRNAs sintètics (modificades químicament) codificants de proteïna.
\item Oligonucleòtids antisentit.
\item siRNAs.
\item miRNAs.
\end{itemize}

El DNA codificant ha d'arribar al nucli per ser transcrit. Entre aquests:
\begin{itemize}
\item DNA codificant per proteïna.
\item DNA codificant per RNA antisentit: shRNA i pir-miRNAmímics.
\end{itemize}

Els àcids nucleics de cadena simple o doble són degradats al citoplasma per nucleases.

Els de cadena simple són més fàcilment degradats que els de cadena doble.

Els àcids nucleics lliures al citoplasma interactuen amb biomolècules catiòniques multivalents com ara espermina i el DNA amb proteïnes d'unió del DNA.

Aquestes molècules condensen l'àcid nucleic i el protegeixen de la degradació.

\subsection{Transport nuclear}
\label{sec:transport-nuclear}

\subsubsection{Mitosi}
\label{sec:mitosi}
Quan les cèl·lules realitzen la mitosi, es trenca la membrana nuclear i el vector/DNA present en el citoplasma es pot barrejar amb els components nucleoplàsmics.

No obstant, només els propis cromosomes i les macromolècules físicament associades amb ells queden inclosos en els nuclis recentment formats, mentre que els orgànuls i altres macromolècules grans que no poden passar a través dels NPC queden excloses.

Els oncoretrovirus depenen de la mitosi per transduir les cèl·lules: no poden transduir cèl·lules en repòs (quiescents).

\subsubsection{Importació/exportació nuclear}
\label{sec:import-nucl}

Tot el trànsit de molècules que entra o surt del nucli intacte es produeix a través del complex de porus nuclears (CPN).
\begin{itemize}
\item \textbf{Difusió:} CPN permet la difusió de molècules, dirigides pel gradient de concentració, de mida < 10 nm (ions, metabòlits o proteïnes petites < 40 kDa) a través dels extrems del canal central.
\item \textbf{Transport actiu:} Les proteïnes > 60 kDa han de ser transportades de manera activa. El canal pot augmentar la grandària fins a un màxim de 40 nm, depenent del subministrament d'energia (hidròlisi de GTP), i el transport requereix la presència d'un senyal de localització nuclear a la càrrega. La translocació a través del CPN cap al nucli i des del nucli al citoplasma es regeix per una classe de proteïnes conegudes com importines i exportines, respectivament.
\end{itemize}

Poc després de la mitosi, però, els NPCs en els recentment formats nuclis són més permeables, el qual trasllada temporalment els criteris d'exclusió per grandària a pesos moleculars més grans.

En funció de l'estrutura i mida de l'àcid nucleic, hi ha diferents maneres d'entrar al nucli:
\begin{itemize}
\item \textbf{DNA i RNA de cadena simple:} Difonen del citoplasma al nucli ràpidament (en segons) i fàcil. El transport és independent de factors citosòlics.

\item \textbf{DNA de doble cadena (> 250 bp):} Procedeix principalment per transport actiu via CPN. El transport és estimulat per l'associació amb proteïnes que contenen senyals de translocació al nucli (NLS, \textit{nuclear localization signal}).

\item Les \textbf{nanopartícules} de petita grandària (10-20 nm) no es troben sovint en el nucli de les cèl·lules, malgrat que la seva petita grandària probablement permet l'accés passiu a l'interior a través del CPN.
\end{itemize}

