%------------------------------------------------------------------------------
% Tema 3. Malalties autosòmiques dominants
%------------------------------------------------------------------------------
\section{Malalties autosòmiques dominants}

Les bases moleculars de la dominància són:
\begin{itemize}
\item Insuficient producte gènic (pèrdua de funció)

\item Aparició de formes dominant negatives: L'al·lel mutat perd la funció i el producte gènic impedeix que l'al·lel no mutat compensi els efectes de la mutació.

\item Guany de funció: La mutació afecta un al·lel, que adquireix unes propietats diferents a l'al·lel original i distorsionen l realitat.
\end{itemize}

La hipercolesterolèmia familiar és la malaltia mendeliana més freqüent. Una mutació de pèrdua de funció en un al·lel és suficient per provocar el fenotip.

\subsection{Síndrome de deficiència de GLUT1}
Abans d’establir-se la causa molecular, s'anomenava discinèsia paroxística cinesigènica familiar (distonia 9); i epilèpsia paroxística discinèsia (distonia 18) induïda per l'exercici; i certs tipus d'epilèpsia, i altres noms a tipus d’epilèpsia.

Els malalts tenen crisis epilèptiques a la infància, sovint als primers mesos de vida.

És un trastorn que afecta el sistema nerviós que pot tenir una varietat de signes i símptomes neurològics. Aproximadament el 90 per cent de les persones afectades tenen una forma del trastorn què sovint es refereix com a síndrome de deficiència de GLUT1 comú.

El pacients tenen en general:
\begin{itemize}
\item Freqüents convulsions (epilèpsia) a partir dels primers mesos de vida. En els nadons, el primer signe de la malaltia pot ser moviments oculars involuntaris que són ràpids i irregulars. L’exercici físic sol ser un inductor de l’

\item Els nadons amb síndrome de deficiència de GLUT1 comú tenen una mida normal del cap en néixer, però el creixement del cervell i el crani és sovint lent, el que pot resultar en una mida de cap anormalment petita.

\item Retard del desenvolupament o discapacitat intel·lectual.

\item Altres problemes neurològics, com la rigidesa causada per tibar anormalment els músculs (espasticitat), dificultats en la coordinació de moviments (atàxia), i dificultats en la parla (disàrtria). Alguns experimenten episodis de confusió, falta d'energia (letargia), mals de cap, o contraccions musculars (mioclonía), en particular durant els períodes de dejuni (sense menjar).

\item En alguns casos poc freqüents pot haver-hi anèmia, amb eritròcits de morfologia alterada 
\end{itemize}

Té una freqüència netre 3000 i 7000 individus.

La malaltia no té heterogeneïtat de locus però sí que té heterogeneïtat al·lèlica.

GLUT1 transporta la glucosa a dins de la cèl·lula mitjançant ``difusió facilitada''. GLUT1 presenta una expressió molt ubiqua (eritròcits, encèfal, placenta, etc). La glucosa és essencial per a la funció de les neurones, han d'haver-hi processos de captació que la sostinguin, en presència de la barrera hematoencefàlica. GLUT1 es troba en les cèl·lules endotelials que formen la barrera hematoencefàlica i en els astròcits, mentre que en les neurones es troba GLUT3. Les neurones poden metabiolitzar la glucosa de manera directa o el lactat produït pels astròcits. 

La única teràpia és controlar l'exercici físic (no disminuir la disponibilitat de glucosa al cervell) i dietes molt cetogèniques. En casos exepcionals, el cervell es pot acondicionar a metabolitzar cossos cetònics. Així es compensa la falta de glucosa al cervell. Es poden donar suplements d'àcids grassos de cadena curta (C7), que poden arribar directament al cervell pasant la barrera hematoencefàlica. Una altra estratègia és estimular p.e amb hormones l'expressió de GLUT1 i aconseguit prou GLUT1.

\subsection{Polineuropatia amilodòptica familiar de tipus I}
És una malaltia dominant de guany de funció. La malaltia és asimptomàtica.

Afecta el sistema GI. L'esperança de vida no va més ellà de 10 anys. +

La transsiretina és una proteïna de transport a la sang. És un transportado sanguiini. Producció hepàtica. La proteïna funciojs com a tràiler. 

És molt freqüent al nord de Portugal. Zones costeres de Suècia i Japó.

És una malaltia amb penetrància incompleta. S'explica per factors ambientals i per aspectes genètics fora del locus afectat (gens/polimorfismes protectors).

A Suècia van fer un screening poblacional i van veure que hi havia famílies que portaven la mutació 4 o 5 generacions sense afectats i altres famílies amb una recurrència molt alta.

% Perquè es tracta amb trasplantament de fetge.