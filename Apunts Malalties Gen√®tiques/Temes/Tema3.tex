%------------------------------------------------------------------------------
% Tema 3. Malalties autosòmiques dominants
%------------------------------------------------------------------------------
\section{Malalties autosòmiques dominants}

Les bases moleculars de la dominància són:
\begin{itemize}
\item Insuficient producte gènic (pèrdua de funció)

\item Aparició de formes dominant negatives: L'al·lel mutat perd la funció i el producte gènic impedeix que l'al·lel no mutat compensi els efectes de la mutació.

\item Guany de funció: La mutació afecta un al·lel, que adquireix unes propietats diferents a l'al·lel original i distorsionen l realitat.
\end{itemize}

La hipercolesterolèmia familiar és la malaltia mendeliana més freqüent. Una mutació de pèrdua de funció en un al·lel és suficient per provocar el fenotip.

\subsection{Síndrome de deficiència de GLUT1}
Abans d’establir-se la causa molecular, s'anomenava discinèsia paroxística cinesigènica familiar (distonia 9); i epilèpsia paroxística discinèsia (distonia 18) induïda per l'exercici; i certs tipus d'epilèpsia, i altres noms a tipus d’epilèpsia.

Els malalts tenen crisis epilèptiques a la infància, sovint als primers mesos de vida.

És un trastorn que afecta el sistema nerviós que pot tenir una varietat de signes i símptomes neurològics. Aproximadament el 90 per cent de les persones afectades tenen una forma del trastorn què sovint es refereix com a síndrome de deficiència de GLUT1 comú.

El pacients tenen en general:
\begin{itemize}
\item Freqüents convulsions (epilèpsia) a partir dels primers mesos de vida. En els nadons, el primer signe de la malaltia pot ser moviments oculars involuntaris que són ràpids i irregulars. L’exercici físic sol ser un inductor de l’

\item Els nadons amb síndrome de deficiència de GLUT1 comú tenen una mida normal del cap en néixer, però el creixement del cervell i el crani és sovint lent, el que pot resultar en una mida de cap anormalment petita.

\item Retard del desenvolupament o discapacitat intel·lectual.

\item Altres problemes neurològics, com la rigidesa causada per tibar anormalment els músculs (espasticitat), dificultats en la coordinació de moviments (atàxia), i dificultats en la parla (disàrtria). Alguns experimenten episodis de confusió, falta d'energia (letargia), mals de cap, o contraccions musculars (mioclonía), en particular durant els períodes de dejuni (sense menjar).

\item En alguns casos poc freqüents pot haver-hi anèmia, amb eritròcits de morfologia alterada 
\end{itemize}

Té una freqüència netre 3000 i 7000 individus.

La malaltia no té heterogeneïtat de locus però sí que té heterogeneïtat al·lèlica.

GLUT1 transporta la glucosa a dins de la cèl·lula mitjançant ``difusió facilitada''. GLUT1 presenta una expressió molt ubiqua (eritròcits, encèfal, placenta, etc). La glucosa és essencial per a la funció de les neurones, han d'haver-hi processos de captació que la sostinguin, en presència de la barrera hematoencefàlica. GLUT1 es troba en les cèl·lules endotelials que formen la barrera hematoencefàlica i en els astròcits, mentre que en les neurones es troba GLUT3. Les neurones poden metabiolitzar la glucosa de manera directa o el lactat produït pels astròcits. 

La única teràpia és controlar l'exercici físic (no disminuir la disponibilitat de glucosa al cervell) i dietes molt cetogèniques. En casos exepcionals, el cervell es pot acondicionar a metabolitzar cossos cetònics. Així es compensa la falta de glucosa al cervell. Es poden donar suplements d'àcids grassos de cadena curta (C7), que poden arribar directament al cervell pasant la barrera hematoencefàlica. Una altra estratègia és estimular p.e amb hormones l'expressió de GLUT1 i aconseguit prou GLUT1.

\subsection{Polineuropatia amilodòptica familiar de tipus I}
És una malaltia dominant de guany de funció. La malaltia és asimptomàtica.

Afecta el sistema GI. L'esperança de vida no va més ellà de 10 anys. +

La transsiretina és una proteïna de transport a la sang. És un transportado sanguiini. Producció hepàtica. La proteïna funciojs com a tràiler. 

És molt freqüent al nord de Portugal. Zones costeres de Suècia i Japó.

És una malaltia amb penetrància incompleta. S'explica per factors ambientals i per aspectes genètics fora del locus afectat (gens/polimorfismes protectors).

A Suècia van fer un screening poblacional i van veure que hi havia famílies que portaven la mutació 4 o 5 generacions sense afectats i altres famílies amb una recurrència molt alta.

% Perquè es tracta amb trasplantament de fetge.


\subsection{Síndrome de resistència generalitzada a les hormones tiroïdals}

% Posar recordatori d'acció de les hormones tiroïdals

$AGGTCA(N)_4AGGTCA$ Binding site per TR. Dímer amb RXR.

El TR té un DBD de zinc fingers. Té dominis de dimerització, d'unió a T3, d'unió a RXR i unió a la maquinària basal de transcripció.

Els humans tenim 2 gens que codifiquen per TR: TRalfa i TRbeta. Els 2 gens donen 4 proteïnes. alfa1 és un TR, alfa 2 no és un TR i beta1 i beta2 sí que són TR.

% completar amb apunts ENDO

Els pacients que tenen resistència a les hormones tiroïdals tenen mutacions al TRbeta1 que les fan dominants negatives (les proteïnes no actuen com a receptors). La mutació mai afecta el DBD del receptor. Les mutacions solen afectar el domini de transactivació (inducció de transcripció), i a vegades també la unió a la hormona. Mai afecten el domini de dimerització. Els receptors mutats ocupen els llocs TRE impedint que el receptor WT faci la seva funció (explicació molecular de la dominància negativa).

La dominància negativa es sol trobar en proteïnes que interaccionen amb altres elements (DNA, proteïnes, hormones).

\subsection{Osteogènesi imperfecta}
Provoca hiperfragilitat òssia. Formes dominants negatives. Penetrància i expressivitat variable.

% Recordatori colàgen

El colàgen està format per una tripe hèlix. La formació d'aquesta hèlix és espontània degut a aminoàcids concrets com la prolina hidroxilada.

Les mutacions dominants negatives fan que l'estructura del colàgen resulti diferent que la natural.

Si la mutació no fos dominant negativa sinó nul·lla, no s'observa cap fenotip.

Les mutacions dominants negatives interfereixen amb la funció normal, de manera que tenen un fenotip més greu.

Penetrància: Percentatge de descendència amb la malaltia.

\subsection{Cribatge neonatal}
Programa de prevenció secundària... Depèn de l'Hospital Clínic. Cribatge no és sinònim de diagnòstic. El cribatge suposa la detecció de la malaltia en els nadons. Comença l'any 1970 amb la fenilcetonúria a la Diputació de Barcelona.

20 malalties es fan en espectrometria de masses. Es busquen metabòlits en altes concentracions.

L'hipotiroïdisme congènit i la fibrosi quística es fa per fluoroimmunoanàlisi.

La cobertura és pràcticament del 100\%.

Totes aquestes malalties tenen com a punt crític, un tractament precoç per evitar danys majors.

\subsubsection{Fenilcetonúria}
És una deficiència autosòmica recessiva de fenilalanina hidroxilasa, que converteix la Phe en Tyr. Això provoca l'acumulació de Phe a la sang. Causa retard mental, de desenvolupament...

La mutació sol ser al gen de l'enzim i hi ha molta heterogeneïtat genètica. Moltes mutacions de pèrdua de funció (alteració d'splicng, stop). Altres vegades la mutació està en alguns enzims de la via del reciclatge de la tetrahidrobiopterina i les conseqüències funcionals són les mateixes que les de les mutacions a la fenilalanina hidroxilasa. En les mutacions en els enzims de les vies de reciclatge de la fenilalanina hidroxilasa, el tractament és la suplementació amb tetrahidrobiopterina.

\subsubsection{Hipotiroïdisme}
Activitat disminuïda de les hormones tiroïdals. Com que les hormones tiroïdals tenen un paper crític al desenvolupament neural dels nens, s'administren hormones tiroïdals als neonats.

L'hipotiroïdisme congènit central acostuma a ser autosòmic recessiu.

L'espectrometria de masses permet expandir el cribatge neonatal.

Síndrome del xarop d'auró.

El que es fa és una dieta dificient en l'aminoàcid que causi la malaltia.

Les immunodeficiències combinades greus.

El sistema de detcció ha ser ser barat i ràid.

Immunodeficiència immunitàri greu:
Les immunodeficiències primàries (IDP) són un grup de malalties causades per l’alteració quantitativa i/o funcional dels diferents mecanismes implicats en la resposta immunitària, les quals inclouen més de 300 defectes congènits.

Es classifiquen en 9 grans grups segons les característiques que tinguin. Dins d’aquestes immunodeficiències, les immunodeficiències combinades greus (SCID) representen entre un 10 i 15\%.

Són les formes més greus d’IDP. Si no tenen un diagnòstic i tractament precoços, duen inexorablement a la mort del pacient abans de l’any de vida. La prevalença al nostre país es calcula (de manera retrospectiva) i és de 1/57.000 nounats vius.

La deficiència en adenosina deaminasa causa la malaltia dels nens bombolla. Mortalitat al voltant del 50\% després de la primera infecció.

Des de l’any 2008, es disposa d’una tècnica de detecció precoç comuna per a les diferents formes genètiques de SCID aplicable a la prova de sang seca de taló: quantificació dels cercles d’excisió del receptor de limfòcits T (T-cell receptor excision circles, TREC) mitjançant PCR quantitativa en temps realeina no invasiva útil per investigar la producció de cèl·lules T pel timus i permet el cribratge universal de les SCID.

Les dades aportades en els diferents estats dels EUA, on s'aplica des de 2010, demostren la utilitat en la detecció d'aquests pacients, amb un percentatge de falsos positius i negatius molt baix.

Per SCID es fa qPCR dels TRECS. Primera prova en fer-se per qPCR. % Buscar TRECS

Criteris per incloure una malaltia en el programa de detecció precoç:
\begin{itemize}
\item La malaltia és un problema de salut important
  
\item Hi ha un tractament acceptable per als pacients
  
\item Facilitats diagnòstiques i de tractament
  
\item Reconeixement en una etapa latent o asimptomàtica
\end{itemize}

\subsection{Cribatge en adults}
Empreses que fan genotipat de polimorfismes per malalties multifactorials (risc d'Alzheimer, cardiovascular).

La malaltia de Tay Sachs és una malaltia autosòmica recessiva que afecta el metabolisme dels gangliòsids. Els gangliòsids són glicolípids importants pel sistema nerviós. Patologia per acumulació. Mutació al gen beta-hexosaminidasa A. La malaltia és molt greu, els nens tenen una esperança de vida curta (3-4 anys). No té tractament.

La taca vermell-cirera s'usava com a marcador per la malaltia abans de desenvolupar un diagnòstic molecular.

La HEXA elimina la N-acetilgalactosamina del GM2.

La mutació altera la pauta de lectura i introdueix un codó stop prematur.

És una malaltia molt rara (1/6400) però hi ha poblacions on és més freqüent com la dels jueus Ashkenazi (1/27). També és molt prevalent al Canadà francòfon i a les poblacions d'origen francès (Cajun). Es fan campanyes de cribatges poblacionals per identificar els portadors de la mutació de Tay Sachs.

Abans del diagnòstic molecular es feia assaig enzimàtic HEXA en sang. Els heterozigots tenien el 50\% de l'activitat enzimàtica.

En Tay Sachs es pot fer diagnòstic prenatal i preimplantacional.

Per les $\beta$-talassèmies també es pot fer cribatge poblacional en adults.