%---------------------------------------------------------------------------------
% Tema 4. Malalties del genoma mitocondrial
%---------------------------------------------------------------------------------
\section{Malalties del genoma mitocondrial}
\label{sec:mitocondrial}

Els conceptes de dominància i recessivitat no s'apliquen exactament aquí.

Les malalties del DNA mitocondrial no es consideren diagnosticades fins que es fa el diagnòstic molecular/genètic.

\subsection{Biologia del DNA mitocondrial}
\label{sec:dnamitocondrial}

El DNA mitocondrial representa l'1\% del DNA cel·lular total en funció del nombre de mitocondris de la cèl·lula.

El DNA mitocondrial és circular de doble cadena i de 16,7 kb. S'assumeix que tots els mitocondris tenen DNA mitocondrial.S'assumeix que hi ha poques còpies del DNA mitocondrial, entre 3 i 4 per mitocondri i entre 1000 i 10000 còpies per cèl·lula.. Els gens del genoma mitocondrial no tenen introns i només una regió no codificant que té els elements reguladors comuns. Hi ha més còpies de gens mitocondrials que de gens nuclears.

Conté 13 mRNA, 22 tRNA i 2 rRNA.

Els 13 mRNA codifiquen per 13 proteïnes components de la cadena respiratòria i fosforilació oxidativa. % Mecanisme de cadena respiratòria i fosforilació oxidativa

Es reparteixen així:
\begin{itemize}
\item Complex I (NADH deshidrogenasa): 7 subunitats
\item Complex III: 1 subunitat
\item Complex IV: 3 subunitats
\item Complex V (ATP sintasa): 2 subunitats. Sense el gradient de protons, aquest enzim és una ATPasa.
\end{itemize}

El genoma mitocondrial es coordina amb el genoma nuclear per sintetitzar els complexes mitocondrials sincrònicament. El mitocondri té un sistema de traducció propi. Els ribosomes mitocondrials són més petits. Els rRNA (12S i 18S) components d'aquests ribosomes estan codificats al genoma mitocondrial. Les proteïnes dels ribosomes mitocondrials estan codificats al genoma nuclear. Els tRNA són menys petits i diversos, amb un codi genètic diferent.

El D-loop és una zona no codificant que té els elements reguladors de la replicació i la transcripció. Les 2 cadenes del genoma mitocondrial s'anomenen H i K (\textit{heavy} i \textit{light}). La que té pocs gens és la L i la H conté molts gens.

El DNA mitocondrial es transcriu en bloc, dóna lloc a un mRNA policistrònic que es processa i produeix els RNA específics. Es fa un transcrit que comença al D-loop i després continua per tota la cadena H. La transcripció es pot aturar fins a la seqüència del tRNA de la Leu. La RNA polimerasa és diferent que la del genoma nuclear.

Hi ha 2 opinions sobre la finalització prematura:
\begin{itemize}
\item Factors a l'inici de la transcripció
\item Factors al final de la transcripció
\end{itemize}

La replicació del DNA mitocondrial va desacoblada a la replicació del genoma nuclear. Una cèl·lula proliferativa pot no replicar el DNA mitocondrial i una cèl·lula quiescent pot replicar el DNA mitocondrial. En l'exercici crònic, hi ha una replicació de DNA mitocondrial al múscul esquelètic per reforçar la cadena respiratòria. La replicació del genoma mitocondrial comença al D-loop i es replica només la cadena H fins a una zona reguladora en què comença la cadena L. Es forma un heterotríplex: les 2 cadenes mare i un tros de la cadena nova. La DNA polimerasa és diferent de la nuclear, està codificada al DNA nuclear i es sintetitza al citoplasma.

\subsection{Mutacions al genoma mitocondrial}
\label{sec:mutacions-al-genoma}
La taxa de mutació és més alta que al genoma nuclear. Això s'explicaria perquè el DNA mitocondrial no està estructurat en cromatina. Es considera que el DNA mitocondrial està més exposat a mutàgens. Al mitocondri es generen moltes ROS, que són potents mutàgens. Si la reducció de l'oxigen és incompleta, es formen ions superòxid.

Les mutacions podran ser polimòrfiques o patològiques. La primera seqüència que es va obtenir es fa servir de referència i s'anomena seqüència Cambridge. Un problema important és discriminar la patogenicitat de la mutació.

El DNA mitocondrial no es recombina i és de transmissió materna.

% Hi ha traces de DNA mitocondrial patern en els humans?? Relació amb la ovella Dolly