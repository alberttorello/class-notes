%------------------------------------------------------------------------------
% Tema 7. Malalties dominants lligades al cromosoma X
%------------------------------------------------------------------------------
\section{Malalties dominants lligades al cromosoma X}
\label{sec:malalt-domin-llig}

\subsection{Síndrome de l'X fràgil}
\label{sec:x-fragil}

Els símptomes són:
\begin{itemize}
\item Discapacitat mental amb problemes d'aprenentatge
\item Dèficit d'atenció
% ...
\end{itemize}

L'any 1970 es va descrobrir un marcador citogenètic en X-fràgil.

Es parla de dominància perquè les dones heterozigotes estan afectades. La penetrància no és completa: 80\% en homes i 35\% en dones.

El gap al cromosoma X només es detecta en el 50\% dels afectats homes i 30\% de les afectades.

Les illes CpG són regions de DNA de menys de 1kb que conté molts dinucleòtids CpG desmetilats. Correspon a gens actius. A la regió 5'-UTR del gen FMR1 hi ha una un $(CGG)_{30}$. La mutació que causa X-fràgil es basa en l'expansió d'aquest triplet. Hi ha una situació de pre-mutació, que no és plenament normal però tampoc patològica. 

La illa CpG es metila i el gen es silencia. En la pre-mutació hi ha un augment de la transcripció però no de traducció. La pre-mutació pot donar alguns símptomes.

La tremolor i atàxia cerebel·losa en les homes està causat per pre-mutació en el gen FMR1. 

En aquesta malaltia hi ha anticipació (quan salta de generació, es torna més greu).