%------------------------------------------------------------------------------
% Tema 9. Malalties cromosòmiques
%------------------------------------------------------------------------------

\section{Malalties cromosòmiques}
\label{sec:malalt-crom}

Els humans no tenim cromosomes telocèntrics. La zona stalk codifica per rRNA, i estan repetitis per tots els cromosomes.

L'índex centromèric mesura la lingitud del braç curt respecte el llarg.

L'anueploïdia no és compatible amb la vida, però les cèl·liles són mosaic de l'inividu.

%% Disomia uniparntetals

\subsection{Disomia uniparental}
\label{sec:disomia-uniparental}

El nombre de cromosomes és normal però per un parell concret els 2 cromosomes homòlegs probenen del mateix progenitor.

Les conseqüències dependran de si hi ha reions d'imprintinh o hi ha mutacions causants de patologies. Isodisonimies i anisodisomies.


\subsection{Aberracions estructurals}
\label{sec:aberr-estr}

Duplicació, inversió (pericèntrica i paracèntrica), 

Les duplicacions augmenten la dosi gènica....

Les inversions mantenen la dosi gènica. El problema és si en la inversió es trenca inm gen o no, i es perden els enhancers.

Insercions

Isocromosoma:  com isocromosomes, per localitzar les proteïnes en una preparació de nuclis

Cromosomes derivatius

Translocació robertsoniana: translocació entre 2 cromosomes acrocèntrics. Es perden els braços curts dels cromosomes. No té efecte a llarg termini.

La translocació robertsoniana t(21, 14) és balancejada no té efectes
fenotípics i es transmet a la descendència que poden ser síndrome de
Down.

\subsection{Síndrome de Down}
\label{sec:sindrome-de-down}

Les trisomies són el 0.3\% de tots els naixements i el 25\% dels
avortaments espontanis. És la causa principal del retard mental. Hi ha
3 trisomies compatibles amb la vida:
\begin{enumerate}
\item Trisomia cromosoma 13:  Síndrome de Patau
\item Trisomia cromosoma 18: Síndrome d'Edwards
\item Trisomia cromosoma 21: Síndrome de Down
\end{enumerate}

La variabilitat fenotípica de la síndromde Down pot ser deguda a:
\begin{itemize}
\item 
\item  % posar 18.2
\end{itemize}

La principal causa de la trisomia és una no desjunció dels gàmetes. La
majoria de les no disjuncions són d'origen matern.

El SD es caracteritxa per una combinació d'aneuploïdies parcials del
cromosoma 21.

Hi ha dades que suggereixen que la presència d'un yampó i dels
cotrols.

\subsection{Sídndromia/Tam1 10}
\label{sec:sidndromiatam1-10}


ppt sindromd Té una


La meitat dels malalts d'Edwards no passen la 1ra setmana.


\subsection{Síndrome Paró}
\label{sec:sindrome-paro}


\subsection{Síndromde de Klinefelter}
\label{sec:sindr-de-klin}

Tenen un cariotip 47XXY. En aquest cas no hi ha la disjunció dels
cromosomes sexuals durantr la gametogènesi.

\subsection{Síndrome de Turner}
\label{sec:sindrome-de-turner}

FISH: 

NMD és un mecanisme de degradació de RNA quan hi ha mutacions
non-sense (mutació stop). Aquest mecanisme evita la presència a la
cèl·lula de pèptids truncants que puguin actuar com a dominants
negatius.

Els transcripts amb codó stop prematur es poden generar per:
\begin{itemize}
\item Mutacions: En funció on estigui el codó stop passarà NMD o
  no. Com més a prop estigui el codó stop prematur del 3', menys
  probabilitat hi ha que succeeixi NMD.
\item \textit{Splicing} alternatiu
\item Gens d'immunoglobulines: Recombinació de la cadena lleugera, el
  procés és imprecís i pot generar mutacions sense sentit.
\end{itemize}

Com interpreta la cèl·lula que el codó stop és una terminació
prematura del transcrit? El PTC ha d'estar en qualsevol exó que no
sigui l'últim per interpretar que ha de fer NMD.

NMD necessita la traducció per poder succeir, que s'inhibeix per
cicloheximida, puromicina, etc. També necessita \textit{splicing}.

Cal tenir en compte que l'splicing és un procés nuclear i la traducció
té lloc al citoplasma. Hi ha un complex proteic, el EJC (Exon Junction
Complex) al nucli que s'uneix a les unions entre exons. Al citoplasma,
aquesta unió es ``memoritza''. El ribosoma elimina els EJC a mesura
que llegeix el mRNA. Quan hi ha PTC, el ribsosoma s'atura i encara
quedarà com a mínim 1 EJC unit al mRNA. S'uneixen un seguit de
proteïnes entre el ribosoma i EJC i promou la degradació.

Hi ha 3 factors que actuen en \textit{trans}, UPF1 és una helicasa que
actua quan la traducció detecta en PTC. UPF1 interactua amb UPF2 i
UPF3. UPF2 i UPF3 formen el EJC. Hi ha altres proteïnes com eIF4AIII,
MLN51 i Y14/MAGOH.

Quan el mRNA es sintetitza, hi ha un complex unit a la caputxa
(Cap-Binding Complex, CAP). El CPC és un heterodímer que promou
splicing i que després és substituït per eIF4E.

NMD té lloc en mRNA que tenen units CPC a la caputxa. La possiblitat
d'una primera ronda de traducció al nucli encara no està resolta. No
se sap si NMD és nuclear, citosòlic o té lloc en ambdós llocs.

La NMD té lloc a la primera ronda de traducció. Es recluta la UPF1 al
CBC, UPF1 interacciona amb factors eRF (eRF1, eRF3) que reconeixen el
codó stop i UPF2 i UPF3 està unit al EJC.

% Completar