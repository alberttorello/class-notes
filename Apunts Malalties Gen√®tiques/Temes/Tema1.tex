%------------------------------------------------------------------------------
% Tema 1. Introducció
%------------------------------------------------------------------------------
\section{Introducció}
\subsection{Història}
Garrod el 1905 va descobrir el concepte d'error innat del metabolisme. Va observar un fenomen bioquímic precís que era innata i que es transmetia familiarment.

La primera malaltia que es va descriure és la alcaptonúria. Afecta el metabolisme de la fenilalanina. La via metabòlica de la fenilalanina té un intermediari que és l'àcid homogenístic. Hi ha una mutació en homozigosi de pèrdua de funció a l'enzim després de l'àcid homogentísic. Els pacients acumulen àcid homogentísic i dona coloració negra als fluids biològics.

La malaltia del xarop d'auró presenta alteracions del metabolisme dels aminoàcids ramificats. La orina feia olor de xarop.

L'any 1945 Beadle i Taum associen un gen-un enzim. El 1948 es va trobar el primer defecte enzimàtic degut a una alteració genètica; amb la metahemoglobinèmia. Alteració de l'estat redox del Fe de la hemoglobina. L'enzim reductor de la metahemoglobina és una NADPH deshidrogenasa que manté la hemoglobina reduïda.

\subsection{Mutacions}
Una \textbf{malaltia genètica} afecta als gens i que es transmeten familiarment. Les \textbf{malalties congènites} no tenen perquè ser genètiques. 

\textbf{Mutació:} Alteració permanent i transmissible del material genètic.
  \begin{itemize}
  \item Germinal: Són les mutacions transmissibles a la descendència.
  \item Somàtica
  \end{itemize}

A més les mutacions es poden classificar com:
\begin{enumerate}
\item Genoma: Triploïdies, trisomes...
\item Cromosòmiques: Translocacions, delecions, inversions
\item Mutacions puntuals
\end{enumerate}

Hi ha mutacions que es veuen afavorides per determinats factors. Les mutacions cromosòmiques es veuen afavorides per l'edat de la mare i les puntuals per l'edat del pare. Els gens més grans tenen una taxa de mutació més alta. Els gens que es troben en cromatina oberta són més susceptibles a la mutació.

Segons l'efecte, les mutacions poden ser:
\begin{itemize}
\item Patogènica
\item Polimorfisme: No es pot associar a un fenotip concret. Canvi neutre. Una substitució a la última posició d'un triplet és sinònima i no genera un canvi d'aminoàcid.
\end{itemize}

El grau de polimorfisme dels humans és molt alt (1-2\%). El locus polimòrifc és un locus on un mínim d'un 1 \% de la població no conté l'al·lel més abundant.

Finalment, les malalties genètiques es poden classificar en:
\begin{itemize}
\item Malalties cromosòmiques: Translocacions, aneuploïdies, inversions, delecions. Entre el 0.4 i el 2.5\% dels nens hospitalitzats.

\item Malalties monogèniques o mendelianes: Malalties causades per 1 sol locus. Solen ser malalties rares. Entre el 6-8\% dels nens hospitalitzats occidentals ho fan per malalties monogèniques. Per la població general, la prevalença és de l'1\%.

\item Malalties multifactorials (multigèniques/multifactorials): No se'ls pot atribuir una genètica mendeliana però sí que hi ha un component genètic transmissible. Són canvis genètics en diversos loci que individualment no tenen potencila patogènic. Són molt susceptibles a factors ambientals. Són les malalties més prevalent en la població (hipertensió, hipercolesterolèmia, obesitat). Es genera un genotip de susceptibilitat que s'afavoreix o no per l'ambient (dieta, hàbits esportius...).22-31\% dels nens hospitalitzats. La prevalença és aproximadament del 60\%.
\end{itemize}

Les malalties, però, poden ser heterogènies:
\begin{itemize}
\item \textbf{Heterogeneïtat genètica:} Referit a malaltia, que alhora pot ser:
  \begin{itemize}
  \item Al·lèlica: La població afectada presenta 2 variants diferents al mateix gen. És molt freqüent.
  \item No al·lèlica: 2 pacients afectats per la mateixa malaltia per mutacions en locus diferents. 

La metahemoglobinèmia hereditària pot ser causafa per 2 gens d'alfa globina, 3 gens de beta globina i 5 pel gen de la NADH deshidrgenasa que manté reduït el Fe. Té heterogeneïtat al·lèlica i de locus.

Les mucopolisacaridosis estan causades per dèficits en la degradació de mucopolisacàrids. Aquestes vies de degradació són molt complexes. S'acumulen mucopolisacàrids a la cèl·lula. La mucopolisacaridosi de Hurler és letal La mucopolisacaridosi de Scheie és molt suau, no compromet la viabilitat... Aquestes malalties estan causats per mutacions al gen L-iduronidasa.
  \end{itemize}
\end{itemize}

Un caràcter es pot estudiar en termes de:
\begin{itemize}
\item Penetrància: S'usa en malalties dominants. Proporció d'heterozigots que manifesten qualsevol símptoma de la la mlaltia. Si la malaltia té 100\% de penetrància, vol dir que tots els afectats presenten la mutació.

\item Expressivitat: Variabilitat i intensitat dels símptomes que es manifesten en un individu. individus amb la mateixa mutació poden presentar símptomes diferents. L'expressivitat pot variar amb l'edat.
\end{itemize}
