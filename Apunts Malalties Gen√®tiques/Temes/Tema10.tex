%------------------------------------------------------------------------------
% Tema 10. Imprinting i malalties relacionades
%------------------------------------------------------------------------------
\section{Imprinting i malalties relacionades}
\label{sec:impr-i-malalt}

\subsection{\textit{Imprinting}}
\label{sec:imprinting}

És el mecanisme que fa que una cèl·lula expressi un sol dels al·lels dels seus progenitors i no els dos simultàniament.

En general s'aplica a gens autosòmics, però la inactivació del X en mamífers podria considerar-se una variant d'imprinting genòmic.

Ocorre en alguns gens específics algunes regions cromosòmiques. Consisteix en una marca o empremta diferencial dels al·lels patern i matern. Es dóna en la gametogènes i té com a resultat l'expressió diferencial dels al·lels patern / matern d'aquests gens específics al llarg del desenvolupament i de la vida de l'individu.

Hi ha al voltant de 100 gens improntats descrits. S'estima que poden ser uns 1000. El primer gen identificat en mamífers va ser Igf2r el 1991 En general, la metilació és al·lel-específica.

El genoma que heretem per via paterna i el que vam heretar per via materna NO es comporten de manera totalment equivalent. En embrions de ratolí manipulats, on els genomes són o bé paterns o bé materns no són viables malgrat que són viables.

A vegades té lloc la diploïdia uniparental de manera natural. Un conceptus androgenètic es desenvolupa com una mola hidatiforme - masses de vellositats coriòniques i altres estructures placentàries, sense teixits embrionaris.

Un conceptus ginogenético dóna lloc a un quist dermoideu que generarà teratomes ovàrics que consisteixen en una massa de teixits adults ben diferenciats però totalment desorganitzats, que inclouen os, dents, cartílag, pell, etc ... i que no tenen cap tipus d'estructura extraembrionària.

La disomia uniparental consisteix en si en un conceptus amb cariotip normal (46, XX o bé 46, XY) hi ha una parella de cromosomes d'origen uniparental (o bé 2 còpies maternes, o bé 2 còpies paternes) es pot manifestar un fenotip patològic que diferirà segons l'origen parental del cromosoma implicat.

L'exemple clàssic està en el cromosoma 15:
\begin{itemize}
\item Disomia paterna del cr. 15: Síndrome d'Angelman
\item Disomia materna del cr. 15: Síndrome de Prader-Willi
\end{itemize}

Algunes delecions subcromosòmiques poden causar patologies diferents segons l'origen parental. També ho veiem en el cromosoma 15:
\begin{itemize}
\item \item Deleció de la regió 15q12 paterna: Síndrome de Prader-Willi
\item Deleció de la regió 15q12 materna: Síndrome d'Angelman 
\end{itemize}

\subsection{Síndrome de Beckwith-Wiedemann}
\label{sec:sindrome-de-beckwith}

Alguns caràcters autosòmics s'hereten amb un patró autosòmic dominant però només es manifesten quan s'hereten d'un dels progenitors. És dominant però només s'expressa quan s'hereta per via materna.

Les característiques clíniques són:
\begin{itemize}
\item Sobrecreixement, predisposició a tumors,
malformacions congènites.

\item Macroglosia: mida de la llengua més gran del normal. 

\item Macrosomia fetal: mida del fetus més gran del normal.

\item Hemihiperplasia: sobrecreixement o desenvolupament excessiu de la meitat d'un òrgan específic o de part o de tots els òrgans i parts d'un costat del cos.

\item Plecs o fosetas a les orelles.
\item Defecte en la paret abdominal: onfalocele, hèrnia umbilical, etc.
\end{itemize}

En aquest cas, els gens es troben a 11p15.5 en 2 regions separades. Al cromosoma matern no s'expressa IGF2. En aquesta regió hi ha enhancer que té acció fins a un possible insulator. L'acció de l'enhancer arriba a H19 però l'insulator impedeix que es transcrigui IGF2. En canvi, al cromosoma patern CTCF no es pot unir a l'insulator i l'acció de l'enhancer permet l'expressió d'IGF2 i que no s'expressi H19.

La metilació a OT1 fa que funcioni KCNQ1.

% mirar el paper i completar

En la gametogènesi, es borren les marques d'imprinting i es tornen a dipositar.