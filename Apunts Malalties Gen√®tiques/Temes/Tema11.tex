%------------------------------------------------------------------------------
% Tema 11. Cerca d'un gen per exoma. Opitz C
%------------------------------------------------------------------------------
\section{Cerca d'un gen per exoma. Opitz C}
\label{sec:cerca-dun-gen}

Un anàlisi per WES d'una pacient va donar una mutació en heterozigosi p.Q638 a MAGEL2. La mutació es manifestava com a dominant.

Es van publicar diverses mutacions a MAGEL2 amb fenotip similar a Prader-Willi però no com a Opitz C.

MAGEL2 es troba a la regió associada amb Prader-Willi, i té expressió paterna (l'al·lel matern està silenciat).

Per determinar la fase: van digerir el genoma amb un enzim sensible a metilació.

En alguns casos, el pare té la mutació però heratat de la mare; llavors no té cap fenotip.

MAGEL2 s'uneix i activa a E3 RING Ubq lligases. Implicat en SUMOilació i transport retrògrad (de l'endosoma a Golgi).

També s'han trobat mutacions a TRAF7 (transducció de senyal, E3 Ubq lligasa).

Una altra mutació en FOXP1, una mutació \textit{de novo} wt/c.1428+1G>A a una regió donadora d'splicing.