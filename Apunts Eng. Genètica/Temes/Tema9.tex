%--------------------------------------------------------
% Tema 9. Expressió heteròloga de gens i sistemes de mutagènesi dirigida
%--------------------------------------------------------
\section{Expressió heteròloga de gens i sistemes de mutagènesi dirigida}
\label{sec:expr-heter-de}

\subsection{Proteïnes de fusió}
\label{sec:proteines-de-fusio}

La GST es situa entre el promotor i el MCS. La GST (que és bacteriana) evita que el bacteri degradi la proteïna que es vol expressar i permet purificar-la de manera ràpida amb una cromatografia d'afinitat amb glutatió. La GST és una proteïna intracel·lular. El MCS té una diana per trombina, que permetrà separar la GST de la proteïna expressada.

El sistema His6 consisteix en l'addició d'una cua d'Hys upstream del MCS. Les proteïnes de fusió es poden purificar amb columnes de níquel.

Per expressar proteïnes de secreció, s'ha d'afegir un pèptid senyal. Els bacteris tenen 3 sistemes de secreció: directament, per reconeixement del pèptid senyal a través del complex pullulanasa.

En el cas de pEZZ, utilitza dominis del tipus immunoglobulina. Es purifica a través del lacZ.

\subsection{Problemes amb l'expressió a \textit{E. coli}}
\label{sec:probl-amb-lexpr}

Hi pot haver:
\begin{itemize}
\item \textbf{Poca quantitat de RNA}
  \begin{itemize}
  \item Baix nivell de transcripció: provar un canvi de promotor
  \item Manca d'estabilitat del transcrit: allargar cua de poliA
  \end{itemize}

\item \textbf{Problemes amb la traducció}
  \begin{itemize}
  \item Bloqueig del ribosome binding site degut a la formació d'estructures secundàries.
  \item Triplets anteriors a AUG: si són codons stop o UAU, CUU) no es tradueix.
  \item Triplets posteriors: seqüències riques en A i U provoquen que el ribosoma ho interpreti com un terminador.
  \item Presència d'aminoàcids rars: aminoàcids que els bacteris utilitzen poc. Es pot solucionar afegint aquests aminoàcids al medi de cultiu.
  \item Ús esbiaixat de codons.
  \end{itemize}

\item \textbf{Problemes amb la proteïna}
  \begin{itemize}
  \item Inestabilitat de la proteïna: ús de soques deficients en proteases.
  \item Plegaments incorrectes: es pot solucionar co-transformant amb xaperones o utilitzant cèl·lules eucariotes.
  \item Manca de processament post-traduccional: ús de cèl·lules eucariotes.
  \item Formació de cossos d'inclusió: es pot solucionar modificant la temperatura, el temps d'inducció, el medi de cultiu o fer servir una altra soca d'\textit{E. coli}. També es pot incloure tags de fusió de secreció o delecionar els extrems amino i carboxi de la proteïna.
  \end{itemize}
\end{itemize}

LLEVATS:
Inclou un gen de selecció Leu.