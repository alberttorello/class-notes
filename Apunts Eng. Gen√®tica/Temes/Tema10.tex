%--------------------------------------------------------
% Tema 10. Modificació genètica d'organismes
%--------------------------------------------------------
\section{Modificació genètica d'organismes}
\label{sec:modif-genet-dorg}

\subsection{Mètodes d'obtenció d'animals transgènics}
\label{sec:metod-dobt-danim}

Un organisme transgènic és aquell al que se li ha introduït una
informació genètica exògena. Aquesta informació es troba en un
fragment de DNA que anomenem transgen.
 
Per generar un organisme transgènic, el transgen s'introdueix i
s'integra en el genoma d'una cèl·lula capaç de generar posteriorment
un organisme. Línia germinal o ESC derivades de blastocist.

Hi ha diferents mètodes per a introduir un transgen a diferents tipus
cel·lulars, i el transgen es pot integrar de diferents maneres en el
genoma:
\begin{itemize}
\item Microinjecció de DNA
\item Transducció retroviral: Es poden infectar cèl·lules ES o
  directament el blastocist.
\item Transfecció: Es pot fer en qualsevol cultiu.
\item Substitució de nucli: Introducció d'un nucli somàtic en un òvul
  anucleat.
\end{itemize}

\subsubsection{Microinjecció en zigots}
\label{sec:micr-en-igots}

El que interessa és que el DNA s'integri al zigot perquè el DNA es
mantingui en les successives divisions cel·lulars. És un mètode molt
eficient.

\subsubsection{Vectors lentivirals}
\label{sec:vectors-lentivirals}

En cèl·lules empaquetadores es transfecten les proteïnes del virus i
el gen d'interès amb un promotor determinat. La cèl·lula empaquetadora
fa un virus infectiu però no replicatiu.

És una tècnica poc invasiva, molt eficient (80-100\%) i pot infectar
cèl·lules no proliferants.

D'altra banda, hi ha una limitació en quant a la mida del DNA a
introduir. En funció del moment d'integració pot generar
mosaïcisme. Els virus costen d'obtenir i pot haver-hi fenòmens de
silenciament del transgen.

\subsubsection{Transfecció de cèl·lules ES }
\label{sec:transf-de-cel.l}

Són cèl·lules de la ICM del blastocist. Es transfecten per
electroporació o precipitació amb fosfat càlcic i es seleccionen amb
un gen selector.

Aquestes cèl·lules s'injecten en un altre blastocist i es generen
ratolins quimera.

El ratolí quimera es pot creuar amb un WT i després fer un creuament
de la descendència heterozigota per obtenir homozigots.

\subsubsection{Transferència nuclear}
\label{sec:transf-nucl}

Es transfecten cèl·lules diferenciades, s'extreu el nucli i s'injecten
en oòcits anucleats.

\subsection{Animals transgènics per integració aleatòria}
\label{sec:anim-transg-per}

El DNA microinjectat pot integrar-se en punts de trencament de la
doble cadena per inserció aleatòria. La inserció pot tenir lloc en
gens, fet que pot silenciar-los.

Pot passar que s'hagi inserit en una localització sota la influència
d'un altre promotor. També es pot inserir en un promotor que el
silenciï. Es pot incorporar en una zona d'heterocromatina o
alteracions epigenètiques. Cal tenir en compte que quan s'insereixen
diverses còpies d'un gen, aquests es silencien.

La integració aleatòria d'un gen informador (reporter gene) revela el
potencial de regulació dels elements del genoma. El gen es clona sota
el control d'un promotor mínim. Aquest sistema permet estudiar els
enhancers i silencers específics de teixit.

També es poden expressar gens ectòpics o bé sobreexpressar-los.

És possible abolir parcialment la expressió d'un gen utilitzant petits
oligonucleòtids que promouen la degradació del mRNA de un gen diana
(RNA d'interferència): knockdowns.

La reducció de la expressió del gen diana pot ser transitòria o permanent.

En la reducció transitòria, els organismes són tractats amb dsRNAs
(\textit{double stranded RNA}) o siRNAs (\textit{small interference
  RNA}).

En la reducció permanent, es generen animals transgènics que expressen
shRNAs (\textit{shorthairpin RNA}).

Al genoma hi ha seqüències amb un promotor, una ORF i una repetició
invertida. Això genera RNA de doble cadena.

% Diferència entre siRNA, shRNA, miRNA

\subsection{Animals transgènics per integració dirigida}
\label{sec:anim-transg-per-1}
La integració dirigida pot ser per:
\begin{itemize}
\item Recombinació homòloga: El DNA que es vol inserir conté uns
  braços d'homologia amb la regió on es vol inserir.
\item Nucleases que reconeixen seqüències específiques
\end{itemize}

Mitjançant el \textit{gene targeting} es poden generar animals:
\begin{itemize}
\item \textit{knock-out}: S'aboleix la funció d'un gen
\item \textit{knock-in}: S'introdueix un canvi específic en el gen
  d'interès que pot suposar la pèrdua total o parcial de la funció del gen.
\end{itemize}

S'usen 2 selectors: 1 dins i 1 fora. Si hi ha HR el GS1 queda integrat
i el GS2 queda fora. Les cèl·lules seran resistents a 1 però no a 2.

Si la integració és aleatòria, les cèl·lules són resistents a 1 i 2.

% Exemple

\subsubsection{Nucleases específiques}
\label{sec:nucl-espec}

% Copiar el que tinc de teràpia.


\subsection{Regulació de l'activació i el silenciament del transgen}
\label{sec:regul-de-lact}

Es poden regular l'expressió o el silenciament del transgen en funció
del temps i l'espai fent servir:
\begin{itemize}
\item Transactivadors transcripcionals (activen la transcripció en
  trans)
  \begin{itemize}
  \item Sistema Gal4-UAS
  \item Sistema TetR-tetO
  \end{itemize}
\item Recombinació específica de seqüència
  \begin{itemize}
  \item Sistema Cre-loxP
  \item Sistema Flp-FRT
  \end{itemize}
\end{itemize}

Són sistemes binaris basats en la interacció de dos components
(transgens), inoperatius per separat però amb efecte quan es troben en
un mateix animal: l'efector i el transgen diana.

\subsubsection{Sistema GAL4-UAS}
\label{sec:sistema-gal4-uas}
Es creen 2 soques de Drosophila: Promotor específic de teixit-GAL4 i
una altra UAS-gen X. Quan es creuen les mosques, una fracció de la
descendència heredarà les 2 construccions. En aquests individus,
s'expressarà GAL4 al teixit que toqui, i activarà la transcripció del
gen X en aquell teixit específicament.

Hi ha variacions d'aquest sistema. 
\begin{itemize}
\item Sensible a hormones: El sistema només s'activa en presència
  d'una hormona concreta.
\item Sensible a temperatura
\end{itemize}

\subsubsection{Sistema Cre-loxP}
\label{sec:sistema-cre-loxp}

% Copiar de teràpia.

Una soca de ratolins expressa Cre i una altra soca expressa el target
gene flanquejat per loxP. Quan es creuen les 2 soques, s'elimina el
target gene i es genera un KO.

Un exemple: Construcció promotor específic de teixit-Cre-LBD ER
(domini d'unió a estrògens). Quan s'afegeix tamoxifen, s'activa la
Cre. Aquí hi ha control en l'espai i en el temps.