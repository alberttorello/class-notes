%--------------------------------------------------------
% Tema 8. Caracterització de l'expressió gènica: proteïnes
%--------------------------------------------------------

\section{Caracterització de l'expressió gènica: proteïnes}
\label{sec:caract-de-lexpr}

\subsection{Purificació de proteïnes}
\label{sec:purif-de-prot}

\subsubsection{Extracció de proteïnes}
\label{sec:extr-de-prot}


\subsubsection{Cromatografia}
\label{sec:cromatografia}

\paragraph{Cromatografia d'exclusió molecular} \hfill \\
La fase estacionària està formada per partícules polimèriques de diferent porositat. La separació es basa en la diferent grandària de les partícules (massa molecular).

Les proteïnes més grans no poden passar pels porus de la matriu de filtració i són eluïdes amb major rapidesa que les de baix pes molecular que penetren pels porus de la matriu i recorren un camí molt més tortuós i més llarg.

\paragraph{Cromatografia de bescanvi iònic} \hfill \\
 La fase estacionària està composta per partícules de polímers amb càrrega elèctrica positiva (bescanvi aniònic) o càrrega elèctrica negativa (bescanvi catiònic).

Les proteïnes queden immobilitzades degut a la seva càrrega elèctrica. L'elució es realitza augmentant progressivament la força iònica de la fase mòbil (eluent fins que s'assoleix el pI de la proteïna d'interès.

\paragraph{Cromatografia d'afinitat} \hfill \\
La fase estacionària està composta per partícules de polímers unides covalentment al lligand (anticòs, glutatió, proteïna A, etc...). La proteïna problema queda immobilitzada per unió al lligand.
L'elució de la proteïna a aïllar, augmentant la força iònica de la fase mòbil.

P.e si es volen purificar Ig, la fase mòbil es recobreix amb proteïna A. Les Ig s'uneixen inespecíficament a la proteïna A. Després dels rentats, es pot eluir.



\subsection{Electroforesi i isoelectroenfoc}
\label{sec:electr-i-isoel}
S'han de separar les proteïnes per mida i no per càrrega elèctrica. Abans de fer l'electroforesi s'han de desnaturalitzar les proteïnes. La matriu és de poliacrilamida.

Els tractaments previs són:
\begin{itemize}
\item Ebullició de les proteïnes (100ºC)
\item $\beta$-mercaptoetanol o DTT per desfer els ponts disulfur per reducció
\item SDS, un detergent aniònic que confereix càrrega negativa a les proteïnes. S'addiciona a la solució de proteïnes i al gel.
\end{itemize}

Hi ha procediments per tenyir les proteïnes de manera inespecífica, com el blau de Coomasie.

\subsubsection{Electroforesi bidimensional}
\label{sec:electr-bidim}
S'agafen les proteïnes directament i es posen en un gel d'isoelectroenfoc. El gel d'isoelectroenfoc conté amfolines, molècules que en un camp elèctric s'ordenen segons el pI (2,5-11). Les proteïnes migren per aquest gel d'isoelectroenfoc fins que perden la càrrega. Llavors es bullen les proteïnes, SDS i mercaptoetanol i es fa un SDS-PAGE.

Permet conèixer el grau de fosforilació, glicosilació.

Per identificar la proteïna, es retalla la taca.

\subsection{Proteòmica}
\label{sec:proteomica}
La proteïna es digereix parcialment amb tripsina i s'obtenen pèptids. Els pèptids es passen per un espectròmetre de masses. Es calcula la ratio massa/càrrega. Aquest perfil es compara amb una base de dades i així es pot conèixer la seqüència del pèptid, i després es pot saber a quina proteïna pertanyen.

\subsection{Western Blot}
\label{sec:western-blot}


\subsection{Dot Blot}
\label{sec:dot-blot}


\subsection{ELISA}
\label{sec:elisa}


\subsection{Immunodetecció \textit{in situ}}
\label{sec:immunodeteccio-situ}


\subsection{Microxips}
\label{sec:microxips}

