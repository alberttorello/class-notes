%--------------------------------------------------------
% Tema 1. Preparació i anàlisi dels àcids nucleics
%--------------------------------------------------------
\section{Preparació i anàlisi dels àcids nucleics}

\subsection{Introducció}
\begin{itemize}
\item Mitjans del segle XIX Mendel va establir les lleis de l'herència biològica.
\item Principis del segle XX: Morgan i Sutton estableixen la teoria cromosòmica de l'herència i que els gens estan als cromosomes.
\item 1944: Avery, MacLeod, MacCarty estableixen que el material genètic és ADN.
\item El 1952-66 es va establir l'estructura del DNA, el codi genètic i elucidar els processos de transcripció i traducció.
\item El 1972-73 es comencen a fer servir tècniques de DNA recombinant, produint-se així el naixement de l'enginyeria genètica.
\item El 1985 es va introduir la tècnica de PCR.
\item 1990: Seqüenciació de genomes
\end{itemize}

\subsubsection{Clonatge}
L'enginyeria genètica és un conjunt de mètodes que permeten el clonatge d'un fragment de DNA d'interès i introduir-lo a un altre organisme.

El clonatge de DNA consisteix en:
\begin{enumerate}
\item Construcció de la molècula de DNA recombinant: L'insert s'introdueix en un vector. El vector conté elements que permetin la replicació i expressió d'aquest DNA. El conjunt del vector i l'insert lligats constitueixen la molècula de DNA recombinant.
\item Introducció a la cèl·lula hoste
\item Multiplicació del DNA recombinant
\item Propagació del DNA recombinant
\item Obtenció dels clons
\end{enumerate}

En el procés de clonatge:
\begin{enumerate}
\item El primer que cal tenir clar és el procés biològic que es vol estudiar.
\item Identificació del gen a estudiar. S'extreu i es purifica el DNA de l'organisme d'interès. En el cas dels eucariotes, molt sovint s'utilitza RNA i mitjançant la retrotranscripció obtenir la seqüència de cDNA.
\item Fragmentar el DNA
\item Elecció del vector: Si el volem propagar en procariotes o eucariotes, en funció de la mida de l'insert...
\item S'ha d'obrir el vector per introduir l'insert.
\item Fusió del DNA recombinant.
\item Introducció en bacteris i producció de clons.
\item Seleccionar el clon que conté el gen d'interès.
\end{enumerate}

\subsubsection{Llibreries}
Les llibreries genòmiques es generen mitjançant la fragmentació del DNA de l'organisme de l'interès i es clonen en un vector d'interès i s'introdueix en un organisme senzill per generar clons de manera que cada clon contingui una molècula de DNA recombinant amb un insert diferent.

És necessari saber quants clons es necessiten per tenir representat un genoma. Dependrà de la mida del genoma d'interès i de la capacitat del vector (mida de l'insert que és capaç de lligar).

L'altre possibilitat és generar una llibreria de cDNA. Estan construïdes a partir de mRNA obtingut de les cèl·lules o teixits d'estudi, que s'han retrotranscrit a cDNA i lligat en un vector de clonatge. Cada clon presenta una còpia del mRNA obtingut.

\subsubsection{Aplicacions}
L'enginyeria genètica pretén produir una proteïna en concret per múltiples finalitats.

\begin{itemize}
\item Biofarmacèutic: Producció d'antibiòtics, proteïnes heteròlogues, generació de proteïnes amb noves funcions (més solubilitat, proteïnes quimera), generació de vacunes.
\item Agrícola i ramader: modificació genètica de plantes, animals transgèncics.
\item Clínic: Diagnòstic, Teràpia gènica
\item Forense
\end{itemize}

\subsection{Purificació del DNA total procariota}
El material genètic a purificar pot ser:
\begin{itemize}
\item DNA genòmic procariota
\item DNA genòmic eucariota
\item RNA, per generar llibreries de cDNA
\item DNA plasmídic
\item DNA de bacteriòfags 
\end{itemize}

\subsubsection{Etapes en la preparació del DNA total procariota}
Les etapes més generals són:
\begin{enumerate}
\item Creixement i concentració de les cèl·lules
\item Ruptura de les cèl·lules
\item Purificació del DNA
\item Concentració del DNA
\end{enumerate}

\paragraph{Creixement i concentració del culiu bacterià} \hfill \\
Els bacteris creixen en un medi ric.

L'estimació del creixment es fa per la lectura de la densitat
òptica a 600 nm. En la fase exponencial hi ha una relació directa entre DO i
nombre de cèl·lules; sempre i quan siguin el mateix medi de cultiu,
mateixa quantitat d'inòcul i temperatura de cultiu.

Després d'un temps, el cultiu es concentra mitjançant centrifugació a
velocitats baixes per evitar trencaments cel·lulars.

\paragraph{Ruptura de les cèl·lules} \hfill \\
Els mètodes físics (sonicació, pressió) tenen el problema que poden
trencar el DNA. La millor opció és el tractament químic que trenquin
la membrana i alliberin el contingut. Un dels components que
s'utilitza és el lisozim que actua sobre la paret de
mureïna. S'afegeix EDTA (tetraacetat d'etilendiamida) per quelar el Mg i desestabilitzar les
membranes; a més inhibeix molts enzims bacterians (nucleases,
polimerases). També s'addiciona SDS, un detergent per disgregar els
lípids de la membrana.

Es torna a centrifugar per eliminar els residus més grans. El pellet
contindrà bacteris no lisats i complexos grans. El sobrenedant té DNA,
RNA i proteïnes de baix pes molecular.

\paragraph{Purificació del DNA} \hfill \\
Per eliminar el RNA, s'introdueix una RNasa. Després, s'eliminen les
proteïnes mitjançant proteases inespecífiques com la proteïnasa K o la
pronasa. El tractament amb fenol dissol les proteïnes hidrofòbiques;
s'afegix 1 volum de fenol i es
mescla per inversió i es centrifuga i sortiran dues fases:
\begin{itemize}
\item A la part superior hi ha la part polar amb el DNA.
\item Interfase: Proteïnes
\item A la part inferior hi ha el fenol.
\end{itemize}

A la fase aquosa hi poden quedar traces de fenol, que inhibeix la
majoria d'enzims. Llavors s'afegeix fenol amb cloroform equilibrat
(25:24:1 fenol cloroform isoamilalcohol), s'agita per inversió, es
centrifuga. El cloroform arrossega el fenol al fons del tub.

Es recull la fase aquosa i s'afegeix cloroform, s'agita per inversió i es centrifuga un altre
cop. S'agafa la fase aquosa on hi ha el DNA pur.

\paragraph{Precipitació del DNA} \hfill \\
Hi ha 2 mètodes:
\begin{itemize}
\item Si hi ha molt volum; la precipitació es fa per
  sals+etanol. S'afegeix 2,5 volums d'etanol absolut fred a -20 ºC i
  s'introdueix una vareta de vidre i es remena tot rotant la
  vareta. Quan el DNA s'uneix a la vareta, s'introdueix en un tub amb
  buffer TE i es gira la vareta en sentit contrari.

\item Si el volum és petit, s'afegeix 1/10 de NaAc i 2,5 volums
  d'etanol fred a -20 ºC \textit{overnight}. Després es centrifuga, és
  possible que el pellet no sigui visible i es descarta el sobrenedant
  i es fa un rentat amb EtOH 70 \%, amb una altra centrifugació. El
  pellet es seca a temperatura ambient i el pellet es resuspèn en TE/
  aigua pura.
\end{itemize}

Algunes soques bacterianes presenten, p.e, una càpsula molt gruixuda o
molt LPS i no és pràctic fer una extracció fenol:cloroform:isoamil
alcohol.

Les cèl·lules vegetals presenten parets de cel·lulosa o xilosa, pel
que s'utilitzen xilanasa o cel·lulasa.

El \textbf{mètode CTAB} (bromur de cetil-trimetil-amoni) agafa l'extracte
bacterià i aplica la RNasa. Després afegeix el CTAB, que forma
complexos amb el DNA i és fàcil separar-los per centrifugació. Es
descarta el sobrenedant i es recupera el pellet en un buffer amb
NaCl. S'afegeixen 2 volums d'etanol i es centrifuga, el pellet es
renta amb EtOH 70 \%.

Molts kits comercials es basen en la \textbf{cromatografia
  d'intercanvi iònic}. Un cop s'han lisat les cèl·lules i s'obté
l'extracte cel·lular es tracta amb RNasa. S'aplica aquesta suspensió
en unes columnes de cromatografia amb resines de càrregues
positives. El DNA s'associa  a la resina i hi queda unit. Es fa un
rentat de la resina per eliminar unions inespecífiques. Finalment es
fa l'elució amb un tampó que desestabilitza la unió del DNA-resina.

\subsection{Purificació del DNA plasmídic}