%--------------------------------------------------------
% Tema 1. Preparació i anàlisi dels àcids nucleics
%--------------------------------------------------------
\section{Preparació i anàlisi dels àcids nucleics}

\subsection{Introducció}
\begin{itemize}
\item Mitjans del segle XIX Mendel va establir les lleis de l'herència biològica.
\item Principis del segle XX: Morgan i Sutton estableixen la teoria cromosòmica de l'herència i que els gens estan als cromosomes.
\item 1944: Avery, MacLeod, MacCarty estableixen que el material genètic és ADN.
\item El 1952-66 es va establir l'estructura del DNA, el codi genètic i elucidar els processos de transcripció i traducció.
\item El 1972-73 es comencen a fer servir tècniques de DNA recombinant, produint-se així el naixement de l'enginyeria genètica.
\item El 1985 es va introduir la tècnica de PCR.
\item 1990: Seqüenciació de genomes
\end{itemize}

\subsubsection{Clonatge}
L'enginyeria genètica és un conjunt de mètodes que permeten el clonatge d'un fragment de DNA d'interès i introduir-lo a un altre organisme.

El clonatge de DNA consisteix en:
\begin{enumerate}
\item Construcció de la molècula de DNA recombinant: L'insert s'introdueix en un vector. El vector conté elements que permetin la replicació i expressió d'aquest DNA. El conjunt del vector i l'insert lligats constitueixen la molècula de DNA recombinant.
\item Introducció a la cèl·lula hoste
\item Multiplicació del DNA recombinant
\item Propagació del DNA recombinant
\item Obtenció dels clons
\end{enumerate}

En el procés de clonatge:
\begin{enumerate}
\item El primer que cal tenir clar és el procés biològic que es vol estudiar.
\item Identificació del gen a estudiar. S'extreu i es purifica el DNA de l'organisme d'interès. En el cas dels eucariotes, molt sovint s'utilitza RNA i mitjançant la retrotranscripció obtenir la seqüència de cDNA.
\item Fragmentar el DNA
\item Elecció del vector: Si el volem propagar en procariotes o eucariotes, en funció de la mida de l'insert...
\item S'ha d'obrir el vector per introduir l'insert.
\item Fusió del DNA recombinant.
\item Introducció en bacteris i producció de clons.
\item Seleccionar el clon que conté el gen d'interès.
\end{enumerate}

\subsubsection{Llibreries}
Les llibreries genòmiques es generen mitjançant la fragmentació del DNA de l'organisme de l'interès i es clonen en un vector d'interès i s'introdueix en un organisme senzill per generar clons de manera que cada clon contingui una molècula de DNA recombinant amb un insert diferent.

És necessari saber quants clons es necessiten per tenir representat un genoma. Dependrà de la mida del genoma d'interès i de la capacitat del vector (mida de l'insert que és capaç de lligar).

L'altre possibilitat és generar una llibreria de cDNA. Estan construïdes a partir de mRNA obtingut de les cèl·lules o teixits d'estudi, que s'han retrotranscrit a cDNA i lligat en un vector de clonatge. Cada clon presenta una còpia del mRNA obtingut.

\subsubsection{Aplicacions}
L'enginyeria genètica pretén produir una proteïna en concret per múltiples finalitats.

\begin{itemize}
\item Biofarmacèutic: Producció d'antibiòtics, proteïnes heteròlogues, generació de proteïnes amb noves funcions (més solubilitat, proteïnes quimera), generació de vacunes.
\item Agrícola i ramader: modificació genètica de plantes, animals transgèncics.
\item Clínic: Diagnòstic, Teràpia gènica
\item Forense
\end{itemize}

\subsection{Purificació d'àcids nucleics}
El material genètic a purificar pot ser:
\begin{itemize}
\item DNA genòmic procariota
\item DNA genòmic eucariota
\item RNA, per generar llibreries de cDNA
\item DNA plasmídic
\item DNA de bacteriòfags 
\end{itemize}