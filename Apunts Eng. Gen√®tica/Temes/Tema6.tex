%--------------------------------------------------------
% Tema 6. Genoteques
%--------------------------------------------------------

\section{Genoteques}
\label{sec:genoteques}

\subsection{Llibreries gen�miques}
\label{sec:llibr-genom}
Les llibreries gen�miques estan constru�des mitjan�ant inserci� de diferents fragments del genoma de l'organisme, obtinguts per digesti�, en un vector de clonatge apropiat.

El nombre de clons representatius d'un genoma dependran de la mida del genoma i la mida dels fragments generats.

Es pot calcular com:
\begin{equation}
  \label{eq:1}
  N = \frac{ln(1-P)}{ln(1-\dfrac{a}{b})}
\end{equation}

On $N$ �s el n�mero de clons, $P$ �s la probabilitat que qualsevol gen es trobi a la llibreria, $a$ �s la mida dels fragments inserits al vector i $b$ �s la mida total del genoma.

\subsection{Llibreries de cDNA}
\label{sec:llibreries-de-cdna}
S'a�lla el RNA total i es retrotranscriu a cDNA. Els fragments resultants es clonen en un vector apropiat. Les llibreries es poden enriquir per un determinat gen.

\subsection{Estrat�gies de selecci� dels clons recombinants}
\label{sec:estr-de-selecc}
Hi ha sistemes de:
\begin{itemize}
\item \textbf{Detecci� directa:} S'avalua si �s recombinant i si cont� el gen d'inter�s
  \begin{itemize}
  \item �s de medi selectiu: Si es vol clonar un gen de resist�ncia a antibi�tic, es creixen els clons en un medi amb un antibi�tic determinat per la selecci� dels recombinants i un segon antibi�tic per seleccionar els que contenen el gen que estem buscant.
  
  \item Mutants auxotr�fics: Bacteris mutants que no tinguin el gen que estem buscant. Si un bacteri �s deficitari per un enzim biosint�tic, no creixer� en un medi selectiu sense Trp. Si es transforma un bacteri Trp- amb un vector que porta l'enzim, aquest clon creixer� en un medi selectiu per Trp.
  \end{itemize}
\item \textbf{Identificaci� del clon}
  \begin{itemize}
  \item Detecci� de la mol�cula recombinant per hibridaci� de sondes
    \begin{enumerate}
    \item Transfer�ncia dels clons a filtres de nylon o nitrocel�lulosa
    \item Lisi dels clons: Tractament amb proteases i calor per desnaturalitzar el DNA
    \item Fixaci� del DNA amb UV
    \item Hibridaci� del DNA marcat amb sonda
    \item Rentat per eliminar unions inespec�fiques
    \item Revelat
    \end{enumerate}
  \item Detecci� de la prote�na per immunodetecci�
    \begin{enumerate}
    \item Transfer�ncia dels clons a una membrana
    \item Lisi de les col�nies
    \item Incubaci� amb antic�s i amb antic�s secundari, si fos necessari.
    \end{enumerate}
  \end{itemize}
\end{itemize}