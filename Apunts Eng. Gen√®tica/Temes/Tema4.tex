%--------------------------------------------------------
% Tema 4. Clonatge. Cloning PCR
%--------------------------------------------------------
\section{Clonatge. Cloning PCR}
\label{sec:clon-clon-pcr}

La unió d'un insert en un vector genera una molècula de DNA recombinant. L'insert pot provenir de:
\begin{enumerate}
\item Genoma digerit amb endonucleases
\item Fragments de DNA digerits amb endonucleases
\item cDNA
\item Fragments amplificats per PCR o RT-PCR
\end{enumerate}

\subsection{PCR per clonar}
\label{sec:pcr-per-clonar}

En funció de la naturalesa del producte de PCR, es segueixen estratègies diferents:
\begin{enumerate}
\item Extrems roms: Si el producte de PCR té extrems roms, es pot clonar en un vector obert amb extrems roms o transformar els extrems roms de l'insert en cohesius per lligar-lo en un vector amb extrems cohesius.

\item Nucleòtid addicional: La TaqPOL genera un extrem protuberant en 3' amb un A. Llavors es pot clonar en un vector ja preparat amb una T protuberant. L'altra opció és transformar-los en extrems roms amb la nucleasa S1 o la polimerasa T4.

\item Primers amb diana: 
  \begin{itemize}
  \item Diana dins el primer: Es dissenyen primers flanquejants del gen que continguin les dianes de restricció que interessin.
  \item Diana en extensió del primer en 5': El gen no conté la diana de restricció adequada per clonar-lo en el vector, llavors s'introdueixen les dianes per PCR.
  \end{itemize}
\end{enumerate}

\subsubsection*{Problema}


\subsection{Clonatge en vectors}
\label{sec:clonatge-en-vectors}
Quan es clona un gen en un vector, poden passar 3 coses:
\begin{itemize}
\item Vectors auto-lligats
\item Molècula recombinant amb el gen que volem
\item Molècules recombinants errònies
\end{itemize}

\subsubsection{Selecció dels transformants}
\label{sec:selecc-dels-transf}
S'usen vectors que continguin gens de resistència a antibiòtics. Quan els bacteris es cultiven en un medi que conté antibiòtics només creixen els bacteris que han incorporat el vector.

\subsubsection{Transformació}
\label{sec:transformacio}
Hi ha bacteris que tenen transformació natural. El més habitual és induir la transformació mitjançant la generació de cèl·lules competents.

\begin{itemize}
\item Cèl·lules competents amb un tractament amb \ch{CaCl2} en fred, el plàsmid s'uneix a la paret, s'incuba a 42ºC durant 2 minuts.

\item Electroporació: Una descàrrega elèctrica provoca forats a la paret, pel qual entra el plàsmid. Es posen els bacteris en un medi de recuperació.
\end{itemize}
