%--------------------------------------------------------
% Tema 9. Expressió heteròloga de gens i sistemes de mutagènesi dirigida
%--------------------------------------------------------
\section{Expressió heteròloga de gens i sistemes de mutagènesi dirigida}
\label{sec:expr-heter-de}

\subsection{Expressió heteròloga de gens}
\label{sec:expressio-heterologa}
La capacitat de clonar gens suposa que un gen que codifiqui per una
proteïna animal o vegetal d'interès es pot aïllar d'un organisme,
inserir-lo en un vector de clonatge i introduir-lo en un bacteri. Si
les manipulacions es fan correctament el gen s'expressarà i la
proteïna recombinant es sintetitzarà, i així es poden obtenir grans
quantitats de proteïna.

Per obtenir proteïnes recombinants, es necessiten vectors de clontage
especials, i a vegades no s'obté la quantitat de proteïna esperada.

L'expressió heteròloga de gens té moltes aplicacions:
\begin{itemize}
\item \textbf{Recerca bàsica:}
  \begin{itemize}
  \item Estudis funcionals-bioquímics
  \item Estudis d'estructura 3D
  \item Estudi de dominis funcionals
  \item Estudis de localització
  \item Producció d'anticossos
  \item Funció de gens predits en el Projecte Genomes
  \end{itemize}

\item \textbf{Recerca aplicada (biotecnologia):} Producció de proteïnes
  d'interès comercial, com ara fàrmacs o enzims.
\end{itemize}

\subsubsection{Tipus de proteïnes que es poden expressar}
\label{sec:tipus-de-proteines}
Cal tenir en compte, que en funció de la manipulació dels gens que es
dugui a terme, l'organisme d'on prové i l'organisme on es produeix, es
poden obtenir diferents tipus de proteïnes:
\begin{itemize}
\item \textbf{Proteïna \textit{wild-type}, silvestre o salvatge:} Proteïna amb
  la seqüència d'aminoàcids original.

\item \textbf{Proteïna mutant:} Proteïna amb algun canvi d'aminoàcids respecte
  la original.

\item \textbf{Proteïna nativa:} Proteïna que prové de l'organisme original que
  la produeix.

\item \textbf{Proteïna recombinant:} Proteïna produïda amb tècniques
  d'enginyeria genètica.

\item \textbf{Proteïna de fusió o quimèrica:} Proteïna formada per la fusió de
  2 gens diferents o per 2 fragments de gens (amb tècniques
  d'enginyeria genètica). 
\end{itemize}

\subsubsection{Sistemes d'expressió}
\label{sec:sistemes-dexpressio}
A l'hora d'escollir el sistema d'expressió s'han de tenir en compte
diferents factors:
\begin{multicols}{2}
\begin{enumerate}
\item En funció de la classe de la proteïna
\item Informació ja existent en la literatura
\item Cost
\item Rendiment
\item Plegament de la proteïna
\item Modificacions post-traduccionals
\end{enumerate}
\end{multicols}

En base a aquestes consideracions, els principals sistemes d'expressió
són:
\begin{itemize}
\item Sistemes d'expressió \textit{in vitro}: Resultat molt ràpid.
\item Sistemes bacterians: Fàcil i econòmic.
\item Llevats: Fàcil i més econòmic que un sistema de mamífer.
\item Insectes: Ideal per una qualitat moderada-alta
\item Mamífers: Producció de proteïnes natives i actives
\end{itemize}

\subsubsection{Sistemes d'expressió lliures de cèl·lules}
\label{sec:sist-dexpr-lliur}
 


\subsubsection{Proteïnes de fusió}
\label{sec:proteines-de-fusio}

La GST es situa entre el promotor i el MCS. La GST (que és bacteriana)
evita que el bacteri degradi la proteïna que es vol expressar i permet
purificar-la de manera ràpida amb una cromatografia d'afinitat amb
glutatió. La GST és una proteïna intracel·lular. El MCS té una diana
per trombina, que permetrà separar la GST de la proteïna expressada.

El sistema His6 consisteix en l'addició d'una cua d'Hys upstream del
MCS. Les proteïnes de fusió es poden purificar amb columnes de níquel.

Per expressar proteïnes de secreció, s'ha d'afegir un pèptid
senyal. Els bacteris tenen 3 sistemes de secreció: directament, per
reconeixement del pèptid senyal a través del complex pullulanasa.

En el cas de pEZZ, utilitza dominis del tipus immunoglobulina. Es
purifica a través del lacZ.

\subsubsection{Problemes amb l'expressió a \textit{E. coli}}
\label{sec:probl-amb-lexpr}

Hi pot haver:
\begin{itemize}
\item \textbf{Poca quantitat de RNA}
  \begin{itemize}
  \item Baix nivell de transcripció: provar un canvi de promotor
  \item Manca d'estabilitat del transcrit: allargar cua de poliA
  \end{itemize}

\item \textbf{Problemes amb la traducció}
  \begin{itemize}
  \item Bloqueig del ribosome binding site degut a la formació
    d'estructures secundàries.
  \item Triplets anteriors a AUG: si són codons stop o UAU, CUU) no es
    tradueix.
  \item Triplets posteriors: seqüències riques en A i U provoquen que
    el ribosoma ho interpreti com un terminador.
  \item Presència d'aminoàcids rars: aminoàcids que els bacteris
    utilitzen poc. Es pot solucionar afegint aquests aminoàcids al
    medi de cultiu.
  \item Ús esbiaixat de codons.
  \end{itemize}

\item \textbf{Problemes amb la proteïna}
  \begin{itemize}
  \item Inestabilitat de la proteïna: ús de soques deficients en proteases.
  \item Plegaments incorrectes: es pot solucionar co-transformant amb
    xaperones o utilitzant cèl·lules eucariotes.
  \item Manca de processament post-traduccional: ús de cèl·lules eucariotes.
  \item Formació de cossos d'inclusió: es pot solucionar modificant la
    temperatura, el temps d'inducció, el medi de cultiu o fer servir
    una altra soca d'\textit{E. coli}. També es pot incloure tags de
    fusió de secreció o delecionar els extrems amino i carboxi de la
    proteïna.
  \end{itemize}
\end{itemize}

\subsubsection{Expressió heteròloga en llevats}
\label{sec:expr-heter-en}

Inclou un gen de selecció Leu. Es fan servir llevats deficitaris en la
síntesi de Leu; de manera que només els llevats que adquireixen el
vector sobreviuen en un medi auxotròfic per Leu.

L'ús de llevats permet clonar fragments més grans de DNA.

S'han desenvolupat YAC, que tenen el mateix comportament genètic i
admenten fragments de DNA molt més grans. Presenten centròmer,
telòmers. Porta origen de replicació bacterià i gen de resistència a
ampicil·lina per si de cas es vol mantenir en bacteris.

\subsubsection{Clonació de la insulina}
\label{sec:clonacio-de-la}

Clonació de la insulina: es va clonar primer la cadena B del genoma
humà, després la cadena A i es van clonar les 2 en vectors diferents i
transformar en 2 cultius diferents. Els bacteris degradaven la
proteïna i es va decidir fusionar amb la beta-gal. La proteïna es
tallava amb bromur de cianogen i es separaven les cadenes A i B, i
mesclant les 2 cadenes en condicions adequades feien els ponts
disulfur i s'obtenia insulina funcional. Quan la insulina s'injecta al
múscul, difon lentament cap a la sang; però al múscul s'agregava i
cristal·litzava. Es van identificar els aminoàcids que interaccionaven
per cristal·litzar i es van canviar per Asp sense canviar la funció de
la insulina.

\subsection{Mutagènesi dirigida}
\label{sec:mutagenesi-dirigida}

S'han desenvolupat tècniques d'enginyeria genètica que permeten
canviar els aminoàcids codificats per un determinat gen. D'aquesta
manera és possible generar proteïnes recombinants amb característiques
adients per a determinades aplicacions industrials o
terapèutiques. Aquestes tècniques, conegudes globalment com Sistemes
de Mutagènesi Dirigida, són la base de l'Enginyeria de Proteïnes
actual.

\begin{itemize}
\item Millorar l'eficiència catalítica d'un enzim
\item Incrementar la termoestabilitat o la tolerància a diferents pH
  d'una proteïna
\item Canviar les condicions en les que un enzim és actiu
\item Modificar els requeriments de cofactors d'un enzim
\item Millorar el rendiment en la producció d'una proteïna,
  incrementant la seva resistència a les proteases.
\end{itemize}

\subsubsection{Heterodúplex}
\label{sec:heteroduplex}
Es clona un gen en un vector amb origen f1, s'activa f1 i quan es
replica, es genera un plasmidi de cadena senzilla i es fa un primer
que inclogui la mutació que es vol introduir.  Llavors queda un
heterodúplex (1 cadena normal i l'altra mutada). Quan el bacteri
repliqui el plàsmid hi haurà 2 poblacions: 1 amb el plasmidi normal i
l'altre el plasmidi mutat.

La selecció es fa amb l'ús de 2 primers en la mutació i un plàsmid amb
el gen ampR mutat. El primer contindrà la mutació corregida, de manera
que el plàsmid que contingui la mutació serà resistent a ampicil·lina
i el plàsmid no mutat no s'amplificarà en els bacteris.

\subsubsection{Reemplaçament de cassette}
\label{sec:reempl-de-cass}

La versió actual és per PCR amb megaprimer. Es dissenya un primer que
hibridi a la zona on es vol introduir la mutació i que contingui
aquests mutació (primer mutagènic). Es dissenya un altre primer a
l'extrem més proper i es fa una PCR amb els 2 primers.

La cadena curta resultant que conté la mutació es fa servir com a
megaprimer, que anellarà a un extrem i es fa servir un primer de
l'altre extrem per amplificar tot el gen.