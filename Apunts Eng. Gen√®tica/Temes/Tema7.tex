%--------------------------------------------------------
% Tema 7. Caracterització de l'expressió gènica: trànscrits
%--------------------------------------------------------
\section{Caracterització de l'expressió gènica: trànscrits}
\label{sec:caract-de-lexpr}

Hi ha diferents tècniques per estudiar els trànscrits:
\begin{itemize}
\item Northern blot
\item Dot(slot)blot
\item RT-PCR
\item Hibridació in situ
\item Microxips
\item SAGE
\item Gene reporter
\item Promoter bashing
\item Gel retardation (Electrophoretic mobility shift assay)
\end{itemize}

\subsection{Southern Blot}
\label{subsec:northern-blot}
Abans, però el \textbf{Southern Blot} es basa en:
\begin{enumerate}[\itembolasazules{\arabic*}]
\item Fragmentació del DNA amb endonucleases + gel d'agarosa
\item Desnaturalització amb tractaments alcalins (NaOH)
\item Transferència a un filtre. El tampó difon pel gel d'agarosa fins
  al filtre, on arrosega el DNA per capil·laritat i el fixa. El filtre
  pot ser de nitrocel·lulosa o de niló.
\item Bloqueig del filtre i fixa el DNA amb UV.
\item Prehibridació amb DNA inespecífic per cobrir regions sense DNA i
  evitar falsos positius.
\item Hibridació de la sonda (s'ha de conèixer la seqüència)
\item Detecció de la sonda: Amb un anticòs que reconeix la molècula
  marcadora, l'anticòs té unit un enzim.
\end{enumerate}

\subsubsection{Control de l'astringència}
\label{sec:contr-de-lastr}
L'astringència fa referència al grau d'exigència a la
complementarietat entre la diana i la sonda. Què succeirà si hi ha
algun mismatch?? Aquest fet es pot modular experimentalment. Els
paràmetres són:
\begin{itemize}
\item Temperatura: Com més elevada sigui la temperatura, més elevada
  serà l'astringència.

\item Concentració de cations monovalents: Com més alta sigui la
  concentració, la hibridació serà menys astringent. 

\item Solvents orgànics (formamida): Com més solvents orgànics, més astringent
  serà la hibridació. Aquest paràmetre és estàndard (normalment el 50\%).
\end{itemize}

Si volem ser poc exigents, s'ha fer fer a temperatura baixa i alt
contingut en \ch{Na+}.


\subsection{Northern Blot}
\label{sec:northern-blot-1}
El \textbf{Northern Blot} permet transferir RNA d'un gel d'agarosa a
una membrana de nitrocel·lulosa o niló. El procediment és equivalent
al Southern.

Alguns punts importants:
\begin{itemize}
\item L'extracció de RNA és molt delicada. Hi ha RNases a les mans, a
  les superfícies...
\item Desnaturalització per evitar estructures secundàries que són
  típiques del tRNA però poden passar a mRNA.
\item Gel i transferència desnaturalitzants
\item Sonda: disseny i marcatge
\item Condicions d'astringència
\item Controls positius i negatius
\item Mai es digereix amb enzims de restricció
\end{itemize}

El Northern es fa servir per determinar si s'expressa un gen en un
moment o teixit concret, si hi ha un mRNA corresponent a una sonda, la
presència relativa del transcrit en diferents teixits o moments o
comprovar la mida real del mRNA.

Exemple 33: Desenvolupament de Drosophila. Diferents variants
d'splicing al llarg del desenvolupament. Control positiu -> gen
ribosomal.

Els dot blots de RNA es basen en aplicar la mostra de RNA sobre la
membrana i es fixa i s'hibrida. Serveixen per detectar la presència
d'un RNA sense separar per mida.