%--------------------------------------------------------
% Tema 7. Caracterització de l'expressió gènica: trànscrits
%--------------------------------------------------------
\section{Caracterització de l'expressió gènica: trànscrits}
\label{sec:caract-de-lexpr}

Hi ha diferents tècniques per estudiar els trànscrits:
\begin{itemize}
\item Northern blot
\item Dot(slot)blot
\item RT-PCR
\item Hibridació in situ
\item Microxips
\item SAGE
\item Gene reporter
\item Promoter bashing
\item Gel retardation (Electrophoretic mobility shift assay)
\end{itemize}

\subsection{Southern Blot}
\label{subsec:northern-blot}
Abans, però el \textbf{Southern Blot} es basa en:
\begin{enumerate}[\itembolasazules{\arabic*}]
\item Fragmentació del DNA amb endonucleases + gel d'agarosa
\item Desnaturalització amb tractaments alcalins (NaOH)
\item Transferència a un filtre. El tampó difon pel gel d'agarosa fins
  al filtre, on arrosega el DNA per capil·laritat i el fixa. El filtre
  pot ser de nitrocel·lulosa o de niló.
\item Bloqueig del filtre i fixa el DNA amb UV.
\item Prehibridació amb DNA inespecífic per cobrir regions sense DNA i
  evitar falsos positius.
\item Hibridació de la sonda (s'ha de conèixer la seqüència)
\item Detecció de la sonda: Amb un anticòs que reconeix la molècula
  marcadora, l'anticòs té unit un enzim.
\end{enumerate}

\subsubsection{Control de l'astringència}
\label{sec:contr-de-lastr}
L'astringència fa referència al grau d'exigència a la
complementarietat entre la diana i la sonda. Què succeirà si hi ha
algun mismatch?? Aquest fet es pot modular experimentalment. Els
paràmetres són:
\begin{itemize}
\item Temperatura: Com més elevada sigui la temperatura, més elevada
  serà l'astringència.

\item Concentració de cations monovalents: Com més alta sigui la
  concentració, la hibridació serà menys astringent. 

\item Solvents orgànics (formamida): Com més solvents orgànics, més astringent
  serà la hibridació. Aquest paràmetre és estàndard (normalment el 50\%).
\end{itemize}

Si volem ser poc exigents, s'ha fer fer a temperatura baixa i alt
contingut en \ch{Na+}.


\subsection{Northern Blot}
\label{sec:northern-blot-1}
El \textbf{Northern Blot} permet transferir RNA d'un gel d'agarosa a
una membrana de nitrocel·lulosa o niló. El procediment és equivalent
al Southern.

Alguns punts importants:
\begin{itemize}
\item L'extracció de RNA és molt delicada. Hi ha RNases a les mans, a
  les superfícies...
\item Desnaturalització per evitar estructures secundàries que són
  típiques del tRNA però poden passar a mRNA.
\item Gel i transferència desnaturalitzants
\item Sonda: disseny i marcatge
\item Condicions d'astringència
\item Controls positius i negatius
\item Mai es digereix amb enzims de restricció
\end{itemize}

El Northern es fa servir per determinar si s'expressa un gen en un
moment o teixit concret, si hi ha un mRNA corresponent a una sonda, la
presència relativa del transcrit en diferents teixits o moments o
comprovar la mida real del mRNA.

Una gota que conté la molècula per ser detectada s'aplica directament
sobre una membrana. Això és seguit per la detecció per sondes de
nucleòtids (northern blot i southern blot).

La tècnica ofereix importants estalvis en temps. No obstant això, no
ofereix informació sobre la mida de la biomolècula blanc. A més, si
dues molècules de diferents mides són detectades, se seguirà veient
com un sol punt. Per tant el dot blot només pot confirmar la presència
o absència d'una biomolècula o biomolècules que poden ser detectats
per les sondes de DNA.

\subsection{Reverse Transcriptase PCR}
\label{sec:reverse-transcr-pcr}
És una tècnica semiquantitiva. Es comença amb RNA que es retrotranscriu a
DNA fent un únic cicle d'amplificació amb primers oligo-dT. Es
segueix amb una PCR convencional amb un primer que hibridi a l'altre
extrem de cua de poli-A per obtenir un DNA de doble cadena. Es pot
amplificar des d'1 pg de RNA.

Els resultats s'analitzen en un gel d'agarosa. Permet la determinació
del nivell d'expressió d'un RNA particular.

Els punts importants són:
\begin{itemize}
\item L'extracció de RNA és molt delicada
\item Disseny dels primers
\item Condicions d'astringència
\item Controls positius i negatius
\end{itemize}

La tècnica permet amplificar gens evitant els introns per seqüenciar,
clonar en bacteris, etc.

\subsection{Hibridació \textit{in situ}}
\label{sec:hibridacio-situ}
Consisteix en la detecció d'una seqüència específica de RNA sobre
talls histològics o sobre un organisme sencer. L'objectiu és detectar
el lloc d'expressió dels gens (ubicació dins de la cèl·lula, tipus
cel·lulars, teixits, regió de l'organisme).

El procés és:
\begin{enumerate}
\item Fixació del material biològic per tal de mantenir l'estructura
  del teixit.
\item Permeabilització per permetre l'entrada de la sonda.
\item Disseny de la sonda i marcatge.
\item Pre-hibridació per emmascarar els llocs inespecífics.
\item Hibridació.
\item Rentats.
\item Detecció del senyal i reconeixement de les estructures. La
  detecció es fa amb anticossos units a fluorocroms o revelats
  enzimàtics. Els fluorocroms permeten detectar múltiples gens i fer
  estudis de co-expressió. El més sensible és l'ús dels enzims.
\end{enumerate}

El control negatiu es fa un vector amb 2 promotors que vagin en
sentits diferents de manera que es generi una sonda \textit{sense} i
una altra \textit{antisense}. La sonda \textit{sense} no hibridarà amb
res i serà el control negatiu.

\subsection{Microxips}
\label{sec:microxips}
Permet l'anàlisi dels patrons d'expressió gènica de forma
massiva. Permet detectar tots els gens que s'estan expressant en un
determinat moment o teixit. Permet establir connexions entre teixits,
cèl·lules.

Els microxips consisteixen un una hibridació amb moltes sondes de
forma simultània. Les sondes són seqüències de gens coneguts. Les
sondes estan fiuxades en una matriu sòlida (filtre de niló o
portaobjectes) en un ordre conegut. El RNA de l'organisme o cèl·lules
que es volen analitzar es transcriuen a cDNA, els quals es marquen amb
fluorescència.

\subsection{Estudi de promotors i zones reguladores}
\label{sec:estudi-de-zones}
