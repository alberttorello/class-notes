%--------------------------------------------------------
% Seminari 6. Interferència d'RNA
%--------------------------------------------------------
\section{Interferència d'RNA}
\label{sec:interferencia-drna}

A. Fire va fer els experiments amb \textit{C. elegans} un nemàtode de 951 cèl·lules. El llnatge de C. elegans està ja determinat.

Abans de iRNA el que es feia era injectar un antisense del mRNA del gen que es volia silenciar. Els efectes passaven a la descendència dels C. elegans.

Quan es feia síntesi de mRNA a partir de polimerases de fags T7 i SP6 van pensar que es produïa també RNA de doble cadena.

Quan injectaven RNA de cadena senzilla sense o antisense, la descendència era wild-type.
Quan ho feien injectant sense i antisense alhora aconseguien un silenciament que perdurava a la descendència i que semblava una mutació de pèrdua de funció. Van estudiar el gen upc22, que es troba als miofilaments i els animals presentaven moviments espasmòdics. Els RNA complementaris a promotots o introns no tenia cap efecte. El RNA havia de ser complementari a mRNA.

Fire et al 1998 Nature 391 806-811

Efecte del RNAi mex3 on endogenous expression: a) control sense, b) adult, c) descendència , d) embrió que prové d'un adult injectat amb RNAi. 

El RNA de doble cadena a la cèl·lula, DICER el talla en fragments de 19-20 nt, després es separen en cadenes simples i s'uneixen als complexes RISC. Per complementació de bases, RISC s'hibridarà a mRNA endògens i els degradarà. Un cop el RNA de RISC es degradi, el gen començarà a expressar-se de nou.

Endògenament, el sistema és una defensa contra virus de dsRNA.

La tècnica permet fer KD sense animals mutants.

Els RNAi es poden injectar a l'abdomen d'una larva.

Una altra manera d'introduir RNAi és a través de l'alimentació. Com que \textit{C .elegans} mengen \textit{E. coli}, es pot posar el RNA en un plàsmid dins d'\textit{E. coli}.