%------------------------------------------------------------------------------
% Tema 5. Genètica de la obesitat
%------------------------------------------------------------------------------
\section{Genètica de la obesitat monogènica}
\label{sec:genetica-de-la}

\subsection{Obesitat monogènica}
\label{subsec:obesitat-monogenica}
Les nenes deficients en leptina tenen problemes de maduració sexual i per tant de fertilitat.

Els pacients deficients en MC4R (receptor de melanocortina), que són heterozigots per la mutació. Presenta obesitat a la infància, hiperfàgia, hiperinsulinèmia.

Hi ha 2 tipus de pèptids que controlen la sacietat quan senyalitzen a l'hipotàlem:
\begin{itemize}
\item Orexigènics Inhibits per la leptina
\item Anorexigènics: Estimulats per la leptina. El gen de la POMC quan s'expressa genera un polipèptid de 31 kDa, que en funció de les cèl·lules on s'expressa la proteïna es trenca en diferents fragments que seran els que tenen l'activitat biològica. Aquests fragments poden ser la ACTH, alpha-MSH... alpha-MSH atura la gana, interaccionant amb un receptor hipotalàmic. Quan el nen té dèficits en POMC, es poden donar agonistes pel receptor hipotalàmic.
\end{itemize}

\subsection{Obesitat multifactorial}
\label{subsec:obes-mult}

% tipus d'estudi per identificar gens de susceptibilitat

El gen FTO té una variant que s'associa molt potentment a un BMI alt. En canvi, variants polimòrfiques de la leptina no tenen associació amb la obesitat. BDNF fa de connexió entre l'hipotàlem i l'escorça per generar la sensació conscient de gana/sacietat.

FTO se sabia que tenia un NLS, un domini de demetilasa. FTO es troba en una regió genòmica amb clústers d'IRX (iroguis-related). Introns del FTO regulen els enhancers del gen IRX; el grau d'interacció condiciona l'expressió d'IRX3. IRX3 re demetilasagula la despesa energètica. Els al·lels FTO de risc per obesitat que quan interaccionen amb IRX3 augmenten l'expressió d'IRX3, i IRX3 disminueix la despesa energètica.