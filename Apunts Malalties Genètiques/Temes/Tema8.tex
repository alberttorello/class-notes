%------------------------------------------------------------------------------
% Tema 8. Heterogeneïtat gènica. Retinitis pigmentosa
%------------------------------------------------------------------------------
\section{Heterogeneïtat gènica. Retinitis pigmentosa}
\label{sec:heter-genic-reti}

Grup de patologies hereditàries de la retina que es caracteritzen per:
\begin{itemize}
\item Ceguera nocturna
\item Degeneració progressiva dels foto-receptors
\item Afectació severa de la visió
\end{itemize}

A la retina s'observa una que el disc òptic és pàlid, hi ha
anastomosis dels vasos retinians i es diposita pigment.

Els primers símptomes són la pèrdua de visió perifèrica.

En la síndrome de Bardet-Biedl hi ha implicats 2-3 gens per tal que es
produeixi la malaltia.

Definicions d'heterogeneïtat:
\begin{itemize}
\item Heterogeneïtat gènica o de locus: Diferents gens causen una
  mateixa malaltia.
\item Heterogeneïtat al·lèlica amb heterogeneïtat clínica: Diferents
  al·lels d'1 mateix gen causa diferents malalties.
\end{itemize}

% Fototransducció

Com que hi pot haver molts gens candidats el que s'ha fet és buscar
gens que participin de la fototransducció. Es poden analitzar gens que
tinguin una funció compatible amb l'afectació de la patologia.

L'alternativa és buscar gens per anàlisi de lligament. Es necessiten
tots els membres d'un pedigrí en el que segregui la malaltia. Els
pedígris han de ser grans per obtenir resultats prou potents.

Com que hi ha una gran heterogeneïtat gènica, no es poden sumar
famílies.

Es van descriure primer gens lligats al X, i després formes
autosòmiques dominants.

S'han descrit moltes mutacions al gen de la rodopsina.

Les causes moleculars de la dominància són: 
\begin{itemize}
\item Haploinsuficiència: Situació en què el 50\% de la dosi gènica no
  és suficient per la funció correcta del gen.

\item Mutació amb guany de funció: És aquella mutació que produeix un
  fenotip diferent per l'adquisició de noves propietats.

\item Dominant negatiu: És aquella que dóna lloc a un producte que pot
  inhibir la funció del producte WT a l'heterozigot. Sol passar en
  proteïnes que formen part de complexos multiproteics.
\end{itemize}

En el cas de la rodopsina, la dominància sigui deguda a una forma
dominant negatiu o a un guany de funció.

L'any 1994 es van trobar els primers gens que eren autosòmics
recessius.

En el cas de pedígris petits on no es pot fer lligament, es busca
entre els candidats funcionals.

Es genotipen 3 polimorfismes en tots els pedígris i es seleccionen
aquells que segreguen. Es seqüencien els pacients que cosegragaven
polimorfismes amb la malaltia. Els problemes d'aquest mètode és la
inestabilitat. 

ABCA4 pot provocar una altra malaltia. 

% Mutacons digèniques: 2 gens heteroxigots causen la malaltia.