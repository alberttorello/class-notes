%------------------------------------------------------------------------------
% Tema 9. Malalties cromosòmiques
%------------------------------------------------------------------------------

\section{Malalties cromosòmiques}
\label{sec:malalt-crom}

Els humans no tenim cromosomes telocèntrics. La zona stalk codifica per rRNA, i estan repetitis per tots els cromosomes.

L'índex centromèric mesura la lingitud del braç curt respecte el llarg.

L'anueploïdia no és compatible amb la vida, però les cèl·liles són mosaic de l'inividu.

%% Disomia uniparntetals

\subsection{Disomia uniparental}
\label{sec:disomia-uniparental}

El nombre de cromosomes és normal però per un parell concret els 2 cromosomes homòlegs probenen del mateix progenitor.

Les conseqüències dependran de si hi ha reions d'imprintinh o hi ha mutacions causants de patologies. Isodisonimies i anisodisomies.


\subsection{Aberracions estructurals}
\label{sec:aberr-estr}

Duplicació, inversió (pericèntrica i paracèntrica), 

Les duplicacions augmenten la dosi gènica....

Les inversions mantenen la dosi gènica. El problema és si en la inversió es trenca inm gen o no, i es perden els enhancers.

Insercions

Isocromosoma:  com isocromosomes, per localitzar les proteïnes en una preparació de nuclis

Cromosomes derivatius

Translocació robertsoniana: translocació entre 2 cromosomes acrocèntrics. Es perden els braços curts dels cromosomes. No té efecte a llarg termini.

La translocació robertsoniana t(21, 14) és balancejada no té efectes fenotípics i es transmet a la descendència que poden ser síndrome de Down.