%------------------------------------------------------------------------------
% Tema 6. Malalties lligades al cromosoma X
%------------------------------------------------------------------------------
\section{Malalties lligades al cromosoma X}
\label{sec:malalt-llig-al}

Degut a la ``inactivació aleatòria'' del cromosoma X, hi pot haver dones
heterozigotes per un gen recessiu a X que presenti la malaltia ja que
s'ha silenciat cromosoma ``sa''.

Criteris per la dominància:
\begin{itemize}
\item Individus afectats en cada generació
\item Cada individu té un progenitor afectat (excepte mutacions
  \textit{de novo})
\item Transmissió de mascle a mascle
\end{itemize}

La consanguinitat afavoreix l'aparició de mutacions recessives en
homozigosi. El risc és d'1/4 per cada fill.

Heterozigots compostos: Els 2 afectats tenen 2 al·lels mutats però amb
mutacions diferents.

Per malalties lligades al X, totes les filles d'un pare afectat seran
portadores. Són més incidents en homes. Els fills d'una dona portadora
tenen 1/2 de ser afectats.

\subsection{Distròfia muscular de Duchenne}
\label{sec:distr-musc-de}
El gen de la distrofina és el gen més gran dels humans. La malaltia
està provocada per delecions en el gen de la distrofina.

És una malaltia molt greu i sense tractament. És de les més prevalents
de les malalties rares.

El gen es va aïllar per clonació posicional.

% Característiques clíniques i marcadors

1 en 3500 naixementss de mascles. 1/3 de les mutacions són \textit{de novo}.