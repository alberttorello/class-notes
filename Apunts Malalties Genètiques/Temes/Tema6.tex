%------------------------------------------------------------------------------
% Tema 6. Malalties lligades al cromosoma X
%------------------------------------------------------------------------------
\section{Malalties lligades al cromosoma X}
\label{sec:malalt-llig-al}

Degut a la ``inactivació aleatòria'' del cromosoma X, hi pot haver dones
heterozigotes per un gen recessiu a X que presenti la malaltia ja que
s'ha silenciat cromosoma ``sa''.

Criteris per la dominància:
\begin{itemize}
\item Individus afectats en cada generació
\item Cada individu té un progenitor afectat (excepte mutacions
  \textit{de novo})
\item Transmissió de mascle a mascle
\end{itemize}

La consanguinitat afavoreix l'aparició de mutacions recessives en
homozigosi. El risc és d'1/4 per cada fill.

Heterozigots compostos: Els 2 afectats tenen 2 al·lels mutats però amb
mutacions diferents.

Per malalties lligades al X, totes les filles d'un pare afectat seran
portadores. Són més incidents en homes. Els fills d'una dona portadora
tenen 1/2 de ser afectats.

\subsection{Distròfia muscular de Duchenne}
\label{sec:distr-musc-de}
El gen de la distrofina és el gen més gran dels humans. La malaltia
està provocada per delecions en el gen de la distrofina.

És una malaltia molt greu i sense tractament. És de les més prevalents
de les malalties rares.

El gen es va aïllar per clonació posicional.

% Característiques clíniques i marcadors

1 en 3500 naixementss de mascles. 1/3 de les mutacions són \textit{de novo}.

El gen té promotors alternatius.

El gen es va identificar per clonatge posicional. No s'utilitza informació de la proteïna sinó que es fan servir mapes físics i genètics.

El gen es va localitzar mitjançant el cariotip de translocacions i delecions prou grans. També va incloure anàlisi de lligament.

Per clonar-lo es van seguir 2 estratègies:
\begin{itemize}
\item Toronto: Clonació del punt de ruptura.

\item Clonació dels fragments del X absents en la deleció del nen B.B.
\end{itemize}

Es va fragmentar el DNA del nen B.B i es va aïllar DNA de cèl·lules amb polisomia de X i es va digerir amb MboI. Es van barrejar les mostres i hibridar en un tampó aquós. Els fragments que s'hibridaven contenien el gen DMD i van ser els que es van clonar en un vector.

Els diferentrs promotors donen les diferents isoformes dels teixits.

L'extrem N-terminal s'uneix a l'actina i el C-terminal s'uneix al complex DPC. La proteïna té repeticions del tipus espectrina.

El complex DPC comprèn 3 complexes:
\begin{enumerate}
\item Complex de sarcoglicans: És específic de múscul. Expressió i tràfic coordinat dependent de distrofina. Les mutacions en aquest complex causen distròfia de cintura.
  
\item Complex de distroglicans: Expressió ubiqua. Tall proteolític i glicosilació de la proteïna precursora. Connecta la matriu extracel·lular amb les estructures de distrofina. Té una funció en la transducció de senyal derivada de l'adhesió a la matriu extracel·lular.
  
\item Complex citoplasmàtic:
\end{enumerate}

La utrofina té una mida similar a la distrofina. És d'expressió ubiqua. Es troba al complex DPC de la membrana post-sinàptica de la placa neuromuscular.

\subsection{Mutacions a DMD}
\label{sec:mutacions-DMD}
Un 65\% té delecions al gen DMD. Un 5\% presenten duplicacions d'1 o més exons. La resta tenen mutacions puntuals.

El gen té punts calents de mutació, zones més susceptibles a ser mutades.

La BDMD (distròfia muscular de Becker) és menys greu que la DMD. La incidència és de 1/30.000 en nens. La debilitat muscular comença més tard i la progressió és molt variable. Està causada per mutacions al gen de la distrofina. La mida de les delecions no explicava la gravetat tant variable. Les delecions de Becker estan en pauta i les de Duchenne no. En Becker es mantenen els extrems de la proteïna, i pot fer més o menys la seva funció.

\begin{itemize}
\item Heterogeneïtat gènica o de locus: Diferents gens provoquen una mateixa malaltia. Retinitis pigmentosa.
  
\item Heterogeneïtat al·lèlica: 1 únic gen amb diversos al·lels produeix la mateixa malaltia.
  
\item Heterogeneïtat clínica: Diferents al·lels d'1 gen provoquen diferents malalties.
\end{itemize}

\subsection{Diagnòstic genètic}
