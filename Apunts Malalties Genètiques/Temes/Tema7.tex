%------------------------------------------------------------------------------
% Tema 7. Malalties dominants lligades al cromosoma X
%------------------------------------------------------------------------------
\section{Malalties dominants lligades al cromosoma X}
\label{sec:malalt-domin-llig}

\subsection{Síndrome de l'X fràgil}
\label{sec:x-fragil}

Els símptomes són:
\begin{itemize}
\item Discapacitat mental amb problemes d'aprenentatge
\item Dèficit d'atenció
% ...
\end{itemize}

L'any 1970 es va descrobrir un marcador citogenètic en X-fràgil.

Es parla de dominància perquè les dones heterozigotes estan afectades. La penetrància no és completa: 80\% en homes i 35\% en dones.

El gap al cromosoma X només es detecta en el 50\% dels afectats homes i 30\% de les afectades.

Les illes CpG són regions de DNA de menys de 1kb que conté molts dinucleòtids CpG desmetilats. Correspon a gens actius. A la regió 5'-UTR del gen FMR1 hi ha una un $(CGG)_{30}$. La mutació que causa X-fràgil es basa en l'expansió d'aquest triplet. Hi ha una situació de pre-mutació, que no és plenament normal però tampoc patològica. 

La illa CpG es metila i el gen es silencia. En la pre-mutació hi ha un augment de la transcripció però no de traducció. La pre-mutació pot donar alguns símptomes.

La tremolor i atàxia cerebel·losa en les homes està causat per pre-mutació en el gen FMR1. 

En aquesta malaltia hi ha anticipació (quan salta de generació, es
torna més greu).

Hi ha un gran nombre de portadors amb pre-mutació. Les premutacions
van passant de generació en generació. Els homes passen a les filles
la premutació. De fet, els gàmetes masculins d'homes malalts o amb
pre-mutació sempre contenen al·lels premutats.

La mutació total mostra inestabilitat mitòtica. Els pacients amb
fràgil X, són mosaics. Els mosaics poden ser:
\begin{itemize}
\item Mosaics de tamany
\item Mosaics de metilació
\end{itemize}

Les repeticions CGG pures són més inestables que les que tenen algun
AGG intercalat diferent. En la població normal, molts individus porten
aquesta distribució: $(CGG)_{10} - AGG- (CGG)_{10} - AGG- (CGG)_{10}$.

Els cromosomes d'individus normals es poden agrupar en 3
haplotips. L'haplotip 3 inclou les repeticions més llargues de
CGG. L'haplotip 1 és el més freqüent, inclou uns quants casos en que
només hi ha el primer triplet AGG, i això el faria més inestable.

En repeticions llargues i pures, la polimerasa pot lliscar.

L'exó 1 del gen FMR1 es troba a la regió 5'-UTR (abans del lloc
d'inici de la traducció). L'expansió de nucleòtids es metila
\textit{de novo} i la metilació segueix fins al promotor del gen i el
silencia.

\begin{enumerate}
\item Mecanisme de metilació \textit{de novo}:
  \begin{enumerate}
  \item  L'expansió de CGG pot
  formar estructures secundàries que són reconegudes per
  metiltransferases DNMT1.
  \item L'expansió semblaria un DNA paràsit
  (retrovirus o transposó).
  \end{enumerate}

\item Mecanisme del silenciament: Silenciaciament degut a
  desacetilació d'histones. Els CGG metilats són reconeguts per MecP2
  (síndrome de Rett) que forma un complex amb Sin3 i recluta
  desacetilases d'histones. La desacetilació de H3 i H4 comporta la
  condensació de cromatina i la repressió transcripcional. 
\end{enumerate}

La proteïna FMR1 és d'expressió ubiqua, sobretot en cervell. Conté 2
motius d'unió al RNA:
\begin{itemize}
\item 2 dominis amb homologia a la ribonucleoproteïna K (dominis KH)
\item Agrupacions de residus Gly i Arg (caixes RGG)
\end{itemize}

És una proteïna d'unió al mRNA. Presenta NLS i NES (entrada i sortida
al nucli). Participa en el transport i silenciament de mRNA del soma
cap a la dendrita. Quan  hi ha una senyal, receptor metabotròpic de
glutamat, hi ha una desfosforilació de FMR1 i es permet la traducció
dels mRNA segrestats.

FMRP inhibiria la unió de factors de traducció com eIF4 o bé impediria
el pas dels ribosomes al llarg del mRNA. Es creu que FMRP pot reclutar
miRNA.